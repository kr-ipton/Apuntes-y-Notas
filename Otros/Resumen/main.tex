\documentclass{article}
\usepackage[utf8]{inputenc}

\title{Resumen de la película "Pulp fiction"}
\author{Juárez Torres Carlos Alberto}
\date{3 de Febrero del 2023}

\begin{document}

\maketitle

\section*{Resumen}

"Pulp Fiction" es una película de 1994 dirigida por Quentin Tarantino y con un elenco que incluye a John Travolta, Uma Thurman, Samuel L. Jackson, Harvey Keitel y Tim Roth. La trama de la película sigue varios arcos narrativos que se entrelazan en una estructura no lineal.\\

 El primer arco de la película sigue la historia de Vincent Vega (interpretado por John Travolta) y Jules Winnfield (interpretado por Samuel L. Jackson), dos asesinos profesionales que trabajan para el gángster Marsellus Wallace (interpretado por Harvey Keitel). La trama del primer arco se centra en la dinámica de la relación entre Vincent y Jules, así como en la ejecución de sus misiones para Marcellus.\\

 La primera misión que realizan Vincent y Jules consiste en recuperar una maleta que ha sido robada a Marcellus. La misión lleva a una serie de eventos inesperados y violentos, incluyendo una pelea con un grupo de bandidos armados y la muerte de varios de ellos.\\

La segunda misión de Vincent y Jules es proteger a Mia Wallace (interpretada por Uma Thurman), la esposa de Marcellus. Vincent acompaña a Mia a un club de baile y a un restaurante, donde sucede un evento inesperado que pone en peligro la vida de Mia y Vincent.\\

A lo largo del primer arco, se muestra la relación cercana y amistosa entre Vincent y Jules, así como su profesionalismo y competencia en su trabajo como asesinos. También se destaca el estilo único y el diálogo inteligente que caracteriza la película.\\

El segundo arco sigue a Butch Coolidge (interpretado por Bruce Willis), un boxeador que está planeando huir de la ciudad después de ganar una pelea. Sin embargo, Butch se ve obligado a hacer un desvío en su plan después de prometer a Marcellus Wallace que perderá una pelea importante.\\

A pesar de perder la pelea como se había acordado, Butch escapa y regresa a su apartamento para recoger sus pertenencias y huir de la ciudad. Sin embargo, se da cuenta de que ha olvidado su reloj de oro en su apartamento y regresa a recogerlo, lo que desencadena una serie de eventos peligrosos y violentos que ponen en peligro la vida de Butch y su pareja.\\

A lo largo del segundo arco, se muestra el carácter astuto y valiente de Butch, así como su deseo de huir y vivir una vida normal. También se destaca la combinación de acción y humor en la trama, así como la narrativa no lineal y la interconexión entre los diferentes arcos de la película.\\

El tercer arco sigue a Vincent Vega y Mia Wallace mientras van a un restaurante en una cita. La cena termina en una serie de eventos inesperados que incluyen un robo en el restaurante y un accidente de drogas.\\

A lo largo del tercer arco, se muestra el romance incipiente entre Vincent y Mia, así como la química entre los dos personajes. También se destaca el humor absurdo y la violencia en la trama, así como la narrativa no lineal que caracteriza la película en su conjunto.
\section*{Conclusión}

En general, "Pulp Fiction" es una película épica que combina elementos de la cultura popular, la violencia y el humor para crear una narrativa única y desafiante. Es considerada una de las películas más influyentes de la década de 1990 y ha sido ampliamente elogiada por su estilo visual y narrativo innovador.\\

Personalmente, me gustó mucho el modo narrativo de Quentin al mostrar la vida cotidiana de los seres humanos en diálogos, como en la escena de la cafetería con los atracadores o la discusión entre Vincent y Jules. Con detalles simples pero necesarios, hace que la historia sea verosímil en su contexto desenfrenado, y mantiene la atención del espectador por el desconocimiento de lo que sucederá al final de la conversación. Sin más que decir, es una película entretenida con un estilo interesante, pero no apta para todo tipo de público.

\end{document}
