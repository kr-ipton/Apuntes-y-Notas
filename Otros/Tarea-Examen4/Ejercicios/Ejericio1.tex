\textcolor{blue}{Ejercicio 1.} Dada la matriz $ A= \begin{pmatrix}
a & b\\
c & d
\end{pmatrix}$, donde $a,\: b,\: c\: \mbox{y} \: d$ son números reales, demuestre que el polinomio característico de $A$ se puede reescribir como $p(\lambda)= \lambda^2 -tr(A) \lambda + \det(A)$; donde $tr(A)$ es la traza de A  y $\det(A)$ el determinante de A.\\
Adicionalmente demuestre que todas las soluciones del sistema

$$ x'= \begin{pmatrix}
a & b\\
c & d
\end{pmatrix}x$$

Con a, b, c y d $\in \mathbb{R}$ tienden a cero cuando $t \rightarrow + \infty$ si y solo si $a + d < 0$ y $ad -bc > 0$\\

$\mathbf{Sol:}$ Dada $A= \begin{pmatrix} a & b\\ c & d
\end{pmatrix}$ tenemos que el polinomio característico está dado por $|A - \lambda I|=0$\\


$$\Rightarrow \begin{pmatrix} a & b\\ c & d \end{pmatrix} - \begin{pmatrix}  \lambda & 0 \\ 0 & \lambda \end{pmatrix} = 0$$

$$\Leftrightarrow \begin{pmatrix} a - \lambda  & b\\ c & d - \lambda
\end{pmatrix} = 0$$

$$\Leftrightarrow (a - \lambda)(d - \lambda) - bc = 0$$

$$ \Leftrightarrow \lambda^2 - (a + d)\lambda + ad - bc = 0$$

Por otra parte $tr(A)= a + d$; $det(A)= ad -bc$

$$\Rightarrow p(\lambda)= \lambda^2 - tr(A)\lambda + det(A)= \lambda^2 - (a+d)\lambda + ad -bc = |A - \lambda I|$$

$\therefore$ El polinomio característico se puede expresar como $p(\lambda)= \lambda^2 -tr(A)\lambda + det(A)$

Ahora falta por demostrar que todas las soluciones del sistema

$$x'= \begin{pmatrix}
a & b\\
c & d
\end{pmatrix}x$$

Con a, b , c y d $\in \mathbb{R}$ tiende a cero cuando $t \rightarrow + \infty$ si y solo si $a + d < 0$ y $ad - bc > 0$

$\Rightarrow$

$\mathbf{Sol:}$ Sea $x= e^{\lambda t}u$ solución, los eigenvalores de A son solución de: 

$$\lambda^2 - tr(A)\lambda + det(A)= 0$$

$$\Rightarrow  \lambda = \frac{tr(A) \pm \sqrt{tr(A)^2 - 4det(A)}}{2}$$

Así las soluciones de la ecuación son de la forma 

$$\begin{bmatrix}
x_1(t)\\
x_2(t)
\end{bmatrix}= 
C_1 e^{\lambda_1 t}\begin{bmatrix}
u_1(t)\\
u_2(t)
\end{bmatrix}+
C_2 e^{\lambda_2 t}\begin{bmatrix}
v_1(t)\\
v_2(t)
\end{bmatrix}$$

como $c_1, c_2, u_1, u_2, v_1, v_2 \in \mathbb{R}$ la única forma de que las soluciones tiendan a cero es que $\lambda_1, \lambda_2$ sean negativos, es decir: 

$$ i) tr(A) + \sqrt{tr(A)^2 -4det(A)} < 0 $$

$$ ii) tr(A) - \sqrt{tr(A)^2 -4det(A)} < 0 $$

sumando $i)$ y $ii)$ se debe tener que $tr(A)<0 \Rightarrow a+ d <0$

$\Leftarrow$

El regreso es análogo 

$\therefore$ las soluciones tienden a cero cuando $t \rightarrow + \infty$ si y solo si $a + d <0 $ y $ad -bc >0 $