\documentclass{article}
\usepackage[utf8]{inputenc}
\usepackage{amsmath,amssymb,amsthm}
\usepackage[spanish, mexico]{babel}
\usepackage{graphicx}
\usepackage[table,xcdraw]{xcolor}
\usepackage{float} 
\usepackage{wrapfig}
\usepackage{multirow, array} 
\usepackage{polynom}
\usepackage{parskip}
\usepackage[left=2.00cm, right=2.00cm, top=2.00cm, bottom=2.00cm]{geometry}
\usepackage{float} % para usar [H]

\graphicspath{.JPG}
\renewcommand{\qed}{$\blacksquare$}
\newcommand\tab[1][1cm]{\hspace*{#1}}
\newcommand{\ie}{\textit{i}.\textit{e}.}
\newcommand{\eg}{\textit{e}.\textit{g}.}
\providecommand{\norm}[1]{\lVert#1\rVert}


\begin{document}

\begin{titlepage}
	\centering
	
	{\scshape\LARGE Universidad Nacional Autónoma de México\par}
	\vspace{1cm}
	{\scshape\Large Facultad de Ciencias \par}
	{\huge\bfseries Tarea 2 \par}
	{\Large\itshape Análisis de Algoritmos \par}
	
	\begin{table}[ht]
	\centering
	\rowcolors{1}{pastelgreen}{pastelpink}
	\begin{tabular}{|l|l|}
	\hline
	\multicolumn{2}{|c|}{\cellcolor{pastelgreen}\textbf{Información del curso}} \\ \hline
	Profesor & María de Luz Gasca Soto \\
	Ayudante & Brenda Margarita Becerra Ruíz \\
	Ayudante & Enrique Ehecatl Hernández Ferreiro \\
	\hline
	\end{tabular}
	\end{table}

	{\large\itshape Autor: Juárez Torres Carlos Alberto \par}

	\begin{center}
		\begin{verbatim}
			______________1¶¶¶____¶¶¶1___¶¶¶1_________________
			_____________¶¶¶1___1¶¶1___1¶¶1___________________
			____________1¶¶1___1¶¶1___1¶¶1____________________
			____________1¶¶1___1¶¶1___1¶¶¶____________________
			_____________¶¶¶____¶¶¶1___¶¶¶1___________________
			______________¶¶¶¶___1¶¶¶___1¶¶¶__________________
			_______________1¶¶¶1___¶¶¶1___¶¶¶¶________________
			_________________1¶¶1____¶¶¶____¶¶¶_______________
			___________________¶¶1____¶¶1____¶¶1______________
			___________________¶¶¶____¶¶¶____¶¶¶______________
			__________________1¶¶1___1¶¶1____¶¶1______________
			_________________¶¶¶____¶¶¶1___1¶¶1_______________
			________________11_____111_____11_________________		Toma un café mañanero UwU
			__________¶¶¶¶¶¶¶¶¶¶¶¶¶¶¶¶¶¶¶¶¶¶¶¶¶¶¶¶¶¶¶¶________
			1¶¶¶¶¶¶¶¶¶¶¶__¶¶¶¶¶¶¶¶¶¶¶¶¶¶¶¶¶¶¶¶¶¶¶¶¶¶¶¶________
			1¶¶¶¶¶¶¶¶¶¶¶__1¶¶¶¶¶¶¶¶¶¶¶¶¶¶¶¶¶¶¶¶¶¶¶¶¶¶¶________
			1¶¶_______¶¶__1¶¶¶¶¶¶¶¶¶¶¶¶¶¶¶¶¶¶¶¶¶¶¶¶¶¶¶________
			1¶¶_______¶¶__1¶¶¶¶¶¶¶¶¶¶¶¶¶¶¶¶¶¶¶¶¶¶¶¶¶¶¶________
			1¶¶_______¶¶__¶¶¶¶¶¶¶¶¶¶¶¶¶¶¶¶¶¶¶¶¶¶¶¶¶¶¶¶________
			1¶¶_______¶¶__1¶¶¶¶¶¶¶¶¶¶¶¶¶¶¶¶¶¶¶¶¶¶¶¶¶¶¶________
			_¶¶¶¶¶¶¶¶¶¶¶__¶¶¶¶¶¶¶¶¶¶¶¶¶¶¶¶¶¶¶¶¶¶¶¶¶¶¶¶________
			_¶¶¶¶¶¶¶¶¶¶¶__¶¶¶¶¶¶¶¶¶¶¶¶¶¶¶¶¶¶¶¶¶¶¶¶¶¶¶¶________
			__________¶¶___1¶¶¶¶¶¶¶¶¶¶¶¶¶¶¶¶¶¶¶¶¶¶¶¶¶1________
			__________1¶¶___¶¶¶¶¶¶¶¶¶¶¶¶¶¶¶¶¶¶¶¶¶¶¶¶¶_________
			____________¶¶¶¶¶¶¶¶¶¶¶¶¶¶¶¶¶¶¶¶¶¶¶¶¶¶11__________
			11_____________________________________________111
			1¶¶¶¶¶¶¶¶¶¶¶¶¶¶¶¶¶¶¶¶¶¶¶¶¶¶¶¶¶¶¶¶¶¶¶¶¶¶¶¶¶¶¶¶¶¶¶¶1
			__¶¶111111111¶¶¶¶¶¶¶¶¶¶¶¶¶¶¶¶¶¶¶¶¶¶¶¶¶111111111¶__
			
		\end{verbatim}
	\end{center}
	
	{\large Fecha de entrega: \today\par}
	
\end{titlepage}

%Ejercicios%


\section{Para la siguiente matriz, calcula: }
\textbf{Soluci\'on 1:}\\
\[A  = \begin{pmatrix}
	1 & 1 & 1 \\
	2 & 3 & 4 \\
	5 & -8 & 10\end{pmatrix}= \begin{pmatrix}
	a_{1\hspace{0.5mm}1} & a_{1\hspace{0.5mm}2} & a_{1\hspace{0.5mm}3} \\
	a_{2\hspace{0.5mm}1} & a_{2\hspace{0.5mm}2} & a_{2\hspace{0.5mm}3} \\
	a_{3\hspace{0.5mm}1} & a_{3\hspace{0.5mm}2} & a_{3\hspace{0.5mm}3} \end{pmatrix}\]
\begin{enumerate}
\item[$a)$] Calculando $|A|$ por el m\'etodo de cofactores sobre el primer rengl\'on tenemos:
\[ \mid A\mid = a_{1\hspace{0.5mm}1}\left| \begin{array}{cc}	
a_{2\hspace{0.5mm}2} & a_{2\hspace{0.5mm}3}\\
a_{3\hspace{0.5mm}2} & a_{3\hspace{0.5mm}3}
\end{array} \right|- a_{1\hspace{0.5mm}2}\left| \begin{array}{cc}	
a_{2\hspace{0.5mm}1} & a_{2\hspace{0.5mm}3}\\
a_{3\hspace{0.5mm}1} & a_{3\hspace{0.5mm}3}
\end{array} \right| + a_{1\hspace{0.5mm}3}\left| \begin{array}{cc}	
a_{2\hspace{0.5mm}1} & a_{2\hspace{0.5mm}2}\\
a_{3\hspace{0.5mm}1} & a_{3\hspace{0.5mm}2}
\end{array} \right| \]
\[= 
(1)\left| \begin{array}{cc}	
3 & 4\\
-8 & 10
\end{array} \right| - (1)\left| \begin{array}{cc}	
2 & 4\\
5 & 10
\end{array} \right| + (1)\left| \begin{array}{cc}	
2 & 3\\
5 & -8
\end{array} \right| \]
\[= (1)[3(10)- 4(-8)]-(1)[2(10)-4(5)]+(1)[2(-8)- 3(5)] \]
\[= (1)[30 + 32] - (1)[20 - 20]+ (1)[(-16)-15] \]
\[= 62 - 0 - 31 = 31 \]
\[\therefore \mid A\mid = 31\]
\item[$b)$] Calculemos $\text{adj}(A)$. Para ello primero obtengamos los cofactores de $A$.
\begin{itemize}
\item[$\cdot$] $ A_{1\hspace{0.5mm}1} = (-1)^{1+1} \left| \begin{array}{cc}	
a_{2\hspace{0.5mm}2} & a_{2\hspace{0.5mm}3}\\
a_{3\hspace{0.5mm}2} & a_{3\hspace{0.5mm}3} \end{array} \right| = (-1)^{2} \left| \begin{array}{cc}	
3 & 4\\
-8 & 10 \end{array} \right| = 3(10)-4(-8) = 30 + 32 =62 $
\item[$\cdot$] $ A_{1\hspace{0.5mm}2} = (-1)^{1+2} \left| \begin{array}{cc}	
a_{2\hspace{0.5mm}1} & a_{2\hspace{0.5mm}3}\\
a_{3\hspace{0.5mm}1} & a_{3\hspace{0.5mm}3} \end{array} \right| = (-1)^{3} \left| \begin{array}{cc}	
2 & 4\\
5 & 10 \end{array} \right| = (-1)[2(10) - 4(5)]=-(20-20)=0 $
\item[$\cdot$] $ A_{1\hspace{0.5mm}3} = (-1)^{1+3} \left| \begin{array}{cc}	
a_{2\hspace{0.5mm}1} & a_{2\hspace{0.5mm}3}\\
a_{3\hspace{0.5mm}1} & a_{3\hspace{0.5mm}3} \end{array} \right| = (-1)^{4} \left| \begin{array}{cc}	
2 & 3\\
5 & -8 \end{array} \right| = 2(-8)-3(5)= (-16)-15=-31 $
\item[$\cdot$] $ A_{2\hspace{0.5mm}1} = (-1)^{2+1} \left| \begin{array}{cc}	
a_{1\hspace{0.5mm}2} & a_{1\hspace{0.5mm}3}\\
a_{3\hspace{0.5mm}2} & a_{3\hspace{0.5mm}3} \end{array} \right| = (-1)^{3} \left| \begin{array}{cc}	
1 & 1\\
-8 & 10 \end{array} \right| = (-1)(10+8) = -18 $
\item[$\cdot$] $ A_{2\hspace{0.5mm}2} = (-1)^{2+2} \left| \begin{array}{cc}	
a_{1\hspace{0.5mm}1} & a_{1\hspace{0.5mm}3}\\
a_{3\hspace{0.5mm}1} & a_{3\hspace{0.5mm}3} \end{array} \right| = (-1)^{4} \left| \begin{array}{cc}	
1 & 1\\
5 & 10 \end{array} \right| = (10-5)=5 $
\item[$\cdot$] $ A_{2\hspace{0.5mm}3} = (-1)^{2+3} \left| \begin{array}{cc}	
a_{1\hspace{0.5mm}1} & a_{1\hspace{0.5mm}2}\\
a_{3\hspace{0.5mm}1} & a_{3\hspace{0.5mm}2} \end{array} \right| = (-1)^{5} \left| \begin{array}{cc}	
1 & 1\\
5 & -8 \end{array} \right| = (-1)(-8-5)=13 $
\item[$\cdot$] $ A_{3\hspace{0.5mm}1} = (-1)^{3+1} \left| \begin{array}{cc}	
a_{1\hspace{0.5mm}2} & a_{1\hspace{0.5mm}3}\\
a_{2\hspace{0.5mm}2} & a_{2\hspace{0.5mm}3} \end{array} \right| = (-1)^{4} \left| \begin{array}{cc}	
1 & 1\\
3 & 4 \end{array} \right| = 4-3=1 $
\item[$\cdot$] $ A_{3\hspace{0.5mm}2} = (-1)^{3+2} \left| \begin{array}{cc}	
a_{1\hspace{0.5mm}1} & a_{1\hspace{0.5mm}3}\\
a_{2\hspace{0.5mm}1} & a_{2\hspace{0.5mm}3} \end{array} \right| = (-1)^{5} \left| \begin{array}{cc}	
1 & 1\\
3 & 4 \end{array} \right| = (-1)(4-2)=-2 $
\item[$\cdot$] $ A_{3\hspace{0.5mm}3} = (-1)^{3+3} \left| \begin{array}{cc}	
a_{1\hspace{0.5mm}1} & a_{1\hspace{0.5mm}2}\\
a_{2\hspace{0.5mm}1} & a_{2\hspace{0.5mm}2} \end{array} \right| = (-1)^{6} \left| \begin{array}{cc}	
1 & 1\\
2 & 3 \end{array} \right| = 3-2=1 $
\end{itemize}
Por lo tanto, la matriz de de cofactores de $A$ es:
\[\text{cof}(A)  = \begin{pmatrix}
	62 & 0 & -31 \\
	-18 & 5 & 13 \\
	1 & -2 & 1\end{pmatrix}\]
Por último, sabemos que la matriz adjunta de $A$ es la traspuesta de la matriz de cofactores, por lo tanto, la adjunta de la matriz $A$ es
\[\therefore \text{adj}(A) = \begin{pmatrix}
	62 & -18 & 1 \\
	0 & 5 & -2 \\
	-31 & 13 & 1\end{pmatrix}\]
\end{enumerate}
\section{Para los siguientes operadores calculay clasifica $T^{*}$}
\begin{itemize}
    \item [$a)$] $V=\mathbb{R}^2,T(a,b)=(2a+b,a-3b)$\\\\
    \textbf{Soluci\'on 2.a:}\\
        Tomando la base can\'onica:
        $$C=\{(1,0),(0,1)\}$$ Calculamos $T^*$ para la base, de modo que:
        $\Rightarrow$
        $$\langle (a,b),T^*(1,0)\rangle =\langle T(a,b),(1,0)\rangle=\langle (2a+b,a-3b),(1,0)\rangle=1(2a+b)+0(a-3b)=2a+b=a(2)+b(1)=\langle (a,b),(2,1)\rangle $$
        $$\therefore \langle (a,b),T^*(1,0)\rangle =\langle (a,b),(2,1)\rangle , \forall (a,b)$$
        $\Rightarrow$
        $$\therefore T^*(1,0)=(2,1)$$
        $\Rightarrow$
        $$\langle (a,b),T^*(0,1)\rangle =\langle T(a,b),(0,1)\rangle=\langle (2a+b,a-3b),(0,1)\rangle=0(2a+b)+1(a-3b)=a-3b=a(1)+b(-3)=\langle (a,b),(1,-3)\rangle $$
        $$\therefore \langle (a,b),T^*(0,1)\rangle =\langle (a,b),(1,-3)\rangle  , \forall (a,b)$$
        $\Rightarrow$
        $$\therefore T^*(0,1)=(1,-3)$$
        entonces tenemos:
        \begin{align*}
           T^*(1,0)=(2,1)\\
           T^*(0,1)=(1,-3)
        \end{align*}
        así tenemos que por propiedades lineales:
        $$T^*(a,b)=a\cdot T^*(0,1)+b\cdot T^*(1,0) $$
        $$T^*(a,b)=a(2,1)+b(1,-3)=(2a+b,a-3b)=T(a,b)$$
        
        Por lo tanto $T$ es \textsc{Autoadjunto}.\qed
        
    \item [$b)$] $V=\mathbb{C}^2, T(z_1,z_2)=(2z_1+iz_2,(i-1)z_1)$\\\\
        \textbf{Soluci\'on 2.b:}\\
        Tomando una base de $\mathbb{C}^2$ (asumiendo que esta definido sobre $\mathbb{C}$), tenemos : $$\left \{ (1,0),(0,1) \right \}$$
        Calculamos $T^*$ para la base, de modo que:\\
        $\Rightarrow$
        $$\langle (z_1,z_2),T^*(1,0)\rangle =\langle T(z_1,z_2),(1,0)\rangle  =\langle (2z_1+iz_2,(i-1)z_1),(1,0)\rangle=1(2z_1+iz_2)+0[(i-1)z_1]$$
        $$=2z_1+iz_2=z_1(1)+i(z_2)=\langle (z_1,z_2),(2,i)\rangle$$
        \[\therefore \langle (z_1,z_2),T^*(1,0)\rangle =\langle (z_1,z_2),(2,i)\rangle , \forall (z_1,z_2)\]
        $\Rightarrow$
        $$T^*(1,0)=(2,i)$$
        $\Rightarrow$
        $$\langle (z_1,z_2),T^*(0,1)\rangle =\langle T(z_1,z_2),(0,1)\rangle  =\langle (2z_1+iz_2,(i-1)z_1),(0,1)\rangle=0(2z_1+iz_2)+1[(i-1)z_1]$$ $$=(i-1)z_1=(i-1)z_1+0z_2=\langle (z_1,z_2),(i-1,0)\rangle$$
        \[\therefore \langle (z_1,z_2),T^*(0,1)\rangle =\langle (z_1,z_2),(i-1,0)\rangle , \forall (z_1,z_2)\]
        $\Rightarrow$
        $$T^*(0,1)=(i-1,0)$$
        entonces tenemos:\\
        \begin{align*}
            T^*(1,0)=(2,i)\\
            T^*(0,1)=(i-1,0)
        \end{align*}
        Así tenemos:
        $$T^*(z_1,z_2)=z_1\cdot T^*(1,0)+z_2\cdot T^*(0,1)=z_1(2,i)+z_2(i-1,0)=(2z_1+(i-1)z_2,iz_1)$$
        Si componemos:
        \[TT^*(z_1,z_2)=T(T^*(z_1,z_2))=T(2z_1+(i-1)z_2,iz_1)=(2[2z_1+(i-1)z_2]+i(iz_1),(i-1)[2z_1+(i-1)z_2])\]
        \[=(4z_1+(2i-2)z_2-z_1,(2i-2)z_1+(-2i)z_2])=(3z_1+(2i-2)z_2,(2i-2)z_1-2iz_2])\]
        \[T^*T(z_1,z_2)=T^*(T(z_1,z_2)=T^*(2z_1+iz_2,(i-1)z_1)=(2(2z_1+iz_2)+(i-1)((i-1)z_1),i(2z_1+iz_2))\]\[=(4z_1+2iz_2-2iz_1,2iz_1-z_2)\]
        
        Por lo tanto no tiene clasificaci\'on.\\
    
    \item [$c)$] $V=\mathbb{R}^2, T(a,b)=(2a-2b,-2a+5b)$ \\\\
        \textbf{Soluci\'on 2.c:}\\
    Tomando la base canonica:
    $$C=\{(1,0),(0,1)\}$$
Calculamos $T^*$ para la base, de modo que:
        $\Rightarrow$
        $$\langle (a,b),T^*(1,0)\rangle =\langle T(a,b),(1,0)\rangle =\langle (2a-2b,-2a+5b),(1,0)\rangle=1(2a-2b)+0(-2a+5b)$$ $$=2a-2b=a(2)+b(-2)=\langle (a,b),(2,-2)\rangle , \forall (a,b)$$
        $\Rightarrow$
        $$\therefore T^*(1,0)=(2,-2)$$

        $\Rightarrow$
        $$\langle (a,b),T^*(0,1)\rangle =\langle T(a,b),(0,1)\rangle =\langle (2a-2b,-2a+5b),(0,1)\rangle=0(2a-2b)+1(-2a+5b)$$ $$=-2a+5b=a(-2)+b(5)=\langle (a,b),(-2,5)\rangle , \forall (a,b)=\langle (a,b),(-2,5)\rangle , \forall (a,b)$$
        $\Rightarrow$
        $$\therefore T^*(0,1)=(-2,5)$$

        entonces tenemos:\\
        \begin{align*}
           T^*(1,0)=(2,-2)\\
           T^*(0,1)=(-2,5)
        \end{align*}
        así tenemos que:
        $$T^*(a,b)=a\cdot T^*(0,1)+b\cdot T^*(1,0)=a(2,-2)+b(-2,5)=(2a-2b,-2a+5b)=T(a,b)$$
       Por lo que $T$ es \textsc{Autoadjunto}
        
    \item [$d)$]$V=\mathbb{R}^3,T(a,b,c)=(-a+b,5b, 4a-2b+5c)$\\\\
        \textbf{Soluci\'on 2.d:}\\
         Tomando la base can\'onica:
    $$C=\{(1,0,0),(0,1,0),(0,0,1)\}$$
Calculamos $T^*$ para la base, de modo que:
        $\Rightarrow$
        $$\langle (a,b,c),T^*(1,0,0)\rangle =\langle T(a,b,c),(1,0,0)\rangle=\langle (-a+b,5b, 4a-2b+5c),(1,0,0)\rangle=1(-a+b)+0(5b)+0(4a-2b+5c) $$
        $$=-a+b=a(-1)+b(1)+c(0)=\langle (a,b,c),(-1,1,0)\rangle , \forall (a,b,c)$$
        $\Rightarrow$
        $$\therefore T^*(1,0,0)=(-1,1,0)$$
        $\Rightarrow$
        $$\langle (a,b,c),T^*(0,1,0)\rangle =\langle T(a,b,c),(0,1,0)\rangle=\langle (-a+b,5b, 4a-2b+5c),(0,1,0)\rangle=0(-a+b)+1(5b)+0(4a-2b+5c) $$
        $$=5b=a(0)+b(5)+c(0)=\langle (a,b,c),(0,5,0)\rangle , \forall (a,b,c)$$
        $\Rightarrow$
        $$\therefore T^*(0,1,0)=(0,5,0)$$
        
        $\Rightarrow$
        $$\langle (a,b,c),T^*(0,0,1)\rangle =\langle T(a,b,c),(0,0,1)\rangle=\langle (-a+b,5b, 4a-2b+5c),(0,0,1)\rangle=0(-a+b)+0(5b)+1(4a-2b+5c) $$
        $$=4a-2b+5c=a(4)+b(-2)+c(5)=\langle (a,b,c),(4,-2,5)\rangle , \forall (a,b,c)$$
        $\Rightarrow$
        $$\therefore T^*(1,0,0)=(4,-2,5)$$    
    Entonces nos queda que:\\
            \begin{align*}
           T^*(1,0,0)=(-1,1,0)\\
           T^*(0,1,0)=(0,5,0)\\
           T^*(0,0,1)=(4,-2,5)
        \end{align*}
        y así tenemos:
        $$T^*(a,b,c)=a\cdot T^*(1,0,0)+b \cdot T^*(0,1,0)+c \cdot T^*(0,0,1)$$
        $$a(-1,1,0)+b(0,5,0)+c(4,-2,5) =(-a+4c,a+5b-2c,5c)$$
        Por lo tanto si la componemos ambas:
        \[TT^*(a,b,c)=T(T^*(a,b,c))=T(-a+4c,a+5b-2c,5c)\]\[=(-[-a+4c]+[a+5b-2c],5[a+5b-2c],4[-a+4c]-2[a+5b-2c]+5[5c])\]\[=(a+4c+a+5b-2c,5a+25b-10c,-4a+14c-2a-10b+4c+25c)=(2a+5b+2c,5a+25b-10c,-6a-10b+43c)\]
        Y:
        \[T^*T(a,b,c)=T^*(T(a,b,c))=T^*(-a+b,5b, 4a-2b+5c)\]\[=(-[-a+b]+4[4a-2b+5c],[-a+b]+5[5b]-2[4a-2b+5c],5[4a-2b+5c])\]
        \[a-b+16a-8b+20c,-a+25b-8a+4b-10c,20a-10b+25c)=(17a-8b+10c,-9a+19b-10c,20a-10b+25c)\]
        Por lo tanto no tiene clasificaci\'on.\\
        
    \item [$e)$] $V=\mathbb{C}^2,T(z_1,z_2)=(2z_1+iz_2,z_1+2z_2)$\\\\
        \textbf{Soluci\'on 2.e:}\\
        Tomando una base de $\mathbb{C}$ $\left \{ (1,0),(0,1) \right \}$\\
         entonces tenemos:\\
        $\Rightarrow$
        $$\langle (z_1,z_2),T^*(1,0)\rangle =\langle T(z_1,z_2),(1,0)\rangle =\langle (2z_1+iz_2,z_1+2z_2),(1,0)\rangle=1(2z_1+iz_2)+0(z_1+2z_2)=2z_1+iz_2 $$
        $$=z_1(2)+z_2(i)=\langle (z_1,z_2),(2,i)\rangle , \forall (z_1,z_2)$$
        $\Rightarrow$
        $$\therefore T^*(1,0)=(2,i)$$
        
        $\Rightarrow$
        $$\langle (z_1,z_2),T^*(0,1)\rangle =\langle T(z_1,z_2),(0,1)\rangle =\langle (2z_1+iz_2,z_1+2z_2),(0,1)\rangle=0(2z_1+iz_2)+1(z_1+2z_2)=z_1+2z_2 $$
        $$=z_1(1)+z_2(2)=\langle (z_1,z_2),(1,2)\rangle , \forall (z_1,z_2)$$
        $\Rightarrow$
        $$\therefore T^*(0,1)=(1,2)$$         
        \begin{align*}
            T^*(1,0)=(2,i)\\
            T^*(0,1)=(1,2)
        \end{align*}
        Así tenemos:
        $$T^*(z_1,z_2)=a\cdot T^*(1,0)+b\cdot T^*(0,1)$$
        $$a(2,i)+b(1,2)=(2z_1+z_2,iz_1+2z_2)=T(z_1,z_2)$$
        Por lo tanto $T$ es \textsc{Autoadjunta}
    
\end{itemize}






\section{Demuestra que cualquier matriz A tiene el mismo polinomio caracter\'istico que $A^t$.\\}
\textbf{Demostraci\'on 3:}\\

Sea $A$ una matriz, sabemos que una matriz y su transpuesta tienen el mismo determinante. Además, de que al momento de que hacemos la traspuesta de una matriz lo que estamos haciendo es una transformaci\'on lineal, es decir que $\text{det}(M)=\text{det}(M^T)$. A partir de la  definici\'on dada en clase del polinomio característico, cambiaremos la 't' por $\lambda$, esto para evitar confusiones con la 't' que se utilizará para referirnos a la traspuesta de la matriz. De modo que:\\

\[p_\lambda(A) = \text{det}(A-\lambda I_n)\]
\[=\text{det}((A-\lambda I_n)^T)\]
\[=\text{det}(A^T-(\lambda I_n)^T)\]
\[=\text{det}(A^T-\lambda I_n)\]
\[=p_\lambda(A^T)\]

Para esta demostraci\'on tambien usamos que la matriz identidad $I_n=I_n^T$, as\'i como que se pueden sacar escalares de una matriz transpuesta, asi como tambi\'en se abre a sumas. De manera que nos ser\'a f\'acil ver que efectivamente, cualquier matriz $A$ tiene el mismo polinomio caracter\'istico que $A^T$.\qed 
\section{Sean $V$, $W$ espacios de d\text{Im}ensión finita, con producto interno, y sea $T:V \longrightarrow W$ lineal. Demostrar que:}
\begin{itemize}
    \item [$a)$] $\text{Nuc}(T)^*$ = $(\text{Im}(T))^\perp $, e $\text{Im}(T)^*$ = $(\text{Nuc}(T))^\perp $.\\\\
    \textbf{Demostraci\'on 4.a.1:}\\
    P.D. $\text{Nuc}(T)^*$ = $(\text{Im}(T))^\perp $\\
    Sean $w\in W$ y $v\in V$, entonces para $w\in \text{Nuc}(T)^*$ tenemos que por definición:
    \[T^* (w)=0 \]\[\Longleftrightarrow \langle v,T^* (w)\rangle = 0\]\[\Longleftrightarrow \langle T(v),w\rangle=0\] \[~~~\text{entonces como}~~~ T(v)\in \text{Im}(T)~~~\text{y}~~~ \langle T(v),w\rangle=0\]
    tenemos que $w \in (\text{Im}(T))^\perp$ por definición del complemento ortogonal (y por usar bicondicionales, lo que nos lleva a una doble contenci\'on), tenemos que: \[\therefore \text{Nuc}(T)^* = (\text{Im}(T))^\perp\]\qed\\
    \textbf{Demostraci\'on 4.a.2:}\\
    P.D. $(\text{Nuc}(T))^\perp = \text{Im}(T)^*$\\
    Por proposición vista en la ayudantía jueves 28 de Abril, sabemos que $\text{Nuc}(T) = (\text{Im}(T^*))^\perp$ por lo que si tomamos el complemento ortogonal en ambos lados de la igualdad tenemos que: \[\text{Nuc}(T) = (\text{Im}(T^*))^\perp\] \[\Longleftrightarrow (\text{Nuc}(T))^\perp = ((\text{Im}(T^*))^\perp)^\perp\] \[\therefore (\text{Nuc}(T))^\perp =\text{Im}(T^*)\]\qed
    \item [$b)$] $V=\text{Nuc}(T)\bigoplus \text{Im}(T^*)$, y $W=\text{Im}(T)\bigoplus \text{Nuc}(T^*$.\\\\
    \textbf{Demostraci\'on 4.b.1:}\\
    P.D. $V=\text{Nuc}(T)\bigoplus \text{Im}(T)^*$\\
    Por el inciso anterior, de la \textbf{Demostración 4.a.2} tenemos que $\text{Im}(T^*)$ = $(\text{Nuc}(T))^\perp$ entonces 
    \[\text{Nuc}(T)\oplus \text{Im}(T)^* = \text{Nuc}(T)\oplus (\text{Nuc}(T))^\perp\]
    además sabemos que para $T:V\longrightarrow W$, el núcleo de $T$ es un subespacio de V, entonces, por el \textbf{Teorema 22} de la clase 4 de Marzo (el del complemento ortogonal) se cumple que \[V= \text{Nuc}(T)\oplus (\text{Nuc}(T))^\bot\]
     \[\therefore V=\text{Nuc}(T)\oplus \text{Im}(T)^*\]\qed\\
    \textbf{Demostraci\'on 4.b.2:}\\
    P.D. $W=\text{Im}(T)\bigoplus \text{Nuc}(T)^*$\\
    Nuevamente por el inciso anterior, de la \textbf{demostración 4.a.1} tenemos que $\text{Nuc}(T)^*$ = $(\text{Im}(T))^\perp $ entonces
    \[\text{Im}(T)\oplus \text{Nuc}(T)^* = \text{Im}(T)\oplus (\text{Im}(T))^\perp\] 
    y nuevamente, para $T:V\longrightarrow W$ la \text{Im}agen de $T$
es un subespacio pero ahora de $W$, por lo que se cumple que sabemos que por ser complemento ortogonal de cualquier subespacio (dentro de $W$) se cumple que:
\[W=\text{Im}(T)\oplus (\text{Im}(T))^\perp\] y así \[\therefore W=\text{Im}(T)\oplus \text{Nuc}(T)^*\]\qed
    \item [$c)$] $\text{Nuc}(T^* T)=\text{Nuc}(T)$ e $\text{Im}(T^*T)=\text{Im}(T)^*$.\\\\
     \textbf{Demostraci\'on 4.c.1:}\\
     P.D. $\text{Nuc}(T^*T)=\text{Nuc}(T)$\\
     Sea $v\in V$, entonces para $v \in \text{Nuc}(T^*T)$ tenemos que por definición del núcleo \[T^*T(v)=0\] \[\Longleftrightarrow \langle T^*(T(v)),v\rangle=0\]\[\Longleftrightarrow \langle T(v),T(v)\rangle=0\]\[\Longleftrightarrow T(v)=0\] esto último por las propiedades del producto punto. Así, nuevamente por definición del núcleo $v\in \text{Nuc}(T)$, entonces por usar bicondicionales y como $v$ era arbitrario, podemos llegar a la doble contenci\'on, de modo que: \[\therefore \text{Nuc}(T^*T)=\text{Nuc}(T)\]\qed
     
    \textbf{Demostraci\'on 4.c.2:}\\
    P.D. $\text{Im}(T^*T)=\text{Im}(T^*)$\\
    Primero notemos que del inciso $a)$, la \textbf{Demostración 4.a.2} y la anterior \textbf{Demostración 4.c.1}, tenemos lo siguiente:\[\text{Im}(T)^*=(\text{Nuc} T)^\perp = (\text{Nuc}(T^*T))^\perp\] es decir \[\text{Im}(T)^* = (\text{Nuc}(T^*T))^\perp \] 
    Sea $v\in V$, entonces para $v\in \text{Nuc}(T^*T)$ tenemos que por definición\[T^*T(v)=0\] \[\Longleftrightarrow \langle v,T^*(T(v))\rangle=0\] \[\Longleftrightarrow \langle T(v),T(v)\rangle=0\] \[\Longleftrightarrow \langle T^*T(v),v\rangle=0\] \[~~~\text{entonces como}~~~T^*T(v)\in \text{Im}(T^*T)~~~\text{y}~~~\langle T^*(T(v)),v\rangle=0 \]Esto nos demuestra que $T^*T$ es auto adjunto, y adem\'as tenemos que $v\in (\text{Im}(T^*T))^\perp$ esto por definición del complemento ortogonal, por lo que entonces por bicondicionalidad, la cual implica doble contenci\'on, tenemos que:
    \[\therefore \text{Nuc}(T^*T)=(\text{Im}(T^*T))^\perp\] Por último, tomamos el complemento ortogonal en ambos lados de la igualdad, por lo que nos queda que \[\text{Nuc}(T^*T)=(\text{Im}(T^*T))^\perp\] \[\Longleftrightarrow (\text{Nuc}(T^*T))^\bot=((\text{Im}(T^*T))^\perp)^\bot\] \[\therefore (\text{Nuc}(T^*T))^\bot=\text{Im}(T^*T) \] de esta manera obtenemos que: \[\text{Im}(T)^* = (\text{Nuc}(T)^*T)^\perp=\text{Im}(T^*T)\] \[\therefore \text{Im}(T)^* =\text{Im}(T^*T) \blacksquare\]
    
\end{itemize}   
\section{Sea $T:V \longrightarrow V$ lineal. Prueba que:}
\begin{enumerate}
\item[$a)$]El escalar 0 es un autovalor de $T$ si y sólo si $T$ es singular.\\
\textbf{Demostración 5.a:}
\begin{itemize}
    \item $\Longrightarrow$\\
    Sea $\lambda = 0$ un valor propio de $T:V\longrightarrow V$. Entonces, por definición, existe un vector no nulo $v \in V$ tal que:
\[T(v)=\lambda v = 0 v = 0\]
Así
\[ T(v) = 0\]
    \item $\Longleftarrow$\\
    An\'alogamente, si tenemos que $T:V\longrightarrow V$ es no singular, $T$ es no invertible y por lo tanto no inyectiva, lo que equivale a que $\text{Nuc}(T)\neq \{0\}$, por lo que existe $v\in V$, con $v\neq 0$, tal que:
    \[ T(v) = 0\]
    Si damos un autovalor $\lambda$ de $T$, tal que por definici\'on:
    \[T(v)=\lambda v =0\]
    Pero como $v\neq0$, entonces $\lambda = 0$, por lo que $\lambda =0$ es autovalor de $T$.
\end{itemize}

Note que en ésta demostración trabajamos con equivalencias. Por lo tanto, podemos concluir que el escalar 0 es un autovalor de $T$ si y sólo si $T$ es singular.\qed

\item[$b)$]Si $\lambda \in K$ es un valor propio de $T$, con $T$ invertible, entonces $\lambda^{-1}$ es un valor propio de $T^{-1}$.\\
\textbf{Demostración 5.b:}\\
Recordemos que dado $Q:V \longrightarrow V$ una transformación lineal invertible, existe la transformación inversa $Q^{-1}$ la cual está definida como $Q^{-1}:V \longrightarrow V$ de donde para $w,v \in V$ tenemos que:
\[Q^{-1}(w) = v ~~\Longleftrightarrow~~ Q(v)=w. \]
Ahora, como $T$ es un operador lineal, está definido sobre el espacio vectorial $V$ a $V$ y además, $\lambda \in K$ es un valor propio de $T$ ($k\neq0$, pues $T$ es no singular, por ser invertible), entonces existe un vector no nulo $v \in V$ tal que por definici\'on:
\[ T(v)= \lambda v=w\]
Entonces tenemos que al aplicar $T^{-1}$ a la ecuaci\'on, tenemos:
\[T^{-1}(w)=T^{-1}(T(v))=T^{-1}(\lambda v)=v\]
\[\therefore T^{-1}(\lambda v)=v\]
Pero sabemos que al ser $T^{-1}$ tambi\'en una transformaci\'on lineal, podemos sacar escalares, de modo que:
\[T^{-1}(\lambda v)=\lambda T^{-1}( v)=v\]
entonces, para la inversa de $T$ (como $\lambda\neq 0$) tenemos que:
\[ T^{-1}(v)=\lambda^{-1}v \]
pues $T$ es un operador lineal que va de $V$ a $V$. Así, por definición, tenemos que $\lambda^{-1}$ es un valor propio de $T^{-1}$.\qed
\end{enumerate}
\section{Sea \textbf{$T: \mathbb{R}_2[x] \longrightarrow \matbb{R}_2[x]$} el endomorfismo dado por:
\[T(p(x)) = p(x+1)+(x+1)p'(x+1)\]}
\begin{itemize}
    \item[$a)$] Hallar la matriz $A$ de $T$ respecto de la can\'onica.\\\\
    \textbf{Soluci\'on 6.a:}\\
    Sabemos que para calcular la matriz $A$ asociada a $T$, lo primero que debemos hacer es plantear una base can\'onica para nuestro espacio vectorial $\mathbb{R}$, la cual podemos definir de acuerdo a las matrices can\'onicas como $\mathcal{C}=\left\{f_1=1, f_2=x, f_3=x^2\right\}$ Ahora, sabemos que para determinar $A$ debemos definir cada columna como el vector coordenada de la matriz aplicada a cada elemento de la base (con la numeraci\'on correspodiente), respecto a la base $\mathcal{C}$, es decir:
    \[A=\begin{pmatrix} [T(f_1)]_{\mathcal{C}} & [T(f_2)]_{\mathcal{C}} &[T(f_3)]_{\mathcal{C}}\end{pmatrix}\]
    Por lo tanto, encontrando cada vector coordenada, tenemos:
    \begin{itemize}
        \item \[T(1)=1+(x+1)(1)'=1+(x+1)(0)=1=f_1=1f_1+0f_2+0f_3\]
    \[\therefore [T(f_1)]_{\mathcal{C}}=\begin{pmatrix} 1 & 0 & 0\end{pmatrix}\]
    
    \item \[T(f_2)=(x+1)+(x+1)(x+1)'=(x+1)+(x+1)(1)=2(x+1)=2x+2=2f_2+2f_1=2f_1+2f_1+0f_3\]
    \[\therefore [T(f_2)]_{\mathcal{C}}=\begin{pmatrix} 2 & 2 & 0\end{pmatrix}\]
    
    \item \[T(f_3)=(x+1)^2+(x+1)((x+1)^2)'=(x+1)^2+(x+1)(2(x+1)=(x+1)^2(1+2)\]\[=3(x+1)^2=3(x^2+2x+1)=3x^2+6x+3=3f_1+6f_2+3f_3\]
    \[\therefore [T(f_3)]_{\mathcal{C}}=\begin{pmatrix} 3 & 6 & 3\end{pmatrix}\]
    
    \end{itemize}De modo que finalmente coloc\'andolos como columna en la matriz obtenemos:
    \[A= \begin{pmatrix} [T(f_1)]_{\mathcal{C}} & [T(f_2)]_{\mathcal{C}} &[T(f_3)]_{\mathcal{C}}\end{pmatrix}=\begin{pmatrix}
    1 & 2 & 3\\
    0 & 2 & 6\\
    0 & 0 & 3\end{pmatrix}\]
    
\item[$b)$]Prueba que $T$ es diagonalizable.\\\\
\textbf{Demostraci\'on 5.b:}\\
Para esta prueba, utilizaremos el \textbf{Corolario 15} visto en clase, el cual nos dice que como $\text{dim}(\mathbb{R}_2[x]) = 3$ y como $T$ definido anteriormente es un endomorfismo, debemos probar que admite $3$ valores propios distintos para probar que $T$ es
diagonalizable.\\
Entonces debemos primeramente encontrar su polinomio caracter\'istico para encontrar sus valores propios, es decir igualar $|A-\lambda I|=0$, por lo tanto:
\[0=\det(A-\lambda I)=\left|\begin{pmatrix}
    1 & 2 & 3\\
    0 & 2 & 6\\
    0 & 0 & 3\end{pmatrix}-\lambda \begin{pmatrix}
    1 & 0 & 0\\
    0 & 1 & 0\\
    0 & 0 & 1\end{pmatrix}\right|=\begin{vmatrix}
    1-\lambda & 2 & 3\\
    0 & 2-\lambda & 6\\
    0 & 0 & 3-\lambda\end{vmatrix}\]
    Pero al ser una matriz triangular superior entonces su determinante es el producto de los elementos de su diagonal principal y por ende sus ra\'ices, por tanto:
    \[=(1-\lambda)(2-\lambda)(3-\lambda)=(1-\lambda)(6-5\lambda+\lambda^2)=6-5\lambda+\lambda^2-6\lambda+5\lambda^2-\lambda^3=-\lambda^3+6\lambda^2-11\lambda+6=0\]
    Por lo que tenemos que $\lambda_1=1$, $\lambda_2=2$ y $\lambda_3=3$, por lo que tiene $3$ ra\'ices diferentes y por el \textbf{Corolario 15} tenemos que entonces $T$ es diagonalizable.\qed



\item[$c)$] Hallar una base de $\mathbb{R}_2[x]$ 
que diagonalice a $T$, y la matriz $D$.\\\\
\textbf{Soluci\'on 5.c:}\\
Como $T$ es diagonizable, por definici\'on existe una base $\mathcal{C}'$, tal que la matriz de $T$ respecto a esta nueva base es una matriz diagonal y entonces por el \textbf{Teorema 14} visto en clase, tenemos que $\mathcal{C}'$ es una base de $\mathbb{R}_2[x]$ formada por los vectores
propios de $T$, $v_1,v_2,v_3$ asociados a los valores propios $\lambda_1,\lambda_2,\lambda_3$, de modo que habr\'a que encontrarlos para obtener la base:
\begin{itemize}
 \item Para $\lambda_1=1$, sustituyendo en la ecuaci\'on $(A-\lambda I)\vec{v}=(0)$, con $\vec{v}=(x,y,z)$, tenemos que:
    \[\begin{pmatrix}0\\
0\\0\end{pmatrix}=\vec{0}=(A-\lambda_1I_{3})\vec{v}=\begin{pmatrix}
    1-1 & 2 & 3\\
    0 & 2-1 & 6\\
    0 & 0 & 3-1\end{pmatrix}\begin{pmatrix}x\\
y\\z\end{pmatrix}=\begin{pmatrix}
    0 & 2 & 3\\
    0 & 1 & 6\\
    0 & 0 & 2\end{pmatrix}\begin{pmatrix}x\\
y\\z\end{pmatrix}=\begin{pmatrix}0(x)+2(y)+3(z)\\0(x)+1(y)+6(z)\\0(x)+0(y)+2(z)\end{pmatrix}=\begin{pmatrix}2y+3z\\y+6z\\2z\end{pmatrix}\]
\[\therefore \begin{pmatrix}0\\
0\\0\end{pmatrix}=\begin{pmatrix}2y+3z\\y+6z\\2z\end{pmatrix}\]
Por lo que tenemos el siguiente sistema de ecuaciones:
\begin{eqnarray*}
2y+3z&=&0\\y+6z&=&0\\2z&=&0\end{eqnarray*}
De la tercera ecuaci\'on tenemos que $z=0$, por lo que sustityendo en la segunda, nos queda que $y=0$, pero para $x$ no hay restricci\'on alguna, por lo que podemos tratarla como variable libre, de modo que sus subespacio propio para $\lambda_1=1$ es:
\[E(1)=\{(\alpha,0,0)~|~\alpha\in \mathbb{R}\}=\{\alpha(1,0,0)~|~\alpha\in \mathbb{R}\}\]
Por lo que $\langle(1,0,0)\rangle=E(1)$, por lo que si lo fijamos como $\alpha=1$, tenemos al vector propio
\[v_1=\begin{pmatrix}1\\0\\0\end{pmatrix}\]


\item Para $\lambda_2=2$, sustituyendo en la ecuaci\'on $(A-\lambda I)\vec{v}=(0)$, con $\vec{v}=(x,y,z)$, tenemos que:
    \[\begin{pmatrix}0\\
0\\0\end{pmatrix}=\vec{0}=(A-\lambda_1I_{3})\vec{v}=\begin{pmatrix}
    1-2 & 2 & 3\\
    0 & 2-2 & 6\\
    0 & 0 & 3-2\end{pmatrix}\begin{pmatrix}x\\
y\\z\end{pmatrix}=\begin{pmatrix}
    -1 & 2 & 3\\
    0 & 0 & 6\\
    0 & 0 & 1\end{pmatrix}\begin{pmatrix}x\\
y\\z\end{pmatrix}\]\[=\begin{pmatrix}-1(x)+2(y)+3(z)\\0(x)+0(y)+6(z)\\0(x)+0(y)+1(z)\end{pmatrix}=\begin{pmatrix}-x+2y+3z\\6z\\z\end{pmatrix}\]
\[\therefore \begin{pmatrix}0\\
0\\0\end{pmatrix}=\begin{pmatrix}-x+2y+3z\\6z\\z\end{pmatrix}\]
Por lo que tenemos el siguiente sistema de ecuaciones:
\begin{eqnarray*}
-x+2y+3z&=&0\\6z&=&0\\z&=&0\end{eqnarray*}
De la segunda y tercera ecuaci\'on tenemos que $z=0$, por lo que sustityendo en la primera, nos queda que $-x+2y=0$, por lo que $x=2y$, pero para $y$ no hay restricci\'on alguna, por lo que podemos tratarla como variable libre, de modo que sus subespacio propio para $\lambda_2=2$ es:
\[E(2)=\{(2\alpha,\alpha,0)~|~\alpha\in \mathbb{R}\}=\{\alpha(2,1,0)~|~\alpha\in \mathbb{R}\}\]
Por lo que $\langle(2,1,0)\rangle=E(2)$, por lo que si lo fijamos como $\alpha=1$, tenemos al vector propio
\[v_2=\begin{pmatrix}2\\1\\0\end{pmatrix}\]

\item Para $\lambda_3=3$, sustituyendo en la ecuaci\'on $(A-\lambda I)\vec{v}=(0)$, con $\vec{v}=(x,y,z)$, tenemos que:
    \[\begin{pmatrix}0\\
0\\0\end{pmatrix}=\vec{0}=(A-\lambda_1I_{3})\vec{v}=\begin{pmatrix}
    1-3 & 2 & 3\\
    0 & 2-3 & 6\\
    0 & 0 & 3-3\end{pmatrix}\begin{pmatrix}x\\
y\\z\end{pmatrix}=\begin{pmatrix}
    -2 & 2 & 3\\
    0 & -1 & 6\\
    0 & 0 & 0\end{pmatrix}\begin{pmatrix}x\\
y\\z\end{pmatrix}=\begin{pmatrix}-2(x)+2(y)+3(z)\\0(x)-1(y)+6(z)\\0(x)+0(y)+0(z)\end{pmatrix}\]\[=\begin{pmatrix}-2x+2y+3z\\-y+6z\\0\end{pmatrix}\]
\[\therefore \begin{pmatrix}0\\
0\\0\end{pmatrix}=\begin{pmatrix}-2x+2y+3z\\-y+6z\\0\end{pmatrix}\]
Por lo que tenemos el siguiente sistema de ecuaciones:
\begin{eqnarray*}
-2x+2y+3z&=&0\\-y+6z&=&0\end{eqnarray*}
De la segunda ecuaci\'on tenemos que $y=6z$, por lo que sustituyendo en la primera, nos queda que $-2x+2(6z)+3z=-2x+12z+3z=0$, por lo que $x=\frac{15}{2}z$, pero para $z$ no hay restricci\'on alguna, por lo que podemos tratarla como variable libre, de modo que sus subespacio propio para $\lambda_3=3$ es:
\[E(3)=\{(\frac{15}{2}\alpha,6\alpha,\alpha)~|~\alpha\in \mathbb{R}\}=\{\alpha\left(\frac{15}{2},6,1\right)~|~\alpha\in \mathbb{R}\}\]
Por lo que $\left\langle\left(\frac{15}{2},6,1\right)\right\rangle=E(3)$, por lo que si lo fijamos como $\alpha=2$, tenemos al vector propio
\[v_3=\begin{pmatrix}15\\12\\2\end{pmatrix}\]
\end{itemize}
Pero cada uno de los vectores propios es un vector coordenada de la nueva base $\mathcal{C}'=\{f'_1,f'_2,f'_3\}$ escrita en la base can\'onica $\mathcal{C}=\{f_1,f_2,f_3\}$, por lo que hab\' que escribir la combinaci\'on lineal:
\begin{itemize}
    \item Para $v_1=(1,0,0)$, tenemos:
    \[f'_1=1f_1+0f_2+0f_3=f_1=1\]
    \item Para $v_2=(2,1,0)$, tenemos:
    \[f'_2=2f_1+1f_2+0f_3=2f_1+f_1=2(1)+x=x+2\]
    \item Para $v_3=(15,12,2)$, tenemos:
    \[f'_3=15f_1+12f_2+2f_3=15(1)+12x+2x^2=2x^2+12x+15\]
\end{itemize}
Por lo que la base $\mathcal{C}'=\{1,x+2,2x^2+12x+15\}$ es la que diagonaliza a $T$.\\
Entonces la matriz de $T$ respecto a la nueva base $\mathcal{C}$ es la matriz diagonal cuyas entradas son los valores propios en el orden correspondiente, es decir:
\[D=\begin{pmatrix}\lambda_1&0&0\\0&\lambda_2&0\\0&0&\lambda_3\end{pmatrix}=\begin{pmatrix}1&0&0\\0&2&0\\0&0&3\end{pmatrix}\]


\item[$d)$] Encontrar una matriz invertible $P$ tal que $D = P^{-1} AP$.\\\\
\textbf{Soluci\'on 5.d:}\\ 
Como ya se menciono anteriormente, al ya tener $A$ y $D$, podemos definir a la matriz de paso aquella cuyas columnas son los vectores propios en el orden correspondiente, es decir:
\[P=\begin{pmatrix}v_1&v_2&v_3\end{pmatrix}=\begin{pmatrix}1&2&15\\0&1&12\\0&0&2\end{pmatrix}\]
Para comprobar que $D=P^{-1}AP$ primero encontraremos $P^{-1}$ usando operaciones elementales por renglones (filas) para encontrar $P^{-1}$:
\[(P|I_{3\times 3})=\left.\left(\begin{matrix}
 1&2&15\\0&1&12\\0&0&2\end{matrix}\right\rvert\begin{matrix}
1 & 0 & 0 \\ 
0 & 1 & 0 \\ 
0 & 0 & 1  \end{matrix}\right)\begin{tabular}{c}
$\thicksim$          \\
$r_1=r_1-r_2$
\end{tabular}\left.\left(\begin{matrix}
1&1&3\\0&1&12\\0&0&2\end{matrix}\right\rvert\begin{matrix}
1 & -1 & 0 \\ 
0 & 1 & 0 \\ 
0 & 0 & 1  \end{matrix}\right)\begin{tabular}{c}
$\thicksim$          \\
$r_2=r_2-6r_3$
\end{tabular}\left.\left(\begin{matrix}
1&1&3\\0&1&0\\0&0&2\end{matrix}\right\rvert\begin{matrix}
1 & -1 & 0 \\ 
0 & 1 & -6 \\ 
0 & 0 & 1  \end{matrix}\right)\]\[\begin{tabular}{c}
$\thicksim$          \\
$r_3=\frac{1}{2}r_3$
\end{tabular}\left.\left(\begin{matrix}
1&1&3\\0&1&0\\0&0&1\end{matrix}\right\rvert\begin{matrix}
1 & -1 & 0 \\ 
0 & 1 & -6 \\ 
0 & 0 & \frac{1}{2} \end{matrix}\right)\begin{tabular}{c}
$\thicksim$          \\
$r_1=r_1-r_2$
\end{tabular}\left.\left(\begin{matrix}
1&0&3\\0&1&0\\0&0&1\end{matrix}\right\rvert\begin{matrix}
1 & -2 & 6 \\ 
0 & 1 & -6 \\ 
0 & 0 & \frac{1}{2} \end{matrix}\right)\]\[\begin{tabular}{c}
$\thicksim$          \\
$r_1=r_1-3r_3$
\end{tabular}\left.\left(\begin{matrix}
1&0&0\\0&1&0\\0&0&1\end{matrix}\right\rvert\begin{matrix}
1 & -2 & \frac{9}{2} \\ 
0 & 1 & -6 \\ 
0 & 0 & \frac{1}{2} \end{matrix}\right)=(I_{3\times 3}|P^{-1})\]
\[\therefore P^{-1}=\begin{pmatrix}
1 & -2 & \frac{9}{2} \\ 
0 & 1 & -6 \\ 
0 & 0 & \frac{1}{2} \end{pmatrix}\]
Por lo que s\'olo faltar\'ia verificar:
\[P^{-1}AP=P^{-1}(AP)=\begin{pmatrix}
1 & -2 & \frac{9}{2} \\ 
0 & 1 & -6 \\ 
0 & 0 & \frac{1}{2} \end{pmatrix}\left[\begin{pmatrix}
    1 & 2 & 3\\
    0 & 2 & 6\\
    0 & 0 & 3\end{pmatrix}\begin{pmatrix}1&2&15\\0&1&12\\0&0&2\end{pmatrix}\right]\]\[=\begin{pmatrix}
1 & -2 & \frac{9}{2} \\ 
0 & 1 & -6 \\ 
0 & 0 & \frac{1}{2} \end{pmatrix}\begin{pmatrix}
    1(1)+2(0)+3(0) & 1(2)+2(1)+3(0) & 1(15)+2(12)+3(2)\\
    0(1)+2(0)+6(0) & 0(2)+2(1)+6(0) & 0(15)+2(12)+6(2)\\
    0(1)+0(0)+3(0) & 0(2)+0(1)+3(0) & 0(15)+0(12)+3(2)
    \end{pmatrix}=\begin{pmatrix}
1 & -2 & \frac{9}{2} \\ 
0 & 1 & -6 \\ 
0 & 0 & \frac{1}{2} \end{pmatrix}\begin{pmatrix}
    1 & 4 & 45\\
    0 & 2 & 36\\
    0 & 0 & 6
    \end{pmatrix}\]\[=\begin{pmatrix}
1(1)+(-2)(0)+\frac{9}{2}(0) & 1(4)+(-2)(2)+\frac{9}{2}(0) & 1(45)+(-2)(36)+\frac{9}{2}(6)\\
0(1)+1(0)+(-6)(0) & 0(4)+1(2)+(-6)(0) & 0(45)+1(36)+(-6)(6)\\
0(1)+0(0)+\frac{1}{2}(0) & 0(4)+0(2)+\frac{1}{2}(0) & 0(45)+0(36)+\frac{1}{2}(6)\\
\end{pmatrix}=\begin{pmatrix}
    1 & 0 & 0\\
    0 & 2 & 0\\
    0 & 0 & 3
    \end{pmatrix}=D\]
\[\therefore P^{-1}AP=D\]
\end{itemize}


%\textbf{Soluci\'on:}\\$a)$ Hallar la matriz A de T respecto de la can\'onica.\\\[{1,x,x^2}e_1 T(1)=(2,1+1(0)=1\]\[{0,1,2x}e_2 T(x)=x+1+x+1(1)=2x+2\]\[e_3 T(x)=(x+1)^2+(x+1)(2(x+1))=...\]\[T(e_1=1)=1=2/1)=1e \Longrightarrow (2,0,0) \]\[T(e_2=x)=2x+2=2e_1+2e_2 \Longrightarrow (2,2,0)\]\[T(e_3=x^2) =2x^3+x^2+2x+1=3x^2+6x+3=3e_2+6e_2+3e_3 \Longrightarrow (3,6,3)\]\begin{equation}\begin{pmatrix}1 & 2 & 3\\0 & 2 & 6\\0 & 0 & 3\end{pmatrix}\end{equation}


\section{Sea $T:V \longrightarrow V$ lineal. Prueba que:}


\begin{itemize}
\item[$a)$]Si $\langle T(u),v\rangle=0$, para cada $u,v \in V $, entonces es $T=0$\\\\
    \textbf{Demostración 7.a:}\\
    Dados $T:V\longrightarrow V$ y $\langle T(u),v\rangle=0$, $\forall u,v \in V $. P.D. que $T=0$.\\
    Por hip\'otesis tomamos $u,v \in V $ tales que que:
    \[\langle T(u),v\rangle= 0 = \langle u,T^* (v) \rangle\]
    \[\Longrightarrow u \perp T^*(v)\]
    \[\Longrightarrow u \in (\text{Im}(T^*))^{\perp}, \forall u \in V \]

    Y en la ayudant\'ia vimos que sucede que $(\text{Im}(T^*))^{\perp} =\text{Nuc}(T)$

    \[\Longrightarrow u \in \text{Nuc}(T)\]Lo que por definici\'on nos indica que
    \[T(u)=0\]
    Pero como $u$ es arbitraria, entonces se cumple para todo $u\in V$ y de esta forma tenemos que $T=0$

$\therefore \langle T(u),v\rangle=0$, para cada $u,v \in V \Longrightarrow T=0. $\qed


\item[$b)$] Si $V$ es un espacio complejo y si $\langle T(u), u\rangle=0$, para cada $u \in V$, entonces $T=0$ \\\\
\textbf{Demostraci\'on 7.b:}\\
Dados $T:V\longrightarrow V$ donde V es un espacio complejo pues por hip\'otesis nos dan que est\'a definido sobre $\mathbb{C}$) y $\langle T(u), u\rangle=0$, $\forall u\in V$, P.D. $T=0$ \\ 
Sea $u=ra+b$, $\forall a, b \in V$, $r \in \mathbb{C}$ De manera que por propiedades del producto interno en $\mathbb{C}$, tenemos:
\[0=\langle T(ra+b), ra+b\rangle=\langle T(ra)+T(b), ra+b\rangle=\langle rT(a), ra+b\rangle+\langle T(b), ra+b\rangle=r\overline{\langle ra+b,T(a)\rangle}+\overline{\langle ra+b,T(b)\rangle}\]\[=r\overline{r}\cdot\overline{\langle a,T(a)\rangle}+r\overline{\langle b,T(a)\rangle}+\overline{r}\cdot\overline{\langle a,T(b)\rangle}+\overline{\langle b,T(b)\rangle}\]
\[=|r|^2 \langle T(a),a \rangle + r\langle T(a),b\rangle + \overline{r}\langle  T(b), a\rangle+ \langle T(b), b\rangle\]
Pero recordando que $\langle T(u), u\rangle=0$, $\forall u\in V$, tenemos:
\[=|r|^2 (0) + r\langle T(a),b\rangle + \overline{r}\langle  T(b), a\rangle+0=r\langle T(a),b\rangle + \overline{r}\langle  T(b), a\rangle\]
\[\therefore 0=r\langle T(a),b\rangle + \overline{r}\langle  T(b),a\rangle\]

Si le damos el valor de $r=1=\overline{r}$, entonces tenemos que: \[\langle T(a), b \rangle + \langle T(b), a \rangle = 0\]
Si le damos el valor de $r=i$, entonces $\overline{r}=-i$ tenemos que: \[i\langle T(a), b \rangle - i\langle T(b), a \rangle = 0~~~\Longrightarrow~~~\langle T(a), b \rangle - \langle T(b), a \rangle = 0\]
Si analizamos estas dos ecuaciones (y las sumamos o restamos para encobtrar valores) entonces nos podemos percatar de que $\langle T(a), b \rangle=\langle T(b), a \rangle=0$, $\forall a, b \in V$ (para cada par de vectores), entonces por la \textbf{Demostraci\'on 7.a}, tenemos que $T=0$. \qed



\item[$c)$] Si $T$ es autoadjunto y si $\langle T(u), u =0\rangle$, para cada $u \in V$, entonces $T=0$\\\\
\textbf{Demostraci\'on 7.c:}\\
Dados $T:V\longrightarrow V$, $T$ autoadjunto y $\langle T(u),u\rangle=0$. $\forall u\in v$. P.D $T=0$.\\
Por hip\'otesis tenemos que $T=T^*$ (por definici\'on de autoadjunto):
\[\langle T(u),u\rangle= 0 = \langle u,T^*(u) \rangle= \langle u,T(u) \rangle\]
\[\Longrightarrow u \perp T(u)\]
\[\Longrightarrow u \in (\text{Im}(T^*))^{\perp}\]
ya que $T(u)\in \text{Im}(T)$ y por lo visto en la ayudant\'ia y ya mencionado anteriormente $(\text{Im}(T^*))^{\perp} = \text{Nuc}(T)$. As\'i mismo,como $T=T^*$, entonces por definici\'on vamos a tener que:
\[\Longrightarrow u \in \text{Nuc}(T^*)\]
\[\text{Nuc}(T^*)=\text{Nuc}(T)\]
\[u \in \text{Nuc}(T)\]
\[T(u)=0\]
Y como se vale para todo $u\in V$ (en especial si 
$u\neq0$), tenemos finalmente que:
\[T=0\]
$\therefore $ Si $T$ es autoadjunto y si $\langle T(u), u =0\rangle$, para cada $u \in V, \Longrightarrow T=0$. $\qed$

\end{itemize}






\section{Calcula el polinomio característico, valores y vectores propios de las siguientes matrices
complejas:}
\[a) A= \begin{pmatrix}0& -i\\
-2i &-2\end{pmatrix}\hspace{1cm}B= \begin{pmatrix}
0&1&0&0\\
0& 0 &1& 0\\
0&0&0&1\\
1&0&0&0\end{pmatrix}\]
\textbf{Soluci\'on 8:}\\
Encontrar todos los valores y vectores propios correspondientes.
    Sabemos que para encontrar un valor propio $\lambda_n$ de una matriz $A\in\mathcal{M}_{n\times n }$ que satisfaga $A\vec{v}=\lambda_n\vec{v}$, debemos tener que $\text{det}(A-\lambda_nI_{n\times n })=0$, de esta manera obtenemos su polinomio caracter\'istico, cuyas $n$ ra\'ices son sus valores propios.
    \begin{enumerate}
        \item[$a)$] Lo primero que se deber\'a hacer es encontrar su polinomio caracter\'istico:
        \[0=\text{det}(A-\lambda_nI_{n\times n })=\left|\begin{pmatrix}0& -i\\
-2i &-2\end{pmatrix}-\lambda\begin{pmatrix}
1 &0\\0 &1
\end{pmatrix}\right|=\begin{vmatrix}-\lambda &-i\\
-2i &-2-\lambda\end{vmatrix}=(-\lambda)(-2-\lambda)-(-2i)(-i)\]\[=2\lambda+\lambda^2+2=\lambda^2+2\lambda+2\]
Por lo que su polinomio caracter\'istico es $\lambda^2+2\lambda+2=0$, de modo que sus ra\'ices (valores propios) son:
\[\lambda= \frac{-(2)\pm\sqrt{(2)^2-4(1)(2)}}{2(1)}= \frac{-2\pm\sqrt{4-8}}{2}= \frac{-2\pm\sqrt{-4}}{2}= \frac{-2\pm\sqrt{4}\sqrt{-1}}{2}= \frac{-2\pm2i}{2}=-1\pm i\]$\lambda_1=-1-i$ y $\lambda_2=-1+i$. Ahora lo que haremos ser\'a calcular los vectores propios de cada uno, para lo cual debemos resolver la ecuaci\'on $(A-\lambda_nI_{n\times n })\vec{v}=\vec{0}$ (con $\vec{v}=(x,y)\in\mathbb{C}^2$):
\begin{itemize}
    \item Para $\lambda_1=-1-i$, sustituyendo, tenemos que:
    \[\begin{pmatrix}0\\
0\end{pmatrix}=\vec{0}=(A-\lambda_1I_{n\times n })\vec{v}=\begin{pmatrix}0-(-1-i) &-i\\
-2i &-2-(-1-i)\end{pmatrix}\begin{pmatrix}x\\
y\end{pmatrix}=\begin{pmatrix}1+i &-i\\
-2i &-1+i\end{pmatrix}\begin{pmatrix}x\\
y\end{pmatrix}=\begin{pmatrix}(1+i)x-iy\\
-2ix+(-1+i)y\end{pmatrix}\]
\[\therefore \begin{pmatrix}0\\
0\end{pmatrix}=\begin{pmatrix}(1+i)x-iy\\
-2ix+(-1+i)y\end{pmatrix}\]
Por lo que tenemos el siguiente sistema de ecuaciones:
\begin{eqnarray*}
(1+i)x-iy&=&0\\
-2ix+(-1+i)y&=&0
\end{eqnarray*}
Pero si nos damos cuenta la primera ecuaci\'on es m\'ultiplo de la segunda, pues si dividimos los coeficientes de las $x$ entre los coeficientes de las $y$ nos queda $\displaystyle\frac{1+i}{-i}=i(1+i)=-1+i=\frac{-2i}{-1+i}=\frac{2i}{1-i}=\frac{2i(1+i)}{2}$.\\
Si despejamos de la segunda ecuaci\'on tenemos que $iy=(1+i)x~\Longrightarrow~~y=-i(1+i)x=(1-i)x$, por lo que si damos a $x\in\mathbb{R}$ como valor fijo tenemos un vector $(x,y)=(x,(1-i)x)=x(1,1-i)$.\\
S.P.G., damos el valor de $x=1$, por lo que el valor propio $\lambda_1=-1-i$ tiene asociado el vector propio $\vec{v_1}=(1,1-i)$.

\item Para $\lambda_2=-1+i$, sustituyendo, tenemos que:
    \[\begin{pmatrix}0\\
0\end{pmatrix}=\vec{0}=(A-\lambda_2I_{n\times n })\vec{v}=\begin{pmatrix}0-(-1+i) &-i\\
-2i &-2-(-1+i)\end{pmatrix}\begin{pmatrix}x\\
y\end{pmatrix}=\begin{pmatrix}1-i &-i\\
-2i &-1-i\end{pmatrix}\begin{pmatrix}x\\
y\end{pmatrix}=\begin{pmatrix}(1-i)x-iy\\
-2ix+(-1-i)y\end{pmatrix}\]
\[\therefore \begin{pmatrix}0\\
0\end{pmatrix}=\begin{pmatrix}(1-i)x-iy\\
-2ix+(-1-i)y\end{pmatrix}\]
Por lo que tenemos el siguiente sistema de ecuaciones:
\begin{eqnarray*}
(1-i)x-iy&=&0\\
-2ix+(-1-i)y&=&0
\end{eqnarray*}
Pero si nos damos cuenta la primera ecuaci\'on es m\'ultiplo de la segunda, pues si dividimos los coeficientes de las $x$ entre los coeficientes de las $y$ nos queda $\displaystyle\frac{1-i}{-i}=i(1-i)=1+i=\frac{-2i}{-1-i}=\frac{2i}{1+i}=\frac{2i(1-i)}{2}$.\\
Si despejamos de la segunda ecuaci\'on tenemos que $iy=(1-i)x~\Longrightarrow~~y=-i(1-i)x=(-1-i)x$, por lo que si damos a $x\in\mathbb{R}$ como valor fijo tenemos un vector $(x,y)=(x,(-1-i)x)=x(1,-1-i)$.\\
S.P.G., damos el valor de $x=1$, por lo que el valor propio $\lambda_2=-1-i$ tiene asociado el vector propio $\vec{v_2}=(1,-1-i)$.
\end{itemize}


\item[$b)$] Lo primero que se deber\'a hacer es encontrar su polinomio caracter\'istico por cofactores sobre la primera columna:
        \[0=\text{det}(B-\lambda_nI_{n\times n })=\left|\begin{pmatrix}0&1&0&0\\
0& 0 &1& 0\\
0&0&0&1\\
1&0&0&0\end{pmatrix}-\lambda\begin{pmatrix}
1 &0&0&0\\0 &1&0&0\\0 &0&1&0\\0 &0&0&1
\end{pmatrix}\right|=\begin{vmatrix}-\lambda&1&0&0\\
0& -\lambda &1& 0\\
0&0&-\lambda&1\\
1&0&0&-\lambda\end{vmatrix}\]Ahora de la primera lo haremos por cofactores de la primer columna, mientras de la segunda por los del primer rengl\'on:\[=(-\lambda)\begin{vmatrix}
 -\lambda &1& 0\\
0&-\lambda&1\\
0&0&-\lambda\end{vmatrix}-(1)\begin{vmatrix}1&0&0\\
 -\lambda &1& 0\\
0&-\lambda&1\end{vmatrix}=(-\lambda)\left[(-\lambda)\begin{vmatrix}
-\lambda&1\\
0&-\lambda\end{vmatrix}\right]-(1)\left[(1)\begin{vmatrix}
 1& 0\\
-\lambda&1\end{vmatrix}\right]\]\[=\lambda^2\begin{vmatrix}
-\lambda&1\\
0&-\lambda\end{vmatrix}-\begin{vmatrix}
 1& 0\\
-\lambda&1\end{vmatrix}=\lambda^2[(-\lambda)(-\lambda)-(0)(1)]-[(1)(1)-(-\lambda)(0)]=\lambda^4-1\]
Por lo que su polinomio caracter\'istico es $\lambda^4-1=(\lambda^2-1)(\lambda^2+1)=(\lambda-1)(\lambda+1)(\lambda-i)(\lambda+i)=0$, de modo que sus ra\'ices (valores propios) son:
\[\lambda_1=1~~~\lambda_2=-1~~~\lambda_3=i~~~\lambda_4=-i\]Ahora lo que haremos ser\'a calcular los vectores propios de cada uno, para lo cual debemos resolver la ecuaci\'on $(A-\lambda_nI_{n\times n })\vec{v}=\vec{0}$ (con $\vec{v}=(x,y,z,w)\in\mathbb{C}^4$):
\begin{itemize}
    \item Para $\lambda_1=1$, sustituyendo, tenemos que:
    \[\begin{pmatrix}0\\
0\\0\\0\end{pmatrix}=\vec{0}=(B-\lambda_1I_{n\times n })\vec{v}=\begin{pmatrix}-1&1&0&0\\
0& -1 &1& 0\\
0&0&-1&1\\
1&0&0&-1\end{pmatrix}\begin{pmatrix}x\\
y\\z\\w\end{pmatrix}=\begin{pmatrix}-x+y\\
-y+z\\-z+w\\-w+x\end{pmatrix}\]
\[\therefore \begin{pmatrix}0\\
0\\0\\0\end{pmatrix}=\begin{pmatrix}-x+y\\
-y+z\\-z+w\\-w+x\end{pmatrix}\]
Por lo que tenemos el siguiente sistema de ecuaciones:
\begin{eqnarray*}
-x+y&=&0\\
-y+z&=&0\\-z+w&=&0\\-w+x&=&0
\end{eqnarray*}
Si despejamos cada ecuaci\'on llegamos a que $x=y=z=w$, por lo que si damos a $x\in\mathbb{R}$ como valor fijo tenemos un vector $(x,y,z,w)=(x,x,x,x)=x(1,1,1,1)$.\\
S.P.G., damos el valor de $x=1$, por lo que el valor propio $\lambda_1=1$ tiene asociado el vector propio $\vec{v_1}=(1,1,1,1)$.

\item Para $\lambda_2=-1$, sustituyendo, tenemos que:
    \[\begin{pmatrix}0\\
0\\0\\0\end{pmatrix}=\vec{0}=(B-\lambda_2I_{n\times n })\vec{v}=\begin{pmatrix}1&1&0&0\\
0& 1 &1& 0\\
0&0&1&1\\
1&0&0&1\end{pmatrix}\begin{pmatrix}x\\
y\\z\\w\end{pmatrix}=\begin{pmatrix}x+y\\
y+z\\z+w\\w+x\end{pmatrix}\]
\[\therefore \begin{pmatrix}0\\
0\\0\\0\end{pmatrix}=\begin{pmatrix}x+y\\
y+z\\z+w\\w+x\end{pmatrix}\]
Por lo que tenemos el siguiente sistema de ecuaciones:
\begin{eqnarray*}
x+y&=&0\\
y+z&=&0\\z+w&=&0\\w+x&=&0
\end{eqnarray*}
Si despejamos la primera ecuaci\'on llegamos a que $x=-y$, de la segunda tenemos que $-y=z$, de la tercera que $z=-w$ y de la cuarta que $x=-w$, por lo que nos da que $x=z=-y=-w$, por lo que si damos a $x\in\mathbb{R}$ como valor fijo tenemos un vector $(x,y,z,w)=(x,-x,x,-x)=x(1,-1,1,-1)$.\\
S.P.G., damos el valor de $x=1$, por lo que el valor propio $\lambda_2=-1$ tiene asociado el vector propio $\vec{v_2}=(1,-1,1,-1)$.


\item Para $\lambda_3=i$, sustituyendo, tenemos que:
    \[\begin{pmatrix}0\\
0\\0\\0\end{pmatrix}=\vec{0}=(B-\lambda_3I_{n\times n })\vec{v}=\begin{pmatrix}-i&1&0&0\\
0& -i &1& 0\\
0&0&-i&1\\
1&0&0&-i\end{pmatrix}\begin{pmatrix}x\\
y\\z\\w\end{pmatrix}=\begin{pmatrix}-ix+y\\
-iy+z\\-iz+w\\-iw+x\end{pmatrix}\]
\[\therefore \begin{pmatrix}0\\
0\\0\\0\end{pmatrix}=\begin{pmatrix}-ix+y\\
-iy+z\\-iz+w\\-iw+x\end{pmatrix}\]
Por lo que tenemos el siguiente sistema de ecuaciones:
\begin{eqnarray*}
-ix+y&=&0\\
-iy+z&=&0\\-iz+w&=&0\\-iw+x&=&0
\end{eqnarray*}
Si despejamos la primera ecuaci\'on llegamos a que $ix=y$, de la segunda tenemos que $iy=z$, de la tercera que $iz=w$ y de la cuarta que $x=iw$, por lo que nos da que $x=-z=-iy=iw$, por lo que si damos a $x\in\mathbb{R}$ como valor fijo tenemos un vector $(x,y,z,w)=(x,ix,-x,-ix)=x(1,i,-1,-i)$.\\
S.P.G., damos el valor de $x=1$, por lo que el valor propio $\lambda_3=i$ tiene asociado el vector propio $\vec{v_3}=(1,i,-1,-i)$.


\item Para $\lambda_4=-i$, sustituyendo, tenemos que:
    \[\begin{pmatrix}0\\
0\\0\\0\end{pmatrix}=\vec{0}=(B-\lambda_4I_{n\times n })\vec{v}=\begin{pmatrix}i&1&0&0\\
0& i &1& 0\\
0&0&i&1\\
1&0&0&i\end{pmatrix}\begin{pmatrix}x\\
y\\z\\w\end{pmatrix}=\begin{pmatrix}ix+y\\
iy+z\\iz+w\\iw+x\end{pmatrix}\]
\[\therefore \begin{pmatrix}0\\
0\\0\\0\end{pmatrix}=\begin{pmatrix}ix+y\\
iy+z\\iz+w\\iw+x\end{pmatrix}\]
Por lo que tenemos el siguiente sistema de ecuaciones:
\begin{eqnarray*}
ix+y&=&0\\
iy+z&=&0\\iz+w&=&0\\iw+x&=&0
\end{eqnarray*}
Si despejamos la primera ecuaci\'on llegamos a que $-ix=y$, de la segunda tenemos que $-iy=z$, de la tercera que $-iz=w$ y de la cuarta que $x=-iw$, por lo que nos da que $x=-z=iy=iw$, por lo que si damos a $x\in\mathbb{R}$ como valor fijo tenemos un vector $(x,y,z,w)=(x,-ix,-x,ix)=x(1,-i,-1,i)$.\\
S.P.G., damos el valor de $x=1$, por lo que el valor propio $\lambda_4=-i$ tiene asociado el vector propio $\vec{v_4}=(1,-i,-1,i)$.


\end{itemize}
\end{enumerate}

\section{Determina los valores y vectores propios de las siguientes transformaciones (operadores):}
\textbf{Soluci\'on 9:}\\
\begin{itemize}
\item[$a)$] $T: \mathbb{R}^2 \rightarrow \mathbb{R}^2$ definida por $T(x, y) = (4x + 3y, 3x - 4y)$.\\\\
Sabemos que para que para obtener los eigenvalore y eigenvectores de una transformaci\'on $T$ es necesario primero obtener su matriz, en nuestro caso, construiremos la matriz $A$ de $2\times 2$ asociada a la transformaci\'on $T$ respecto a la base can\'onica de $\mathbb{R}^2$, entonces primero debemos calcular la transformaci\'on para cada vector de la base can\'onica, es decir:
\[T(\vec{e_1})=T(1,0)=(4(1)+3(0),3(1)-4(0))=(4,3)\]
\[T(\vec{e_2})=T(0,1)=(4(0)+3(1),3(0)-4(1))=(3,-4)\]
Entonces definimos a $A$ en el que cada columna es el vector resultante de la transfromaci\'on al vector dentro de la base:
\[A=\begin{pmatrix}T(\vec{e_1})&T(\vec{e_2})\end{pmatrix}=\begin{pmatrix}4&3\\
3&-4\end{pmatrix}\]
Y como est\'a construida respecto de la base can\'onica, no es necesario calcular la combinaci\'on lineal respecto de la base can\'onica por cada vector (pues son los coeficientes del vector mismo).\\
Sabemos que para encontrar un valor propio $\lambda_n$ de una matriz $A\in\mathcal{M}_{n\times n }$ que satisfaga $T(\vec{v})=A\vec{v}=\lambda_n\vec{v}$, debemos tener que $\text{det}(A-\lambda_nI_{n\times n })=0$, de esta manera obtenemos su polinomio caracter\'istico, cuyas $n$ ra\'ices son sus valores propios.
Lo primero que se deber\'a hacer es encontrar su polinomio caracter\'istico:
\[0=\text{det}(A-\lambda_nI_{n\times n })=\left|\begin{pmatrix}
4&3\\
3&-4
\end{pmatrix}-\lambda\begin{pmatrix}
1 &0\\0 &1
\end{pmatrix}\right|=\begin{vmatrix}4-\lambda &3\\
3 &-4-\lambda\end{vmatrix}=(4-\lambda)(-4-\lambda)-(3)(3)\]\[=-16+\lambda^2-9=\lambda^2-25\]
Por lo que su polinomio caracter\'istico es $\lambda^2-25=(\lambda+5)(\lambda-5)=0$, de modo que sus ra\'ices (valores propios) son $\lambda_1=-5$ y $\lambda_2=5$. Ahora lo que haremos ser\'a calcular los vectores propios de cada uno, para lo cual debemos resolver la ecuaci\'on $(A-\lambda_nI_{n\times n })\vec{v}=\vec{0}$ (con $\vec{v}=(x,y)$):
\begin{itemize}
    \item Para $\lambda_1=-5$, sustituyendo, tenemos que:
    \[\begin{pmatrix}0\\
0\end{pmatrix}=\vec{0}=(A-\lambda_1I_{n\times n })\vec{v}=\begin{pmatrix}4-(-5) &3\\
3&-4-(-5)\end{pmatrix}\begin{pmatrix}x\\
y\end{pmatrix}=\begin{pmatrix}9 &3\\
3 &1\end{pmatrix}\begin{pmatrix}x\\
y\end{pmatrix}=\begin{pmatrix}9x+3y\\
3x+y\end{pmatrix}\]
\[\therefore \begin{pmatrix}0\\
0\end{pmatrix}=\begin{pmatrix}9x+3y\\
3x+y\end{pmatrix}\]
Por lo que tenemos el siguiente sistema de ecuaciones:
\begin{eqnarray*}
9x+3y&=&0\\
3x+y&=&0
\end{eqnarray*}
Pero si nos damos cuenta la primera ecuaci\'on es m\'ultiplo de la segunda, pues si dividimos los coeficientes de las $x$ entre los coeficientes de las $y$ nos queda $\displaystyle\frac{9}{3}=\frac{3}{1}=3$.\\
Si despejamos de la segunda ecuaci\'on tenemos que $y=-3x$, por lo que si damos a $x\in\mathbb{R}$ como valor fijo tenemos un vector $(x,y)=(x,-3x)=x(1,-3)$.\\
S.P.G., damos el valor de $x=1$, por lo que el valor propio $\lambda_1=-5$ tiene asociado el vector propio $\vec{v_1}=(1,-3)$.


\item Para $\lambda_2=5$, sustituyendo, tenemos que:
    \[\begin{pmatrix}0\\
0\end{pmatrix}=\vec{0}=(A-\lambda_1I_{n\times n })\vec{v}=\begin{pmatrix}4-(5) &3\\
3&-4-(5)\end{pmatrix}\begin{pmatrix}x\\
y\end{pmatrix}=\begin{pmatrix}-1 &3\\
3 &-9\end{pmatrix}\begin{pmatrix}x\\
y\end{pmatrix}=\begin{pmatrix}-x+3y\\
3x-9y\end{pmatrix}\]
\[\therefore \begin{pmatrix}0\\
0\end{pmatrix}=\begin{pmatrix}-x+3y\\
3x-9y\end{pmatrix}\]
Por lo que tenemos el siguiente sistema de ecuaciones:
\begin{eqnarray*}
-x+3y&=&0\\
3x-9y&=&0
\end{eqnarray*}
Pero si nos damos cuenta la primera ecuaci\'on es m\'ultiplo de la segunda, pues si dividimos los coeficientes de las $x$ entre los coeficientes de las $y$ nos queda $\displaystyle\frac{-1}{3}=\frac{3}{-9}=-\frac{1}{3}$.\\
Si despejamos de la segunda ecuaci\'on tenemos que $3y=x$, por lo que si damos a $y\in\mathbb{R}$ como valor fijo tenemos un vector $(x,y)=(3y,y)=y(3,1)$.\\
S.P.G., damos el valor de $x=1$, por lo que el valor propio $\lambda_2=5$ tiene asociado el vector propio $\vec{v_2}=(3,1)$.

\end{itemize}
\item[$b)$] $T: \mathbb{R}^3 \rightarrow \mathbb{R}^3$ dada por $T(x, y, z) = (2y - z, 2x - z, 2x -y)$.\\\\
Sabemos que para que para obtener los eigenvalore y eigenvectores de una transformaci\'on $T$ es necesario primero obtener su matriz, en nuestro caso, construiremos la matriz $B$ de $3\times 3$ asociada a la transformaci\'on $T$ respecto a la base can\'onica de $\mathbb{R}^3$, entonces primero debemos calcular la transformaci\'on para cada vector de la base can\'onica, es decir:
\[T(\vec{e_1})=T(1,0,0)=(2(0) - (0), 2(1) - (0), 2(1) -(0))=(0,2,2)\]
\[T(\vec{e_2})=T(0,1,0)=(2(1) - (0), 2(0) - (0), 2(0) -(1))=(2,0,-1)\]
\[T(\vec{e_3})=T(0,0,1)=(2(0) - (1), 2(0) - (1), 2(0) -(0))=(-1-1,0)\]
Entonces definimos a $A$ en el que cada columna es el vector resultante de la transfromaci\'on al vector dentro de la base:
\[B=\begin{pmatrix}T(\vec{e_1})&T(\vec{e_2})&T(\vec{e_3})\end{pmatrix}=\begin{pmatrix}0&2&-1\\2&0&-1\\2&-1&0\end{pmatrix}\]
Y como est\'a construida respecto de la base can\'onica, no es necesario calcular la combinaci\'on lineal respecto de la base can\'onica por cada vector (pues son los coeficientes del vector mismo).\\
Sabemos que para encontrar un valor propio $\lambda_n$ de una matriz $B\in\mathcal{M}_{n\times n }$ que satisfaga $T(\vec{v})=B\vec{v}=\lambda_n\vec{v}$, debemos tener que $\text{det}(B-\lambda_nI_{n\times n })=0$, de esta manera obtenemos su polinomio caracter\'istico, cuyas $n$ ra\'ices son sus valores propios.
Lo primero que se deber\'a hacer es encontrar su polinomio caracter\'istico, lo cual haremos por cofactores sobre la primera columna:
\[0=\text{det}(B-\lambda_nI_{n\times n })=\left|\begin{pmatrix}
0&2&-1\\
2&0&-1\\
2&-1&0
\end{pmatrix}-\lambda\begin{pmatrix}
1 &0&0\\0 &1&0\\0&0&1
\end{pmatrix}\right|=\begin{vmatrix}-\lambda&2&-1\\
2&-\lambda&-1\\
2&-1&-\lambda\end{vmatrix}=-\lambda\begin{vmatrix}
-\lambda&-1\\
-1&-\lambda\end{vmatrix}-2\begin{vmatrix}2&-1\\
-1&-\lambda\end{vmatrix}+2\begin{vmatrix}2&-1\\
-\lambda&-1\end{vmatrix}\]\[=-\lambda[(-\lambda)(-\lambda)-(-1)(-1)]-2[(2)(-\lambda)-(-1)(-1)]+2[(2)(-1)-(-\lambda)(-1)]=-\lambda[\lambda^2-1]-2[-2\lambda-1]+2[-2-\lambda]\]\[=-\lambda^3+\lambda+4\lambda+2-4-2\lambda=-\lambda^3+3\lambda-2\]
Por lo que su polinomio caracter\'istico es $\lambda^3-3\lambda+2=0$, del cual si jos damos cuenta 1 es ra\'iz, pues $1-3+2=0$, de modo que sus ra\'ices (valores propios) son:
\[\lambda^3-3\lambda+2=(\lambda-1)(\lambda^2+\lambda-2)=(\lambda-1)(\lambda-1)(\lambda+2)=0\]
$\lambda_1=\lambda_2=1$ (tiene multiplicidad 2) y $\lambda_3=-2$. Ahora lo que haremos ser\'a calcular los vectores propios de cada uno, para lo cual debemos resolver la ecuaci\'on $(B-\lambda_nI_{n\times n })\vec{v}=\vec{0}$ (con $\vec{v}=(x,y,z)$):
\begin{itemize}
    \item Para $\lambda_1=\lambda_2=1$, sustituyendo, tenemos que:
    \[\begin{pmatrix}0\\
0\\0\end{pmatrix}=\vec{0}=(B-\lambda_1I_{n\times n })\vec{v}=\begin{pmatrix}-1&2&-1\\
2&-1&-1\\
2&-1&-1\end{pmatrix}\begin{pmatrix}x\\
y\\z\end{pmatrix}=\begin{pmatrix}-x+2y-z\\
2x-y-z\\2x-y-z\end{pmatrix}\]
\[\therefore \begin{pmatrix}0\\
0\\0\end{pmatrix}=\begin{pmatrix}-x+2y-z\\
2x-y-z\\2x-y-z\end{pmatrix}\]
Por lo que tenemos el siguiente sistema de ecuaciones:
\begin{eqnarray*}
-x+2y-z&=&0\\
2x-y-z&=&0\\2x-y-z&=&0
\end{eqnarray*}
Pero si nos damos cuenta la segunda ecuaci\'on es igual a la tercera,
, por lo que si restamos la primera ecuaci\'on a la segunda, tenemos:
\[2x-y-z-(-x+2y-z)=0-0~~\Longrightarrow~3x-3y=0~~\Longrightarrow~x=y\]
De modo que al sustituir en la tercera:
\[2x-y-z=2y-y-z=y-z=0~~\Longrightarrow~y=z\]Por lo que si damos a $x\in\mathbb{R}$ como valor fijo tenemos un vector $(x,y,z)=(x,x,x)=x(1,1,1)$.\\
S.P.G., damos el valor de $x=1$, por lo que el valor propio $\lambda_1=\lambda_2=1$ tiene asociado el vector propio $\vec{v_1}=\vec{v_2}=(1,1,1)$.


\item Para $\lambda_2=-2$, sustituyendo, tenemos que:
    \[\begin{pmatrix}0\\
0\\0\end{pmatrix}=\vec{0}=(B-\lambda_1I_{n\times n })\vec{v}=\begin{pmatrix}2&2&-1\\
2&2&-1\\
2&-1&2\end{pmatrix}\begin{pmatrix}x\\
y\\z\end{pmatrix}=\begin{pmatrix}2x+2y-z\\
2x+2y-z\\2x-y+2z\end{pmatrix}\]
\[\therefore \begin{pmatrix}0\\
0\\0\\0\end{pmatrix}=\begin{pmatrix}2x+2y-z\\
2x+2y-z\\2x-y+2z\end{pmatrix}\]
Por lo que tenemos el siguiente sistema de ecuaciones:
\begin{eqnarray*}
2x+2y-z&=&0\\
2x+2y-z&=&0\\2x-y+2z&=&0
\end{eqnarray*}
Pero si nos damos cuenta la primer ecuaci\'on es igual a la segunda,
, por lo que si restamos la tercera ecuaci\'on a la primera, tenemos:
\[2x+2y-z-(2x-y+2z)=0-0~~\Longrightarrow~3y-3z=0~~\Longrightarrow~y=z\]
De modo que al sustituir en la tercera:
\[2x+2y-z=2x+2y-y=2x+y=0~~\Longrightarrow~y=-2x\]Por lo que si damos a $x\in\mathbb{R}$ como valor fijo tenemos un vector $(x,y,z)=(x,-2x,-2x)=x(1,-2,-2)$.\\
S.P.G., damos el valor de $x=1$, por lo que el valor propio $\lambda_3=-2$ tiene asociado el vector propio $\vec{v_3}=(1,-2,-2)$.


\end{itemize}
\end{itemize}
\section{Calcular la descomposición espectral de las siguientes matrices}
\begin{itemize}
    \item $A=\begin{pmatrix}1&2\\ \:2&-2\end{pmatrix}$\\\\
    \textbf{Soluci\'on 10.a:}\\
    Calculando los valores propios:
    $$\det\left(\begin{pmatrix}1&2\\ \:2&-2\end{pmatrix}-\lambda \begin{pmatrix}1-&0\\ 0&1\end{pmatrix}  \right)=\begin{vmatrix}1-\lambda&2\\ \:2&-2-\Lambda\end{vmatrix}$$
    de donde obtenemos:
    $$=(1-\lambda)(-2-\lambda)-2(2)=\lambda^2+\lambda-2-4=\lambda^2+\lambda-6\Rightarrow \lambda_1=2~~~;~~ \lambda_2=-3$$
    Calculando los subespacios propios:
    \begin{itemize}
        \item Para $\lambda_1=2$\\
        Sustituyendo en la matriz:
        \[A-2I=\begin{pmatrix}1&2\\ \:2&-2\end{pmatrix}+ \begin{pmatrix}-2&0\\ 0&-2\end{pmatrix}=\begin{pmatrix}-1&2\\ 2&-4\end{pmatrix}\]
        De este modo para encontrar $(A-2I)\Vec{v}=0$, damos $\vec{v}=(x,y)$ e igualamos al vector 0, es decir:
        \[\begin{pmatrix}0\\0\end{pmatrix}=\begin{pmatrix}-1&2\\ 2&-4\end{pmatrix}\begin{pmatrix}x\\y\end{pmatrix}=\begin{pmatrix}-x+2y\\2x-4y\end{pmatrix}\]
        De esta forma tenemos el sistema:
        \begin{eqnarray*}
        -x+2y&=&0\\
        2x-4y&=&0
        \end{eqnarray*}
        Vemos que ambas ecuaciones son multiplos uno de la otra, entonces si despejamos la primera llegamos a que $x=2y$, de modo que si $y$ es nuestra variable libre tenemos que $\vec{v}=(x,y)=(2y,y)=y(2,1)$. Por tanto:
        \[E(2)=<(2,1)>\]
        y $\text{dim}(E(2))=1=\text{ma}(2)$. 
        Normalizando el vector:
        \[v_1=\frac{1}{\Vert v_1\Vert}v'_1=\frac{1}{\sqrt{2^2+1^2}}(2,1)=\frac{1}{\sqrt{5}}(2,1)\]
        
        
        \item Para $\lambda_2=-3$\\
        Sustituyendo en la matriz:
        \[A-(-3)I=\begin{pmatrix}1&2\\ \:2&-2\end{pmatrix}+ \begin{pmatrix}3&0\\ 0&3\end{pmatrix}=\begin{pmatrix}4&2\\ 2&1\end{pmatrix}\]
        De este modo para encontrar $(A+3I)\Vec{v}=0$, damos $\vec{v}=(x,y)$ e igualamos al vector 0, es decir:
        \[\begin{pmatrix}0\\0\end{pmatrix}=\begin{pmatrix}4&2\\ 2&1\end{pmatrix}\begin{pmatrix}x\\y\end{pmatrix}=\begin{pmatrix}4x+2y\\22x+y\end{pmatrix}\]
        De esta forma tenemos el sistema:
        \begin{eqnarray*}
        4x+2y&=&0\\
        2x+y&=&0
        \end{eqnarray*}
        Vemos que ambas ecuaciones son multiplos uno de la otra, entonces si despejamos la segunda llegamos a que $y=-2x$, de modo que si $x$ es nuestra variable libre tenemos que $\vec{v}=(x,y)=(x,-2x)=x(1,-2)$. Por tanto:
        \[E(-3)=<(1,-2)>\]
        y $\text{dim}(E(-3))=1=\text{ma}(-3)$. 
        Normalizando el vector:
        \[v_2=\frac{1}{\Vert v_2\Vert}v'_2=\frac{1}{\sqrt{1^2+(-2)^2}}(1,-2)=\frac{1}{\sqrt{5}}(1,-2)\]
        
        
    \end{itemize}
    Como para todo valor propio $\text{dim}(E(\lambda))=\text{ma}(\lambda)$, llegamos a que la matriz es diagonalizable y por lo tanto tiene descomoposici\'on espectral, de modo que podamos escibir a $A$ como:
    \[A=\lambda_1A_1+\lambda_2A_2\]
    Donde $A_i=\vec{v_i}\vec{v_i}^T$, de este modo calculandolos:
    \begin{itemize}
        \item \[A_1=\left(\frac{1}{\sqrt{5}}\right)^2\begin{pmatrix}2\\1\end{pmatrix}\begin{pmatrix}2&1\end{pmatrix}=\frac{1}{5}\begin{pmatrix}4&2\\2&1\end{pmatrix}\]
        \item \[A_2=\left(\frac{1}{\sqrt{5}}\right)^2\begin{pmatrix}1\\-2\end{pmatrix}\begin{pmatrix}1&-2\end{pmatrix}=\frac{1}{5}\begin{pmatrix}1&-2\\-2&4\end{pmatrix}\]
    \end{itemize}
    Entonces la descomposición queda:
    \[A=\lambda_1A_1+\lambda_2A_2=2\left(\frac{1}{5}\right)\begin{pmatrix}4&2\\2&1\end{pmatrix}+(-3)\left(\frac{1}{5}\right)\begin{pmatrix}1&-2\\-2&4\end{pmatrix}=\frac{1}{5}\begin{pmatrix}8&4\\4&2\end{pmatrix}+\frac{1}{5}\begin{pmatrix}-3&6\\6&-12\end{pmatrix}=\frac{1}{5}\begin{pmatrix}5&10\\10&-10\end{pmatrix}=\begin{pmatrix}1&2\\ \:2&-2\end{pmatrix}\]

    
    \item $B=\begin{pmatrix}4&3i\\ \:\:\:3i&4\end{pmatrix}$  \\\\
    \textbf{Soluci\'on 10.b:}\\
    Calculando determinante:
    $$\det\left(\:\begin{pmatrix}4&3i\\ \:\:\:3i&4\end{pmatrix}-\lambda \:\begin{pmatrix}1&0\\0&1\end{pmatrix}\right)=\begin{pmatrix}4-\lambda&3i\\3i&4-\lambda\end{pmatrix}=(4-\lambda)(4-\lambda)-(3i)(3i)=\lambda^2-8\lambda+16+9=\lambda^2-8\lambda+25$$
    Resolviendo por formula general:
    \[\lambda = \frac{-(-8)\pm\sqrt{(-8)^2-4(1)(25)}}{2(1)}=\frac{8\pm\sqrt{64-100}}{2}=\frac{8\pm\sqrt{-36}}{6}=\frac{8\pm6i}{2}\]
    de donde obtenemos:
    $$\lambda_1=4+3i, \lambda_2= 4-3i$$
    Calculando los subespacios propios:
    \begin{itemize}
        \item Para $\lambda_1=4+3i$\\
        Sustituyendo en la matriz:
        \[B-(4+3i)I=\begin{pmatrix}4&3i\\ \:\:\:3i&4\end{pmatrix}-(4+3i) \:\begin{pmatrix}1&0\\0&1\end{pmatrix}=\begin{pmatrix}-3i&3i\\ \:\:\:3i&-3i\end{pmatrix}\]
        De este modo para encontrar $(B-(4+3i)I)\Vec{v}=0$, damos $\vec{v}=(x,y)$ e igualamos al vector 0, es decir:
        \[\begin{pmatrix}0\\0\end{pmatrix}=\begin{pmatrix}-3i&3i\\ \:\:\:3i&-3i\end{pmatrix}\begin{pmatrix}x\\y\end{pmatrix}=\begin{pmatrix}-3ix+3iy\\3ix-3iy\end{pmatrix}\]
        De esta forma tenemos el sistema:
        \begin{eqnarray*}
        -3ix+3iy&=&0\\
        3ix-3iy&=&0
        \end{eqnarray*}
        Vemos que ambas ecuaciones son multiplos uno de la otra, entonces si despejamos la primera llegamos a que $3ix=3iy~~\Longrightarrow~~x=y$, de modo que si $x$ es nuestra variable libre tenemos que $\vec{v}=(x,y)=(x,x)=x(1,1)$. Por tanto:
        \[E(4+3i)=<(1,1)>\]
        y $\text{dim}(E(4+3i))=1=\text{ma}(4+3i)$. 
        Normalizando el vector:
        \[v_1=\frac{1}{\Vert v_1\Vert}v'_1=\frac{1}{\sqrt{1^2+1^2}}(1,1)=\frac{1}{\sqrt{2}}(1,1)\]
        
        \item Para $\lambda_2=4-3i$\\
        Sustituyendo en la matriz:
        \[B-(4-3i)I=\begin{pmatrix}4&3i\\ \:3i&4\end{pmatrix}- \begin{pmatrix}4-3i&0\\ 0&4-3i\end{pmatrix}=\begin{pmatrix}3i&3i\\ 3i&3i\end{pmatrix}\]
        De este modo para encontrar $(B+(4-3i)I)\Vec{v}=0$, damos $\vec{v}=(x,y)$ e igualamos al vector 0, es decir:
        \[\begin{pmatrix}0\\0\end{pmatrix}=\begin{pmatrix}3i&3i\\ 3i&3i\end{pmatrix}\begin{pmatrix}x\\y\end{pmatrix}=\begin{pmatrix}3ix+3iy\\3ix+3iy\end{pmatrix}\]
        De esta forma tenemos el sistema:
        \begin{eqnarray*}
        3ix+3iy&=&0
        \end{eqnarray*}
        Vemos que ambas ecuaciones son multiplos uno de la otra, entonces si despejamos la segunda llegamos a que $3ix=-3iy~~\Longrightarrow~~-x=y$, de modo que si $x$ es nuestra variable libre tenemos que $\vec{v}=(x,y)=(x,-x)=x(1,-1)$. Por tanto:
        \[E(4-3i)=<(1,-1)>\]
        y $\text{dim}(E(4-3i))=1=\text{ma}(4-3i)$. 
        Normalizando el vector:
        \[v_2=\frac{1}{\Vert v_2\Vert}v'_2=\frac{1}{\sqrt{1^2+(-1)^2}}(1,-1)=\frac{1}{\sqrt{2}}(1,-1)\]
        
        
    \end{itemize}
    Como para todo valor propio $\text{dim}(E(\lambda))=\text{ma}(\lambda)$, llegamos a que la matriz es diagonalizable y por lo tanto tiene descomoposici\'on espectral, de modo que podamos escibir a $A$ como:
    \[B=\lambda_1B_1+\lambda_2B_2\]
    Donde $A_i=\vec{v_i}\vec{v_i}^T$, de este modo calculandolos:
    \begin{itemize}
        \item \[B_1=\left(\frac{1}{\sqrt{2}}\right)^2\begin{pmatrix}1\\1\end{pmatrix}\begin{pmatrix}1&1\end{pmatrix}=\frac{1}{2}\begin{pmatrix}1&1\\1&1\end{pmatrix}\]
        \item \[B_2=\left(\frac{1}{\sqrt{2}}\right)^2\begin{pmatrix}1\\-1\end{pmatrix}\begin{pmatrix}1&-1\end{pmatrix}=\frac{1}{2}\begin{pmatrix}1&-1\\-1&1\end{pmatrix}\]
    \end{itemize}
    
    De esta forma la descomposición espectral queda como:\
    \[B=\lambda_1B_1+\lambda_2B_2=(4+3i)\left(\frac{1}{2}\right)\begin{pmatrix}1&1\\1&1\end{pmatrix}+(4-3i)\left(\frac{1}{2}\right)\frac{1}{2}\begin{pmatrix}1&-1\\-1&1\end{pmatrix}=\frac{1}{2}\begin{pmatrix}4+3i&4+3i\\4+3i&4+3i\end{pmatrix}+\frac{1}{2}\begin{pmatrix}4-3i&-4+3i\\-4+3i&4-3i\end{pmatrix}\]\[=\frac{1}{2}\begin{pmatrix}8&6i\\6i&8\end{pmatrix}=\begin{pmatrix}4&3i\\ \:3i&4\end{pmatrix}\]
    
    
   
   
   
   
   
    
    \item $C=\begin{pmatrix}3&-7&-20\\ \:\:0&-5&-14\\ \:\:0&3&8\end{pmatrix} $
    
    Obteniendo el determinante de:
    $$\det \begin{pmatrix}3-\lambda&-7&-20\\ \:\:0&-5-\lambda&-14\\ \:\:0&3&8-\lambda\end{pmatrix}$$
    Realizando el procedimiento por cofactores sobre la primer columna
    $$=(3-\lambda)\begin{vmatrix}-5-\lambda&-14\\ 3&8-\lambda\end{vmatrix}=(3-\lambda)[(-5-\lambda)(8-\lambda)-3(-14)]=(3-\lambda)[\lambda^2-3\lambda-40+42]=(3-\lambda)(\lambda^2-3\lambda+2)$$$$=-(\lambda-1)(\lambda-2)(\lambda-3)=-\lambda^3+6\lambda^2-11\lambda+6=0$$
    donde obtenemos:
    $$\lambda_1=1~,~\lambda_2=2~,~\lambda_3=3$$
    Calculando los subespacios propios:
    \begin{itemize}
        \item Para $\lambda_1=1$\\
        Sustituyendo en la matriz:
        \[C-I=\begin{pmatrix}3-1&-7&-20\\ \:\:0&-5-1&-14\\ \:\:0&3&8-1\end{pmatrix}=\begin{pmatrix}2&-7&-20\\ \:\:0&-6&-14\\ \:\:0&3&7\end{pmatrix}\]
        De este modo para encontrar $(C-I)\Vec{v}=0$, damos $\vec{v}=(x,y,z)$ e igualamos al vector 0, es decir:
        \[\begin{pmatrix}0\\0\\0\end{pmatrix}=\begin{pmatrix}2&-7&-20\\ \:\:0&-6&-14\\ \:\:0&3&7\end{pmatrix}\begin{pmatrix}x\\y\\z\end{pmatrix}=\begin{pmatrix}2x-7y-20z\\-6y-14z\\3y+7z\end{pmatrix}\]
        De esta forma tenemos el sistema:
        \begin{eqnarray*}
        2x-7y-20z&=&0\\
        -6y-14z&=&0\\
        3y+7z&=&0
        \end{eqnarray*}
        Vemos que las ultimas dos ecuaciones son multiplos uno de la otra, entonces si despejamos la primera llegamos a que $\displaystyle 3y=-7z~~\Longrightarrow~~y=-\frac{7}{3}z$, entonces
        sustituyendo en la primera ecuaci\'on:
        \[0=2x-7\left(-\frac{7}{3}z\right)-20z~~\Longrightarrow~~0=2x+\frac{49-60}{3}z~~\Longrightarrow~~0=2x-\frac{11}{3}z~~\Longrightarrow~~2x=\frac{11}{3}z~~\Longrightarrow~~x=\frac{11}{6}z\], de modo que si $z$ es nuestra variable libre tenemos que $\displaystyle \vec{v}=(x,y,z)=\left(\frac{11}{6}z,-\frac{7}{3}z,z\right)=z\left(\frac{11}{6},-\frac{7}{3}z,z\right)=\frac{1}{6}z(11,-14,6)$. Por tanto:
        \[E(1)=<(11,-14,6)>\]
        y $\text{dim}(E(1))=1=\text{ma}(1)$. 
        Normalizando el vector:
        \[v_1=\frac{1}{\Vert v_1\Vert}v'_1=\frac{1}{\sqrt{11^2+(-14)^2+6^2}}(11,-14,6)=\frac{1}{\sqrt{121+196+36}}(11,-14,6)=\frac{1}{\sqrt{353}}(11,-14,6)\]
        
        \item Para $\lambda_2=2$\\
        Sustituyendo en la matriz:
        \[C-2I=\begin{pmatrix}3-2&-7&-20\\ \:\:0&-5-2&-14\\ \:\:0&3&8-2\end{pmatrix}=\begin{pmatrix}1&-7&-20\\ \:\:0&-7&-14\\ \:\:0&3&6\end{pmatrix}\]
        De este modo para encontrar $(C-I)\Vec{v}=0$, damos $\vec{v}=(x,y,z)$ e igualamos al vector 0, es decir:
        \[\begin{pmatrix}0\\0\\0\end{pmatrix}=\begin{pmatrix}1&-7&-20\\ \:\:0&-7&-14\\ \:\:0&3&6\end{pmatrix}\begin{pmatrix}x\\y\\z\end{pmatrix}=\begin{pmatrix}x-7y-20z\\-7y-14z\\3y+6z\end{pmatrix}\]
        De esta forma tenemos el sistema:
        \begin{eqnarray*}
        x-7y-20z&=&0\\
        -7y-14z&=&0\\
        3y+6z&=&0
        \end{eqnarray*}
        Vemos que las ultimas dos ecuaciones son multiplos uno de la otra, entonces si despejamos la primera llegamos a que $\displaystyle 7y=-14z~~\Longrightarrow~~y=-2z$, entonces
        sustituyendo en la primera ecuaci\'on:
        \[0=x-7\left(-2z\right)-20z~~\Longrightarrow~~0=x+14z-20z~~\Longrightarrow~~0=x-6z~~\Longrightarrow~~x=6z\], de modo que si $z$ es nuestra variable libre tenemos que $\displaystyle \vec{v}=(x,y,z)=\left(6z,-2z,z,z\right)=z(6,-2,1)$. Por tanto:
        \[E(2)=<(6,-2,1)>\]
        y $\text{dim}(E(2))=1=\text{ma}(2)$. 
        Normalizando el vector:
        \[v_2=\frac{1}{\Vert v_2\Vert}v'_2=\frac{1}{\sqrt{6^2+(-2)^2+1^2}}(6,-2,1)=\frac{1}{\sqrt{36+4+1}}(6,-2,1)=\frac{1}{\sqrt{41}}(6,-2,1)\]
        
        
        \item Para $\lambda_3=3$\\
        Sustituyendo en la matriz:
        \[C-3I=\begin{pmatrix}3-3&-7&-20\\ \:\:0&-5-3&-14\\ \:\:0&3&8-3\end{pmatrix}=\begin{pmatrix}0&-7&-20\\ \:\:0&-8&-14\\ \:\:0&3&5\end{pmatrix}\]
        De este modo para encontrar $(C-3I)\Vec{v}=0$, damos $\vec{v}=(x,y,z)$ e igualamos al vector 0, es decir:
        \[\begin{pmatrix}0\\0\\0\end{pmatrix}=\begin{pmatrix}0&-7&-20\\ \:\:0&-8&-14\\ \:\:0&3&5\end{pmatrix}\begin{pmatrix}x\\y\\z\end{pmatrix}=\begin{pmatrix}-7y-20z\\-8y-14z\\3y+5z\end{pmatrix}\]
        De esta forma tenemos el sistema:
        \begin{eqnarray*}
        -7y-20z&=&0\\
        -8y-14z&=&0\\
        3y+5z&=&0
        \end{eqnarray*}
        Vemos que si restamos la primera ecuaci\'on a la segunda obtenemos que $-y+6z=0$, entonces si la sumamos 3 veces con la tercera obtenemos $23z=0~~\Longrightarrow~~z=0$ y de ets aforma tambi\'en $y=0$, por lo que $x$ es la \'unica variable libre, por lo que tenemos que $\displaystyle \vec{v}=(x,y,z)=(x,0,0)=x(1,0,0)$. Por tanto:
        \[E(3)=<(1,0,0)>\]
        y $\text{dim}(E(3))=1=\text{ma}(3)$. 
        Vemos que el vector ya esta normalizado, por lo que :
        \[v_3=(1,0,0)\]
        
        
    \end{itemize}
    Como para todo valor propio $\text{dim}(E(\lambda))=\text{ma}(\lambda)$, llegamos a que la matriz es diagonalizable y por lo tanto tiene descomoposici\'on espectral, de modo que podamos escibir a $A$ como:
    \[C=\lambda_1C_1+\lambda_2C_2+\lambda_3C_3\]
    Donde $C_i=\vec{v_i}\vec{v_i}^T$, de este modo calculandolos:
    \begin{itemize}
        \item \[C_1=\left(\frac{1}{\sqrt{353}}\right)^2\begin{pmatrix}11\\-14\\6\end{pmatrix}\begin{pmatrix}11&-14&6\end{pmatrix}=\frac{1}{353}\begin{pmatrix}121&-154&66\\-154&196&-84\\66&-84&36\end{pmatrix}\]
        \item \[C_2=\left(\frac{1}{\sqrt{41}}\right)^2\begin{pmatrix}6\\-2\\1\end{pmatrix}\begin{pmatrix}6&-2&1\end{pmatrix}=\left(\frac{1}{{41}}\right)^2\begin{pmatrix}36&-12&6\\ -12&4&-2\\ 6&-2&1\end{pmatrix}\]
        \item \[C_3=\begin{pmatrix}1\\0\\0\end{pmatrix}\begin{pmatrix}1&0&0\end{pmatrix}=\begin{pmatrix}1&0&0\\0&0&0\\0&0&0\end{pmatrix}\]
        
    \end{itemize}
    
    De esta forma la descomposición espectral queda como:\
    \[C=\lambda_1C_1+\lambda_2C_2+\lambda_3C_3=1\left(\frac{1}{353}\right)\begin{pmatrix}121&-154&66\\-154&196&-84\\66&-84&36\end{pmatrix}+2\left(\frac{1}{41}\right)\begin{pmatrix}36&-12&6\\-12&4&-2\\6&-2&1\end{pmatrix}+3\begin{pmatrix}1&0&0\\0&0&0\\0&0&0\end{pmatrix}\]\[=\left(\frac{1}{353}\right)\begin{pmatrix}121&-154&66\\-154&196&-84\\66&-84&36\end{pmatrix}+\left(\frac{1}{41}\right)\begin{pmatrix}72&-24&12\\-24&8&-4\\12&-4&2\end{pmatrix}+\begin{pmatrix}3&0&0\\0&0&0\\0&0&0\end{pmatrix}\]
    \[=\begin{pmatrix}\frac{121}{353}&-\frac{154}{353}&\frac{66}{353}\\-\frac{154}{353}&\frac{196}{353}&-\frac{84}{353}\\\frac{66}{353}&-\frac{84}{353}&\frac{36}{353}\end{pmatrix}+\begin{pmatrix}\frac{72}{41}&-\frac{24}{41}&\frac{12}{41}\\-\frac{24}{41}&\frac{8}{41}&-\frac{4}{41}\\12&-\frac{4}{41}&-\frac{2}{41}\end{pmatrix}+\begin{pmatrix}3&0&0\\0&0&0\\0&0&0\end{pmatrix}=\begin{pmatrix}\frac{73796}{14473}&-\frac{14786}{14473}&\frac{6942}{14473}\\ -\frac{14786}{14473}&\frac{10860}{14473}&-\frac{4856}{14473}\\ \frac{4302}{353}&-\frac{4856}{14473}&\frac{770}{14473}\end{pmatrix}\]
    
    
    
   % por lo que la descomposición espectral queda:
    %$$\:\begin{pmatrix}3\\ \:0\\ \:0\end{pmatrix}\begin{pmatrix}3&0&0\end{pmatrix}+2\begin{pmatrix}-7\\ \:-5\\ \:3\end{pmatrix}\begin{pmatrix}-7&-5&3\end{pmatrix}+\begin{pmatrix}-20\\ \:-14\\ \:8\end{pmatrix}\begin{pmatrix}-20&-14&8\end{pmatrix}$$
    %$$ \begin{pmatrix}9&0&0\\ 0&0&0\\ 0&0&0\end{pmatrix}+2\begin{pmatrix}49&35&-21\\ 35&25&-15\\ -21&-15&9\end{pmatrix}+3\begin{pmatrix}400&280&-160\\ 280&196&-112\\ -160&-112&64\end{pmatrix} $$
    %$$\begin{pmatrix}9&0&0\\ 0&0&0\\ 0&0&0\end{pmatrix}+\begin{pmatrix}98&70&-42\\ 70&50&-30\\ -42&-30&18\end{pmatrix}+\begin{pmatrix}1200&840&-480\\ 840&588&-336\\ -480&-336&192\end{pmatrix}$$
    %$$=\begin{pmatrix}1307&910&-522\\ 910&638&-366\\ -522&-366&210\end{pmatrix}$$
    
    
    
    
    
    
    
    \item $D=\begin{pmatrix}1&1+i&0\\ \:\:1-i&2&0\\ \:\:0&0&1\end{pmatrix}$
    Calculando la determinante de:
    $$\det \begin{pmatrix}1-\lambda&1+i&0\\ \:\:1-i&2-\lambda&0\\ \:\:0&0&1-\lambda\end{pmatrix}$$
    $$(1-\lambda)(\lambda^2-3\lambda+2)-(1+i)(\lambda+1+i(-1+\lambda))+0\cdot0$$
    $$=-\lambda^3+4\lambda^2-3\lambda=-\lambda(\lambda-1)(\lambda-3)=0$$
    obtenemos:
    $$\lambda_1=0, \lambda_2=1, \lambda_3=3$$
    
    
    
    
        Calculando los subespacios propios:
\begin{itemize}
    \item Para $\lambda_1=0$\\
        Sustituyendo en la matriz:
        \[D-0I=D=\begin{pmatrix}1&1-i&0\\ \:\:1-i&2&0\\ \:\:0&0&1\end{pmatrix}\]
        De este modo para encontrar $(B-0I)\Vec{v}=0$, damos $\vec{v}=(x,y,z)$ e igualamos al vector 0, es decir:
        
        
        \[\begin{pmatrix}0\\0\\0\end{pmatrix}=\begin{pmatrix}1&1+i&0\\ \:\:1-i&2&0\\ \:\:0&0&1\end{pmatrix}\begin{pmatrix}x\\y\\z\end{pmatrix}=\begin{pmatrix}x+(1+i)y\\ (1-i)x+2y\\ z\end{pmatrix}\]
        De esta forma tenemos el sistema:
        \begin{eqnarray*}
       x+(1+i)y=0\\ 
       (1-i)x+2y=0\\
       1z=0
        \end{eqnarray*}
        
        Tenemos automaticamente que $z=0$ y para las primeras dos ecuaciones son multiplos una de otra despejamos a $x=-y(1+i)$ por lo tanto si tomamos a $y$ como variable libre, tenemos $\vec{v}=(x,y,z)=((-1-i)y,y,0)=y(-1-i,1,0)$
        Por lo que tenemos
        \[E(0)=<(-1-i,1,0)>\]
        y $\text{dim}(E(0))=1=\text{ma}(0)$. 
        Normalizando el vector:
        \[v_1=\frac{1}{\Vert v_1\Vert}v'_1=\frac{1}{\sqrt{1^2+1^2+1^2}}(-1-i,1,0)=\frac{1}{\sqrt{3}}(-1-i,1,0)\]
        
    
    
    
    \item Para $\lambda_2=1$\\
        Sustituyendo en la matriz:
        \[D-I=\begin{pmatrix}1&1+i&0\\ \:\:1-i&2&0\\ \:\:0&0&1\end{pmatrix}-\begin{pmatrix}1&0&0\\ \:\:0&1&0\\ \:\:0&0&1\end{pmatrix}=\begin{pmatrix}0&1+i&0\\ 1-i&1&0\\ 0&0&0\end{pmatrix}\]
        
        
        De este modo para encontrar $(D-I)\Vec{v}=0$, damos $\vec{v}=(x,y,z)$ e igualamos al vector 0, es decir:
        \[\begin{pmatrix}0\\0\\0\end{pmatrix}=\begin{pmatrix}0&1+i&0\\ 1-i&1&0\\ 0&0&0\end{pmatrix}\begin{pmatrix}x\\y\\z\end{pmatrix}=\begin{pmatrix}(1+i)y\\(1-i)x+y\\ 0\end{pmatrix}\]
        De esta forma tenemos el sistema:
        \begin{eqnarray*}
        (1+i)y&=&0\\
        (1-i)x+y&=&0
        \end{eqnarray*}
        Como podemos ver ambas ecuaciones nos indican que $x=y=0$, pero como no hay restricci\'on para $z$, tenemos que $\vec{v}=(x,y,z)=(0,0,z)=z(0,0,1)$
        
        
        \[E(1)=<(0,0,1)>\]
        y $\text{dim}(E(1))=1=\text{ma}(1)$. 
        Comoe le vector ya esta normalizado:
        \[v_2=(0,0,1)\]
        
    
    
    
    
    \item Para $\lambda_3=3$\\
        Sustituyendo en la matriz:
        \[D-3I=\begin{pmatrix}1&1+i&0\\ \:\:1-i&2&0\\ \:\:0&0&1\end{pmatrix}-\begin{pmatrix}3&0&0\\ \:\:0&3&0\\ \:\:0&0&3\end{pmatrix}=\begin{pmatrix}-2&1+i&0\\ 1-i&-1&0\\ 0&0&-2\end{pmatrix}\]
        
        
        De este modo para encontrar $(D-3I)\Vec{v}=0$, damos $\vec{v}=(x,y,z)$ e igualamos al vector 0, es decir:
        \[\begin{pmatrix}0\\0\\0\end{pmatrix}=\begin{pmatrix}-2&1+i&0\\ 1-i&-1&0\\ 0&0&-2\end{pmatrix}\begin{pmatrix}x\\y\\z\end{pmatrix}=\begin{pmatrix}-2x+(1+i)y\\(1-i)x-y\\ -2z\end{pmatrix}\]
        De esta forma tenemos el sistema:
        \begin{eqnarray*}
        -2x+(1+i)y&=&0\\
        (1-i)x-y&=&0        \\
        2z&=&0
        \end{eqnarray*}
        Tenemos directo que $z=0$, entonces observamos que la primer y segunda son multiplos d euna misma ecuaci\'on , por tanto despejando de la primer, tenemos $2x=(1+i)y$, por lo tanto tomamos a $y$ como variable libre, de modo que $\vec{v}=(x,y,z)=y(1+i,2,0)$.
        
        
        \[E(2)=<(1+i,2,0)>\]
        y $\text{dim}(E(1))=1=\text{ma}(1)$. 
        Normalizando el vector:
        \[v_3=\frac{1}{\Vert v_3\Vert}v'_3=\frac{1}{\sqrt{1^2+1^2+2^2}}(1+i,2,0)=\frac{1}{\sqrt{6}}(1+i,2,0)\]
    
        
    \end{itemize}
    Como para todo valor propio $\text{dim}(E(\lambda))=\text{ma}(\lambda)$, llegamos a que la matriz es diagonalizable y por lo tanto tiene descomoposici\'on espectral, de modo que podamos escibir a $A$ como:
    \[D=\lambda_1D_1+\lambda_2D_2+\lambda_3D_3\]
    Donde $D_i=\vec{v_i}\overline{\vec{v_i}}^T$, de este modo calculandolos:
    \begin{itemize}
        \item \[D_1=\left(\frac{1}{\sqrt{3}}\right)^2\begin{pmatrix}-1-i\\1\\0\end{pmatrix}\begin{pmatrix}-1+i&1&0\end{pmatrix}=\frac{1}{3}\begin{pmatrix}2i&-1-i&0\\-1-i&1&0\\0&0&0\end{pmatrix}\]
        \item \[D_2=\begin{pmatrix}0\\0\\1\end{pmatrix}\begin{pmatrix}0&0&1\end{pmatrix}=\begin{pmatrix}0&0&0\\0&0&0\\0&0&1\end{pmatrix}\]
        \item \[D_3=\left(\frac{1}{\sqrt{6}}\right)^2\begin{pmatrix}1+i\\2\\0\end{pmatrix}\begin{pmatrix}1-i&2&0\end{pmatrix}=\frac{1}{6}\begin{pmatrix}2i&2+2i&0\\2-2i&4&0\\0&0&0\end{pmatrix}\]
        
    \end{itemize}
    
    De esta forma la descomposición espectral queda como:\
    \[D=\lambda_1D_1+\lambda_2D_2+\lambda_3D_3=0\frac{1}{3}\begin{pmatrix}2i&-1-i&0\\-1-i&1&0\\0&0&0\end{pmatrix}+\begin{pmatrix}0&0&0\\0&0&0\\0&0&1\end{pmatrix}+3\frac{1}{6}\begin{pmatrix}2i&2+2i&0\\2-2i&4&0\\0&0&0\end{pmatrix}\]\[=\begin{pmatrix}0&0&0\\0&0&0\\0&0&1\end{pmatrix}+\frac{1}{2}\begin{pmatrix}2i&2+2i&0\\2-2i&4&0\\0&0&0\end{pmatrix}=\begin{pmatrix}1&1+i&0\\ \:\:1-i&2&0\\ \:\:0&0&1\end{pmatrix}\]
\end{itemize}
\section{Sean $\sigma_1, \sigma_2, \sigma_3$ las matrices de Pauli vistas en clase. Verificar que:}

Primeramente vamos a definir las matrices de Pauli y despu\'es sustituiremos. Entonces por lo visto en clase: 
\[ \sigma_1 = \begin{pmatrix} 0 & 1 \\ 1 & 0 \end{pmatrix}\hspace{1.5cm}\sigma_2 = \begin{pmatrix} 0 & -i \\ i & 0 \end{pmatrix} \hspace{1.5cm} \sigma_3 = \begin{pmatrix} 1 & 0 \\ 0 & -1 \end{pmatrix}\]
\begin{itemize}
    \item [$a)$] $\sigma_1^2 = \sigma_2^2 =\sigma_3^2 = -i \sigma_1 \sigma_2 \sigma_3 = I$. (La matriz identidad). \\\\
    \textbf{Soluci\'on 11.a:}\\
    Primero elevaremos al cuadrado cada matriz, de manera que tenemos que: 

\[\sigma_1^2 = \begin{pmatrix} 0 & 1 \\ 1 & 0 \end{pmatrix}\begin{pmatrix} 0 & 1 \\ 1 & 0 \end{pmatrix}=\begin{pmatrix} 0(0)+1(1) & 0(1)+1(0) \\ 1(0)+0(1) & 1(1)+0(0) \end{pmatrix}=\begin{pmatrix} 1 & 0 \\ 0 & 1 \end{pmatrix}\]
\[\sigma_2^2 = \begin{pmatrix} 0 & -i \\ i & 0 \end{pmatrix}\begin{pmatrix} 0 & -i \\ i & 0 \end{pmatrix}=\begin{pmatrix} 0(0)+-i(i) & 0(-i)+-i(0) \\ i(0)+0(i) & i(-i)+0(0) \end{pmatrix}=\begin{pmatrix} 1 & 0 \\ 0 & 1 \end{pmatrix}\]
\[\sigma_3^2 = \begin{pmatrix} 1 & 0 \\ 0 & -1 \end{pmatrix}\begin{pmatrix} 1 & 0 \\ 0 & -1 \end{pmatrix}=\begin{pmatrix} 1(1)+0(0) & 1(0)+0(-1) \\ 0(1)+-1(0) & 0(0)+-1(-1) \end{pmatrix}=\begin{pmatrix} 1 & 0 \\ 0 & 1 \end{pmatrix}\]
Para seguir con la soluci\'on de este ejercicio lo que haremos ser\'a realizar la multiplicaci\'on \textbf{$-i \sigma_1 \sigma_2 \sigma_3$} y usamos que la multiplicaci\'on de matrices es asociativa, al igual que el producto por escalar:\\
\[-i \sigma_1 \sigma_2 \sigma_3=-i[ (\sigma_1 \sigma_2) \sigma_3]=-i\left[\left(\begin{pmatrix} 0 & 1\\ 1 & 0\end{pmatrix} \begin{pmatrix} 0 & -i \\ i & 0\end{pmatrix}\right) \begin{pmatrix} 1 & 0 \\ 0 & -1 \end{pmatrix}\right]=-i\left[\begin{pmatrix} 0(0)+1(i) & 0(-i)+1(0)\\ 1(0)+0(i) & 1(-i)+0(0)\end{pmatrix}\begin{pmatrix} 1 & 0 \\ 0 & -1 \end{pmatrix}\right]\]\[=-i\begin{pmatrix} i & 0\\ 0 & -i\end{pmatrix} \begin{pmatrix} 1 & 0 \\ 0 & -1 \end{pmatrix}=-i(i)\begin{pmatrix} 1 & 0\\ 0 & -1\end{pmatrix} \begin{pmatrix} 1 & 0 \\ 0 & -1 \end{pmatrix} = -(-1)\sigma_3^2=I\]
Pero por lo anterior sabemos que $\sigma_3^2=I$, por tanto podemos asegurar que:
\[\sigma_1^2 = \sigma_2^2 =\sigma_3^2 = -i \sigma_1 \sigma_2 \sigma_3 = I\]\qed
    
    
\item [$b)$] $\sigma_1 \sigma_2 + \sigma_2\sigma_1 = 0.$\\\\
    \textbf{Soluci\'on 11.b:}\\
De igual manera definiremos las matrices, sustituiremos y despu\'es realizaremos las operaciones requeridas, ya sea multiplicando, sumando o factorizando escalares para comprobar si se cumple el resultado deseado.

\[\sigma_1 \sigma_2 + \sigma_2\sigma_1= \begin{pmatrix} 0 & 1\\ 1 & 0
\end{pmatrix} \begin{pmatrix} 0 & -i \\ i & 0\end{pmatrix} + \begin{pmatrix} 0 & -i \\ i & 0 \end{pmatrix} \begin{pmatrix} 0 & 1 \\ 1 & 0 \end{pmatrix}= \begin{pmatrix} 0(0)+1(i) & 0(-i)+1(0)\\ 1(0)+0(i) & 1(-i)+0(0)\end{pmatrix}+\begin{pmatrix} 0(0)+1(-i) & 0(1)+-i(0)\\ i(0)+0(1) & i(1)+0(0)\end{pmatrix}\]\[=\begin{pmatrix} i & 0\\ 0 & -i\end{pmatrix}+\begin{pmatrix} -i & 0\\ 0 & +i\end{pmatrix}=\begin{pmatrix} i & 0 \\ 0 & -i\end{pmatrix} -\begin{pmatrix} i & 0 \\ 0 & -i \end{pmatrix} = \begin{pmatrix} 0 & 0 \\ 0 & 0\end{pmatrix}\]
\[\therefore \sigma_1 \sigma_2 + \sigma_2\sigma_1 = 0\]\qed
    
\item [$c)$] $\sigma_x\sigma_y=-\sigma_y\sigma_x=i\sigma_z, x,y,z \in {1,2,3}.$\\\\
    \textbf{Soluci\'on 11.c:}\\
Los casos que debemos demostrar son $(\sigma_1,\sigma_2), (\sigma_3,\sigma_1)$ y $(\sigma_2,\sigma_3)$, por lo tanto:
\begin{itemize}
    \item $\sigma_1\sigma_2=-\sigma_2\sigma_1=i\sigma_3$
    \[\sigma_1\sigma_2= \begin{pmatrix} 0 & 1\\ 1 & 0
\end{pmatrix} \begin{pmatrix} 0 & -i \\ i & 0\end{pmatrix}= \begin{pmatrix} 0(0)+1(i) & 0(-i)+1(0)\\ 1(0)+0(i) & 1(-i)+0(0)\end{pmatrix}=\begin{pmatrix} i & 0 \\ 0 & -i\end{pmatrix}\]
\[-\sigma_2\sigma_1=-\begin{pmatrix} 0 & -i \\ i & 0 \end{pmatrix} \begin{pmatrix} 0 & 1 \\ 1 & 0 \end{pmatrix}= -\begin{pmatrix} 0(0)+1(-i) & 0(1)+-i(0)\\ i(0)+0(1) & i(1)+0(0)\end{pmatrix}=-\begin{pmatrix} -i & 0\\ 0 & +i\end{pmatrix}=\begin{pmatrix} i & 0 \\ 0 & -i \end{pmatrix}\]
\[i\sigma_3 =i \begin{pmatrix} 1 & 0 \\ 0 & -1 \end{pmatrix}=\begin{pmatrix} i & 0 \\ 0 & -i \end{pmatrix}\]
\[\therefore \sigma_1\sigma_2=-\sigma_2\sigma_1=i\sigma_3\]
    \item $\sigma_3\sigma_1=-\sigma_1\sigma_3=i\sigma_2$
\[\sigma_3\sigma_1=\begin{pmatrix} 1 & 0 \\ 0 & -1\end{pmatrix} \begin{pmatrix} 0 & 1 \\ 1 & 0 \end{pmatrix}= \begin{pmatrix} 1(0)+0(1) & 1(1)+0(0)\\ 0(0)+1(-1) & 0(1)+0(-1)\end{pmatrix}=\begin{pmatrix} 0 & 1\\ -1 & 0\end{pmatrix}\]
 \[-\sigma_1\sigma_3= -\begin{pmatrix} 0 & 1\\ 1 & 0
\end{pmatrix} \begin{pmatrix} 1 & 0 \\ 0 & -1\end{pmatrix}= -\begin{pmatrix} 0(1)+1(0) & 0(0)+1(-1)\\ 1(1)+0(0) & 1(0)+0(-1)\end{pmatrix}=-\begin{pmatrix} 0 & -1 \\ 1 & 0 \end{pmatrix}=\begin{pmatrix} 0 & 1 \\ -1 & 0 \end{pmatrix}\]
\[i\sigma_2 =i \begin{pmatrix} 0 & -i \\ i & 0 \end{pmatrix}=\begin{pmatrix} 0 & 1 \\ -1 & 0 \end{pmatrix}\]
\[\therefore \sigma_3\sigma_1=-\sigma_1\sigma_3=i\sigma_2\]
    \item $\sigma_2\sigma_3=-\sigma_3\sigma_2=i\sigma_1$
\[\sigma_2\sigma_3=\begin{pmatrix} 0 & -i \\ i & 0 \end{pmatrix}  \begin{pmatrix} 1 & 0\\ 0 & -1
\end{pmatrix}= \begin{pmatrix} 1(0)+0(-i) & 0(0)+-i(-1)\\ i(1)+0(0) & i(0)+0(-1)\end{pmatrix}=\begin{pmatrix} 0 & i \\ i & 0 \end{pmatrix}\]
\[-\sigma_3\sigma_2=- \begin{pmatrix} 1 & 0\\ 0 & -1
\end{pmatrix} \begin{pmatrix} 0 & -i \\ i & 0\end{pmatrix}= -\begin{pmatrix} 1(0)+0(i) & 1(-i)+0(0)\\ 0(0)+-1(i) & 0(-i)+-1(0)\end{pmatrix}=-\begin{pmatrix} 0 & -i \\ -i & 0 \end{pmatrix}=\begin{pmatrix} 0 & i \\ i & 0 \end{pmatrix}\]
\[i\sigma_1 =i \begin{pmatrix} 0 & 1 \\ 1 & 0 \end{pmatrix}=\begin{pmatrix} 0 & i \\ i & 0 \end{pmatrix}\]
\[\therefore \sigma_2\sigma_3=-\sigma_3\sigma_2=i\sigma_1\]
\end{itemize}
\end{itemize}


%\[\sigma_x = \begin{pmatrix} 0 & 1 \\ 1 & 0 \end{pmatrix} ,  \sigma_y = \begin{pmatrix} 0 & -i \\ i & 0 \end{pmatrix} , \sigma_z = \begin{pmatrix} 1 & 0 \\ 0 & -1 \end{pmatrix}\]\[\Longrightarrow \sigma_x\sigma_y = \begin{pmatrix} 0 & 1\\ 1 & 0 \end{pmatrix} \begin{pmatrix} 0 & -i \\ i & 0\end{pmatrix} = \begin{pmatrix} i & 0 \\ 0 & -i\end{pmatrix}\]\[\Longrightarrow -\sigma_y\sigma_x = - \begin{pmatrix} 0 & -i \\ i & 0\end{pmatrix} \begin{pmatrix} 0 & 1\\ 1 & 0 \end{pmatrix} = \begin{pmatrix} i & 0 \\ 0 & -i\end{pmatrix}\]\[\Longrightarrow i\sigma_z = \begin{pmatrix} 1 & 0 \\ 0 & -1 \end{pmatrix} = \begin{pmatrix} i & 0 \\ 0 & -i \end{pmatrix} \]Por lo tanto podemos concluir que sucede que \textbf{s\'i} se cumple la igualdad. \[\sigma_x = \begin{pmatrix} 0 & -i \\ i & 0 \end{pmatrix} ,  \sigma_y = \begin{pmatrix} 1 & 0 \\ 0 & -1 \end{pmatrix} , \sigma_z = \begin{pmatrix} 0 & 1 \\ 1 & 0 \end{pmatrix}\] \[\Longrightarrow \sigma_x\sigma_y=\begin{pmatrix} 0 & -i \\ i & 0 \end{pmatrix}\begin{pmatrix} 1 & 0 \\ 0 & -1 \end{pmatrix}= \begin{pmatrix} 0 & -i \\ i & 0 \end{pmatrix}\]\[\Longrightarrow -\sigma_y\sigma_x= -\begin{pmatrix} 1 & 0 \\ 0 & -1 \end{pmatrix}\begin{pmatrix} 0 & -i \\ i & 0 \end{pmatrix}= \begin{pmatrix}0 & -i \\ -i & 0\end{pmatrix}\]\[\Longrightarrow i\sigma_z = i\begin{pmatrix} 0 & 1 \\ 1 & 0 \end{pmatrix} = \begin{pmatrix} 0 & i \\ i & 0 \end{pmatrix}\]Por lo tanto podemos concluir que en este caso en particular \textbf{no} se cumple.\[\sigma_x =\begin{pmatrix} 1 & 0 \\ 0 & -1 \end{pmatrix} ,  \sigma_y = \begin{pmatrix} 0 & 1 \\ 1 & 0 \end{pmatrix} , \sigma_z = \begin{pmatrix} 0 & -i \\ i & 0 \end{pmatrix}\] \[\Longrightarrow \sigma_x\sigma_y= \begin{pmatrix} 1 & 0 \\ 0 & -1 \end{pmatrix}\begin{pmatrix} 0 & 1 \\ 1 & 0 \end{pmatrix} = \begin{pmatrix} 0 & 1 \\ -1 & 0 \end{pmatrix}\]\[\Longrightarrow -\sigma_y\sigma_x= -\begin{pmatrix} 0 & 1 \\ 1 & 0 \end{pmatrix}\begin{pmatrix} 1 & 0 \\ 0 & -1 \end{pmatrix} = \begin{pmatrix}0 & 1 \\ -1 & 0 \end{pmatrix}\]\[\Longrightarrow i\sigma_z =i\begin{pmatrix} 0 & -i \\ i & 0 \end{pmatrix} = \begin{pmatrix} 0 & 1 \\ -1 & 0 \end{pmatrix}\]Por lo tanto podemos concluir que en este caso en particular \textbf{s\'i} se cumple. \[\sigma_x =\begin{pmatrix} 0 & 1 \\ 1 & 0 \end{pmatrix} ,  \sigma_y = \begin{pmatrix} 1 & 0 \\ 0 & -1 \end{pmatrix} , \sigma_z = \begin{pmatrix} 0 & -i \\ i & 0 \end{pmatrix}\]\[\Longrightarrow \sigma_x\sigma_y= \begin{pmatrix} 0 & 1 \\ 1 & 0 \end{pmatrix}\begin{pmatrix} 1 & 0 \\ 0 & -1 \end{pmatrix} = \begin{pmatrix} 0 & -1 \\ 1 & 0 \end{pmatrix}\]\[\Longrightarrow -\sigma_y\sigma_x= -\begin{pmatrix} 1 & 0 \\ 0 & -1 \end{pmatrix}\begin{pmatrix} 0 & 1 \\ 1 & 0 \end{pmatrix} = \begin{pmatrix} 0 & -1 \\ 1 & 0 \end{pmatrix}\]\[\Longrightarrow i\sigma_z = i\begin{pmatrix} 0 & -i \\ i & 0 \end{pmatrix} = \begin{pmatrix} 0 & 1 \\ -1 & 0 \end{pmatrix}\]Por lo tanto podemos concluir que en este caso en particular \textbf{no} se cumple.\[\sigma_x = \begin{pmatrix} 1 & 0 \\ 0 & -1 \end{pmatrix},  \sigma_y = \begin{pmatrix} 0 & -i \\ i & 0 \end{pmatrix}, \sigma_z = \begin{pmatrix} 0 & 1 \\ 1 & 0 \end{pmatrix}\]\[\Longrightarrow \sigma_x\sigma_y= \begin{pmatrix} 1 & 0 \\ 0 & -1 \end{pmatrix}\begin{pmatrix} 0 & -i \\ i & 0 \end{pmatrix} = \begin{pmatrix}0 & -i \\ -i & 0 \end{pmatrix}\]\[\Longrightarrow -\sigma_y\sigma_x= -\begin{pmatrix} 0 & -i \\ i & 0 \end{pmatrix}\begin{pmatrix} 1 & 0 \\ 0 & -1 \end{pmatrix}=\begin{pmatrix}0 & -i \\ -i & 0 \end{pmatrix}\]\[\Longrightarrow i\sigma_z = i\begin{pmatrix} 0 & 1 \\ 1 & 0 \end{pmatrix}=\begin{pmatrix}0 & i \\ i & 0 \end{pmatrix}\]Por lo tanto podemos concluir que en este caso en particular \textbf{no} se cumple.\[\sigma_x = \begin{pmatrix} 0 & -i \\ i & 0 \end{pmatrix}  ,\sigma_y = \begin{pmatrix} 0 & 1 \\ 1 & 0 \end{pmatrix} ,\sigma_z = \begin{pmatrix} 1 & 0 \\ 0 & -1 \end{pmatrix}\] \[\Longrightarrow \sigma_x\sigma_y= \begin{pmatrix} 0 & -i \\ i & 0 \end{pmatrix}\begin{pmatrix} 0 & 1 \\ 1 & 0 \end{pmatrix} = \begin{pmatrix}-i & 0 \\ 0 & i \end{pmatrix}\]\[\Longrightarrow -\sigma_y\sigma_x= -\begin{pmatrix} 0 & 1 \\ 1 & 0 \end{pmatrix}\begin{pmatrix} 0 & -i \\ i & 0 \end{pmatrix} = \begin{pmatrix}-i & 0 \\ 0 & i \end{pmatrix}\]\[\Longrightarrow i\sigma_z= i\begin{pmatrix} 1 & 0 \\ 0 & -1 \end{pmatrix} = \begin{pmatrix}i & 0 \\ 0 & -i\end{pmatrix}\]Por lo tanto podemos concluir que en este caso en particular \textbf{no} se cumple.



\end{document}
