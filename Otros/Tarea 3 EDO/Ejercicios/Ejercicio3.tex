\textcolor{blue}{Ejercicio 3.} Determine 2 soluciones linealmente independientes en forma de series de potencias de la ecuación diferencial
\begin{equation}
    (x^{2}+2)y'' +3xy' -y = 0 
    \label{eq:ecuación_ejer_3}
\end{equation}
en torno al punto ordinario x=0.

Ya que tenemos un punto ordinario, proponemos que $y(x)$ sea de la forma:

\begin{equation}
    y(x) = \sum_{n = 0}^{\infty}c_{n}x^{n}
    \label{eq:y(x)}
\end{equation}

Por lo que se tiene que 
\begin{equation}
    y'= \sum_{n = 1}^{\infty}nc_{n}x^{n -1} \hspace{1cm} y \hspace{1cm} y'' = \sum_{n = 2}^{\infty}n(n -1)c_{n}x^{n -2}
    \label{eq:derivadas}
\end{equation}

Si sustituimos $y,y',y''$ en \ref{eq:ecuación_ejer_3}, tenemos que:
\begin{equation*}
\begin{split}
         (x^{2}+2) \left[\sum_{n = 2}^{\infty}n(n -1)c_{n}x^{n -2}\right] +3x \left[\sum_{n = 1}^{\infty}nc_{n}x^{n -1}\right]- \left[ \sum_{n = 0}^{\infty}c_{n}x^{n} \right] = 0 \\
x^{2}\left[\sum_{n = 2}^{\infty}n(n -1)c_{n}x^{n -2}\right]
+2\left[\sum_{n = 2}^{\infty}n(n -1)c_{n}x^{n -2}\right]
+3x \left[\sum_{n = 1}^{\infty}nc_{n}x^{n -1}\right]
-\left[ \sum_{n = 0}^{\infty}c_{n}x^{n} \right] = 0 \\
        \left[\sum_{n = 2}^{\infty}n(n -1)c_{n}x^{n}\right]
        +2\left[\sum_{n = 2}^{\infty}n(n -1)c_{n}x^{n -2}\right]
        +3 \left[\sum_{n = 1}^{\infty}nc_{n}x^{n}\right]
        -\left[ \sum_{n = 0}^{\infty}c_{n}x^{n} \right] = 0 \\
\end{split}
\end{equation*}
Ahora bien, si para la primera suma establecemos que $k=n$, para la segunda $k=n-2$ , para la tercera  y cuarta que $k=n$, se tiene que:
\begin{equation*}
\centering
    \begin{split}
        &\sum_{k=2}^{\infty} k(k-1) c_{k} x^{k}+2 \sum_{k=0}^{\infty}(k+2)(k+2-1) c_{k+2} x^{k}+3 \sum_{k=1}^{\infty} k c_{k} x^{k}-\sum_{k=0}^{\infty} c_{k} x^{k}=0 \\
        &\sum_{k=2}^{\infty} k(k-1) c_{k} x^{k}+2 \sum_{k=0}^{\infty}(k+2)(k+1) c_{k+2} x^{k}+3 \sum_{k=1}^{\infty} k c_{k} x^{k}-\sum_{k=0}^{\infty} c_{k} x^{k}=0 \\
        &\sum_{k=2}^{\infty} k(k-1) c_{k} x^{k}+2(0+2)(0+1) c_{0+2} x^{0}+2(1+2)(1+1) c_{1+2} x^{1}+\sum_{k=2}^{\infty}(k+2)(k+1) c_{k+2} x^{k} \\
        &+3(1) c_{1} x^{1}+3 \sum_{k=2}^{\infty} k c_{k} x^{k}-c_{0} x^{0}-c_{1} x^{1}-\sum_{k=2}^{\infty} c_{k} x^{k}=0 \\
    \end{split}
\end{equation*}
\begin{equation*}
       \sum_{k=2}^{\infty} k(k-1) c_{k} x^{k}+4 c_{2}+12 c_{3} x+\sum_{k=2}^{\infty}(k+2)(k+1) c_{k+2} x^{k}+3 c_{1} x+3 \sum_{k=2}^{\infty} k c_{k} x^{k}-c_{0}-c_{1} x-\sum_{k=2}^{\infty} c_{k} x^{k}=0
\end{equation*}
Ahora bien podemos juntar la suma y eliminar términos semejantes, entonces
\begin{equation*}
    \begin{split}
        4 c_{2}+12 c_{3} x+2 c_{1} x-c_{0}+\sum_{k=2}^{\infty}\left[k(k-1) c_{k}+(k+2)(k+1) c_{k+2}+3 k c_{k}-c_{k}\right] x^{k}=0 \\
    \end{split}
\end{equation*}
Si comparamos los coeficientes tenemos que:
\begin{equation*}
\begin{split}
     &4c_{2}-c_{0}=0 \Rightarrow c_{2}=\frac{1}{4}c_{0}\\
     &12c_{3}+2c_{1}=0 \Rightarrow c_{3}=-\frac{1}{6}c_{1}\\
     &k(k-1) c_{k}+2(k+2)(k+1) c_{k+2}+3 k c_{k}-c_{k}=0\\
     &2\left(k^{2}+3 k+2\right) c_{k+2}=\left(-k^{2}+k-3 k+1\right) c_{k}\\
     &2\left(k^{2}+3 k+2\right) c_{k+2}=\left(-k^{2}-2 k+1\right) c_{k}\\
    &\Rightarrow 
     c_{k+2}=\frac{-k^{2}-2 k+1}{2\left(k^{2}+3 k+2\right)} c_{k}, k=2,3,4
\end{split}
\end{equation*}
Ahora calculando para $k=2,3,4$, se tiene que

\begin{equation*}
\begin{split}
    &c_{4}=\frac{-7}{24}c_{2}=\frac{-7}{24}(\frac{1}{4}c_{0})=-\frac{7}{96}c_{0}\\
    &c_{5}=\frac{7}{120}c_{1}\\
    &c_{6}=\frac{161}{5760}c_{0}\\
\end{split}
\end{equation*}
Debido a la ecuación \ref{eq:y(x)}, entonces
\begin{equation*}
    y(x)=c_{0}+c_{1}x+c_{2}x^{2}+c_{3}x^{3}+c_{4}x^{4}
\end{equation*}
Entonces observando el comportamiento de los coeficientes, podemos ver que si $k$ es par, entonces lo acompañara $c_{0}$ mientras que si es impar, se tendrá a $c_{1}$, entonces factorizando:
\begin{equation*}
    y(x)=c_{0}+c_{1}x+\frac{1}{4}c_{0}x^{2}-\frac{1}{6}c_{1}x^{3}-\frac{7}{96}c_{0}x^{4}+...
\end{equation*}
\begin{equation*}
    y(x)=c_{0}(\frac{1}{4}x^{2}-\frac{7}{96}x^{4}+\frac{161}{5760}x^{6}+...)+c_{1}(x-\frac{1}{6}x^{3}+\frac{7}{120}x^{5}+...)
\end{equation*}
\begin{equation}
    y_{1}(x)=c_{0}(\frac{1}{4}x^{2}-\frac{7}{96}x^{4}+\frac{161}{5760}x^{6}+...)
    \label{eq:y1}
\end{equation}
\begin{equation}
     y_{2}(x)=c_{1}(x-\frac{1}{6}x^{3}+\frac{7}{120}x^{5}+...)
     \label{eq:y2}
\end{equation}
Debido a que no comparten potencias, son linealmente independientes las ecuaciones \ref{eq:y1} y \ref{eq:y2},siendo soluciones.