\justify
\textcolor{blue}{Ejercicio 4.} Un modelo de un par de fuerzas convencionales en combate proporciona el siguiente sistema: 
$$\begin{pmatrix}
\frac{dx_1}{dt}\\
\frac{dx_1}{dt}
\end{pmatrix} = \begin{pmatrix}
-a & -b\\
-c & -d
\end{pmatrix} \begin{pmatrix}
x_1\\
x_2
\end{pmatrix} + \begin{pmatrix}
p\\
q
\end{pmatrix}$$
Las variables $x_1(t)$ y $x_2(t)$ representan la fuerza de los ejércitos enemigos en el instante $t$. Los términos $-ax_1$ y $-dx_2$ representan las tasas de pérdidas operativas, y los términos $-bx_2$ y $-cx_1$ representan las tasas de pérdida en combate para las tropas $x_1$ y $x_2$, respectivamente. Las constantes $p$ y $q$ representan las tasas respectivas de refuerzo. Sea $a=1, b=4, c=3, d=2$ y $p=q=5$. Resuelva el problema con valores iniciales adecuado para determinar cuál de los ejércitos ganará si\\

(a)$x_1(0) = 20, x_2(0) = 20$\\

(b)$x_1(0) = 21, x_2(0) = 20$\\

(c)$x_1(0) = 20, x_2(0) = 21$\\
\\
$\mathbf{Solución:}$
\\
Sustituyendo los valores dados en el sistema, tenemos
$$\begin{pmatrix}
\frac{dx_1}{dt}\\
\frac{dx_1}{dt}
\end{pmatrix} = \begin{pmatrix}
-1 & -4\\
-3 & -2
\end{pmatrix} \begin{pmatrix}
x_1\\
x_2
\end{pmatrix} + \begin{pmatrix}
5\\
5
\end{pmatrix}$$
Calculamos los valores propios de 
$$\begin{pmatrix}
\frac{dx_1}{dt}\\
\frac{dx_1}{dt}
\end{pmatrix} = \begin{pmatrix}
-1 & -4\\
-3 & -2
\end{pmatrix} \begin{pmatrix}
x_1\\
x_2
\end{pmatrix}$$
obtenemos
$$det(A - \lambda I) = \left|\begin{matrix}
-1- \lambda & -4\\
-3 & -2- \lambda
\end{matrix} \right| = (-1- \lambda)(-2- \lambda) - (-4) (-3) = \lambda ^2 + 3\lambda - 10$$
Igualando a cero
$$\lambda ^2 + 3\lambda - 10 = 0$$
$$(\lambda - 2)(\lambda  + 5) = 0$$
donde
$$\lambda _1 = 2 , \lambda _2 = -5$$
Ahora buscamos los vectores propios con
$$det(A - \lambda I) = \begin{pmatrix}
-1- \lambda & -4\\
-3 & -2- \lambda
\end{pmatrix}$$ Para $\lambda_1 = 2$, tenemos
\begin{equation*}
     (A-2I) =
     \begin{pmatrix}
     -1-2 & -4 \\
     -3 & -2-2
     \end{pmatrix}
     =
     \begin{pmatrix}
     -3 & -4\\
     -3 & -4
     \end{pmatrix}
\end{equation*}
Buscamos $ker(A-2I)$
\begin{equation*}
\begin{pmatrix}
     -3 & -4\\
     -3 & -4
     \end{pmatrix}
\begin{pmatrix}
v_{1}\\
v_{2}  
\end{pmatrix} =
\begin{pmatrix}
0\\
0  
\end{pmatrix}
\end{equation*} donde
$$\Rightarrow -3v_1-4v_2 = 0$$
$$\Rightarrow -3v_1 = 4v_2$$ Por lo cual, 
$$v_2 = \begin{pmatrix}
3\\
-4
\end{pmatrix}$$ Para $\lambda_2 = -5$, tenemos
\begin{equation*}
     (A+5I) =
     \begin{pmatrix}
     -1-(-5) & -4 \\
     -3 & -2-(-5)
     \end{pmatrix}
     =
     \begin{pmatrix}
     4 & -4\\
     -3 & 3
     \end{pmatrix}
\end{equation*}
Buscamos $ker(A+5I)$
\begin{equation*}
\begin{pmatrix}
     4 & -4\\
     -3 & 3
     \end{pmatrix}
\begin{pmatrix}
v_{1}\\
v_{2}  
\end{pmatrix} =
\begin{pmatrix}
0\\
0  
\end{pmatrix}
\end{equation*} donde
$$\Rightarrow 4v_1-4v_2 = 0$$
$$\Rightarrow 4v_1 = 4v_2$$ Por lo cual, 
$$v_1 = \begin{pmatrix}
1\\
1
\end{pmatrix}$$ Por los datos anteriores, tenemos que 
$$x_C = c_1 \begin{pmatrix}
     1\\
     1
     \end{pmatrix}e^{2t} + c_2 \begin{pmatrix}
     3\\
     -4
     \end{pmatrix}e^{-5t}$$
Ahora buscamos calcular la solución particular
$$x_P = u_1 \begin{pmatrix}
     1\\
     1
     \end{pmatrix}e^{2t} + u_2 \begin{pmatrix}
     3\\
     -4
     \end{pmatrix}e^{-5t}$$ entonces tenemos
$$u'_1 \begin{pmatrix}
     1\\
     1
     \end{pmatrix}e^{2t} + u'_2 \begin{pmatrix}
     3\\
     -4
     \end{pmatrix}e^{-5t} = \begin{pmatrix}
     5\\
     5
     \end{pmatrix}$$
$$\begin{pmatrix}
     e^{2t}\\
     e^{2t}
     \end{pmatrix}u'_1 + \begin{pmatrix}
     3e^{-5t}\\
     -4e^{-5t}
     \end{pmatrix}u'_2 = \begin{pmatrix}
     5\\
     5
     \end{pmatrix}$$ 
donde\\
$u'_1 = \frac{\left|\begin{matrix}
5 & 3e^{-5t}\\
5 & -4e^{-5t}
\end{matrix} \right|}{\left|\begin{matrix}
e^{2t} & 3e^{-5t}\\
e^{2t} & -4e^{-5t}
\end{matrix} \right|} = \frac{-35e^{-5t}}{-7e^{-3t}} = 5$\\
$u'_2 = \frac{\left|\begin{matrix}
e^{2t} & 5\\
e^{2t} & 5
\end{matrix} \right|}{\left|\begin{matrix}
e^{2t} & 3e^{-5t}\\
e^{2t} & -4e^{-5t}
\end{matrix} \right|} = 0$
$$u'_1 = 5 \Rightarrow u_1 = \int 5dt \Rightarrow u_1 = 5t$$
$$u'_2 = 0 \Rightarrow u_2 = \int 0dt \Rightarrow u_2 = 0$$
Sustituyendo los valores encontrados para $c_1$ y $c_2$
$$x_P = 5t \begin{pmatrix}
     1\\
     1
     \end{pmatrix}e^{2t} + 0 \begin{pmatrix}
     3\\
     -4
     \end{pmatrix}e^{-5t}$$
obtenemos
$$x_P = \begin{pmatrix}
     5t\\
     5t
     \end{pmatrix}e^{2t}$$ 
Por lo tanto, la solución es de la forma $x = x_C + x_P$
$$c_1 \begin{pmatrix}
     1\\
     1
     \end{pmatrix}e^{2t} + c_2 \begin{pmatrix}
     3\\
     -4
     \end{pmatrix}e^{-5t} + \begin{pmatrix}
     5t\\
     5t
     \end{pmatrix}e^{2t}$$
(a) Pero tenemos que si $x_1(0) = 20$ y $x_2(0) = 20$, entonces sustituyendo los valores iniciales dados
$$\begin{pmatrix}
x_1(0)\\
x_2(0)
\end{pmatrix} = c_1 \begin{pmatrix}
1\\
1
\end{pmatrix}e^{2(0)} + c_2 \begin{pmatrix}
3\\
-4
\end{pmatrix}e^{-5(0)} + \begin{pmatrix}
5(0)\\
5(0)
\end{pmatrix}e^{2(0)} = \begin{pmatrix}
20\\
20
\end{pmatrix}$$
obtenemos
$$c_1 \begin{pmatrix}
1\\
1
\end{pmatrix} + c_2 \begin{pmatrix}
3\\
-4
\end{pmatrix} = \begin{pmatrix}
20\\
20
\end{pmatrix}$$
Para determinar $c_1$ y $c_2$, tenemos que
\\
$c_1 = \frac{\left|\begin{matrix}
20 & 3\\
20 & -4
\end{matrix} \right|}{\left|\begin{matrix}
1 & 3\\
1 & -4
\end{matrix} \right|} = \frac{-140}{-7} = 20$\\
$c_2 = \frac{\left|\begin{matrix}
1 & 20\\
1 & 20
\end{matrix} \right|}{\left|\begin{matrix}
1 & 3\\
1 & -4
\end{matrix} \right|} = 0$\\
Sustituyendo los valores de $c_1$ y $c_2$, tenemos
$$\begin{pmatrix}
\frac{dx_1}{dt}\\
\frac{dx_1}{dt}
\end{pmatrix} = 20 \begin{pmatrix}
     1\\
     1
     \end{pmatrix}e^{2t} + 0 \begin{pmatrix}
     3\\
     -4
     \end{pmatrix}e^{-5t} + \begin{pmatrix}
     5t\\
     5t
     \end{pmatrix}e^{2t}$$
Por lo tanto, la solución para estos valores iniciales es:
$$\begin{pmatrix}
\frac{dx_1}{dt}\\
\frac{dx_1}{dt}
\end{pmatrix} = 20 \begin{pmatrix}
     1\\
     1
     \end{pmatrix}e^{2t} + \begin{pmatrix}
     5t\\
     5t
     \end{pmatrix}e^{2t}$$
(b) Ahora tenemos que si $x_1(0) = 21$ y $x_2(0) = 20$, entonces sustituyendo los valores iniciales dados
$$\begin{pmatrix}
x_1(0)\\
x_2(0)
\end{pmatrix} = c_1 \begin{pmatrix}
1\\
1
\end{pmatrix}e^{2(0)} + c_2 \begin{pmatrix}
3\\
-4
\end{pmatrix}e^{-5(0)} + \begin{pmatrix}
5(0)\\
5(0)
\end{pmatrix}e^{2(0)} = \begin{pmatrix}
21\\
20
\end{pmatrix}$$
obtenemos
$$c_1 \begin{pmatrix}
1\\
1
\end{pmatrix} + c_2 \begin{pmatrix}
3\\
-4
\end{pmatrix} = \begin{pmatrix}
21\\
20
\end{pmatrix}$$
Para determinar $c_1$ y $c_2$, tenemos que
\\
$c_1 = \frac{\left|\begin{matrix}
21 & 3\\
20 & -4
\end{matrix} \right|}{\left|\begin{matrix}
1 & 3\\
1 & -4
\end{matrix} \right|} = \frac{-144}{-7} = \frac{144}{7}$\\
$c_2 = \frac{\left|\begin{matrix}
1 & 21\\
1 & 20
\end{matrix} \right|}{\left|\begin{matrix}
1 & 3\\
1 & -4
\end{matrix} \right|} = \frac{-1}{-7} = \frac{1}{7}$\\
Sustituyendo los valores de $c_1$ y $c_2$, tenemos
$$\begin{pmatrix}
\frac{dx_1}{dt}\\
\frac{dx_1}{dt}
\end{pmatrix} = \frac{144}{7} \begin{pmatrix}
     1\\
     1
     \end{pmatrix}e^{2t} + \frac{1}{7} \begin{pmatrix}
     3\\
     -4
     \end{pmatrix}e^{-5t} + \begin{pmatrix}
     5t\\
     5t
     \end{pmatrix}e^{2t}$$
Por lo tanto, la solución para estos valores iniciales es:
$$\begin{pmatrix}
\frac{dx_1}{dt}\\
\frac{dx_1}{dt}
\end{pmatrix} = \frac{144}{7} \begin{pmatrix}
     1\\
     1
     \end{pmatrix}e^{2t} + \frac{1}{7} \begin{pmatrix}
     3\\
     -4
     \end{pmatrix}e^{-5t} + \begin{pmatrix}
     5t\\
     5t
     \end{pmatrix}e^{2t}$$
(c) Por último tenemos que si $x_1(0) = 20$ y $x_2(0) = 21$, entonces sustituyendo los valores iniciales dados
$$\begin{pmatrix}
x_1(0)\\
x_2(0)
\end{pmatrix} = c_1 \begin{pmatrix}
1\\
1
\end{pmatrix}e^{2(0)} + c_2 \begin{pmatrix}
3\\
-4
\end{pmatrix}e^{-5(0)} + \begin{pmatrix}
5(0)\\
5(0)
\end{pmatrix}e^{2(0)} = \begin{pmatrix}
20\\
21
\end{pmatrix}$$
obtenemos
$$c_1 \begin{pmatrix}
1\\
1
\end{pmatrix} + c_2 \begin{pmatrix}
3\\
-4
\end{pmatrix} = \begin{pmatrix}
20\\
21
\end{pmatrix}$$
Para determinar $c_1$ y $c_2$, tenemos que
\\
$c_1 = \frac{\left|\begin{matrix}
20 & 3\\
21 & -4
\end{matrix} \right|}{\left|\begin{matrix}
1 & 3\\
1 & -4
\end{matrix} \right|} = \frac{-143}{-7} = \frac{143}{7}$\\
$c_2 = \frac{\left|\begin{matrix}
1 & 20\\
1 & 21
\end{matrix} \right|}{\left|\begin{matrix}
1 & 3\\
1 & -4
\end{matrix} \right|} = \frac{1}{-7} = -\frac{1}{7}$\\
Sustituyendo los valores de $c_1$ y $c_2$, tenemos
$$\begin{pmatrix}
\frac{dx_1}{dt}\\
\frac{dx_1}{dt}
\end{pmatrix} = \frac{143}{7} \begin{pmatrix}
     1\\
     1
     \end{pmatrix}e^{2t} - \frac{1}{7} \begin{pmatrix}
     3\\
     -4
     \end{pmatrix}e^{-5t} + \begin{pmatrix}
     5t\\
     5t
     \end{pmatrix}e^{2t}$$
Por lo tanto, la solución para estos valores iniciales es:
$$\begin{pmatrix}
\frac{dx_1}{dt}\\
\frac{dx_1}{dt}
\end{pmatrix} = \frac{143}{7} \begin{pmatrix}
     1\\
     1
     \end{pmatrix}e^{2t} - \frac{1}{7} \begin{pmatrix}
     3\\
     -4
     \end{pmatrix}e^{-5t} + \begin{pmatrix}
     5t\\
     5t
     \end{pmatrix}e^{2t}$$
     
