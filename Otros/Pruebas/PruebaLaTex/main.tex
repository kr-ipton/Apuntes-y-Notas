\documentclass{article}
\usepackage[spanish]{babel}

\title{Algoritmo para sacarle punta a un lápiz con un sacapuntas}
\author{Juárez Torres Carlos Alberto}
\date{9 de Febrero del 2023}

\begin{document}

\maketitle

\section*{Problema}
Se necesita sacarle punta a un lápiz sin punta
\section*{Algoritmo}
\subsection*{Precondiciones}
\begin{itemize}
    \item Que se tenga un sacapuntas y un lápiz
    \item Que la superficie donde se va a trabajar sea plana y estable.
\end{itemize}
\subsection*{Postcondiciones}
\begin{itemize}
    \item La punta del lápiz está afilada.
    \item El sacapuntas está cerrado en caso de ser un sacapuntas de caja.
\end{itemize}

\subsection*{Pasos del algoritmo}
\begin{enumerate}
    \item   Preparación: Reúne un sacapuntas y un lápiz. Asegúrate de tener una superficie plana y estable donde puedas trabajar.
    \item   Abrir el sacapuntas: Abre el sacapuntas y coloca el lápiz en el interior, con la parte que se desea afilar hacia arriba y la otra parte hacia abajo.
    \item   Afilar el lápiz: Gira el sacapuntas hacia ti y haz un movimiento suave hacia adelante y hacia atrás para afilar la punta del lápiz. Repite este movimiento hasta que la punta esté afilada al tamaño deseado, usualmente dejar que la punta quede con un achatamiento menor a medio milímetro.
    \item   Verificación: Verifica la punta del lápiz para asegurarte de que esté afilada correctamente, con el achatamiento menor a 3 milímetros.
    \item   Cierre del sacapuntas: Cierra el sacapuntas cuando hayas terminado de afilar el lápiz, este paso solo es necesario si el sacapuntas es de tipo caja.

\begin{enumerate}
    \item Notas importantes:
    \item Sea $x$ una variable de tipo \textsc{BOOL} entonces:
\end{enumerate}


\end{enumerate}
\end{document}
