\section{}El economista británico A. W. Phillips presentó en 1958 un estudio
empírico según el cual existiría entre la tasa de crecimiento del salario monetario y la tasa de
desempleo, una relación inversa. A partir de estos estudios, más adelante, se construyó una
función según la cual también existe una relación inversa entre la tasa de inflación y la de
desempleo. El modelo de Phillips permite, averiguar la trayectoria temporal de la inlación,
es decir, cómo evoluciona la tasa de inflación a lo largo del tiempo, a través de la siguiente
ecuación diferencial
$$\frac{d^{2}\pi (t)}{dt^{2}} + a_{1}\frac{d\pi (t)}{dt} + a_{0}\pi (t) = bg(t)$$
donde $\pi=\pi(t)$ es la tasa de inflación esperada, $a_{1},a_{0},b$ son constantes y $g(t)$ es la tasa de crecimiento monetaria, resuelva la ecuación considerando; $a_{1}=a_{0}=4,b=5$
\begin{itemize}
    \item g(t)= C constante.
\end{itemize}

Reescribimos la ecuación con los valores dados de forma que tenemos
\begin{equation}
    \pi'' + 4\pi' + 4\pi = 5C
\end{equation}
Observamos que tenemos una ecuación no homogénea, entonces buscamos la soluciones de la ecuación homogénea asociada de forma que:
\begin{equation*}
    \pi'' + 4\pi' + 4\pi = 0
\end{equation*}
Proponemos $\pi=e^{rt}$, entonces se tiene que
\begin{equation*}
\centering
\begin{split}
    (r^{2} + 4r + 4)e^{rt} &= 0 \\   
    r^{2} + 4r + 4 &= 0 \\
    (r + 2)^{2} &= 0 \\
    \Rightarrow r_{1}=r_{2} &=-2 \\
    \Rightarrow y_{1} = e^{-2t}
    y_{2}
\end{split}
\end{equation*}
Entonces la solución de la ecuación homogénea asociada es
\begin{equation}
     y_{h} = c_{1}e^{r_{1}t} +c_{2}te^{r_{1}t} = c_{1}e^{-2t} +c_{2}te^{-2t}
\label{eq:homogenea}
\end{equation}
Ahora buscamos la solución particular $Y(t)$, por lo que calculamos $W(y_{1},y_{2})$

$$ W = \left|
\begin{array}{cc}
  e^{-2t}   &  te^{-2t}\\
   -2e^{-2t} & (1-2t)e^{-2t}
\end{array}
\right| = e^{-4t}
$$
\smallskip
\begin{equation*}
\begin{split}
     Y(t)=&-y_{1}(t) \int \frac{y_{2}(t)g(t)}{W(y_{1},y_{2})(t)} dt + y_{2}(t) \int \frac{y_{1}(t)g(t)}{W(y_{1},y_{2})(t)} dt \\
    =& -e^{-2t} \int \frac{te^{-2t}(5C)}{e^{-4t}} dt + te^{-2t} \int \frac{e^{-2t}(5C)}{e^{-4t}} dt\\
    =& -5C e^{-2t} \int te^{2t}dt + 5C te^{-2t} \int e^{2t} dt\\
    =& -5C e^{-2t} (\frac{1}{4}e^{2t}(2t -1)) + 5C te^{-2t} (\frac{1}{2}e^{2t}) \\
    =& -\frac{5}{2}Ct -\frac{5}{4}C  + \frac{5}{2}Ct = \frac{5}{4}C
\end{split}
\end{equation*}
Por lo tanto la solución general es de la forma 
\begin{equation*}
    y(t)= y_{h} + Y(t) = c_{1}e^{-2t} + c_{2}te^{-2t} + \frac{5}{4}C
\end{equation*}
\begin{itemize}
    \item g(t)= m + nt  (m y n constantes) 
\end{itemize}

Reescribimos la ecuación con los valores dados de forma que tenemos
\begin{equation}
    \pi'' + 4\pi' + 4\pi = m + nt
\end{equation}
Observamos que tenemos la misma ecuación homogénea asociada, por lo tanto $y_{1}, y_{2}$ (\ref{eq:homogenea}) son los mismas así como $W(y_{1},y_{2})$, entonces calculamos $Y(t)$
\begin{equation*}
    \begin{split}
        Y(t)=&-y_{1}(t) \int \frac{y_{2}(t)g(t)}{W(y_{1},y_{2})(t)} dt + y_{2}(t) \int \frac{y_{1}(t)g(t)}{W(y_{1},y_{2})(t)} dt \\
    =& -e^{-2t} \int \frac{te^{-2t}(m + nt)}{e^{-4t}} dt + te^{-2t} \int \frac{e^{-2t}(m + nt)}{e^{-4t}} dt\\
    =& -e^{-2t} ( m\int te^{2t}dt + n\int t^{2} e^{2t}dt )+  te^{-2t} (m\int e^{2t} dt + n\int te^{2t}dt) \\
    =& -e^{-2t} (\frac{m}{4}e^{2t}(2t -1) + \frac{n}{4}e^{2t}(2t^{2}- 2t +1)) + te^{-2t} (\frac{m}{2}e^{2t} + \frac{n}{4}e^{2t}(2t -1) )  \\
    =& -\frac{m}{2}t + \frac{m}{4} - \frac{n}{2}t^{2} + \frac{n}{2}t - \frac{n}{4} + \frac{m}{2}t + \frac{n}{2}t^{2} - \frac{n}{4}t \\
    =& \frac{n}{4}t + \frac{m+ 2n}{4}
    \end{split}
\end{equation*}
Por lo tanto la solución general es de la forma 
\begin{equation*}
    y(t)= y_{h} + Y(t) = c_{1}e^{-2t} + c_{2}te^{-2t} + \frac{n}{4}t + \frac{m+ 2n}{4}
\end{equation*}