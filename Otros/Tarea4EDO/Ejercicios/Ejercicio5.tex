\textcolor{blue}{Ejercicio 5.} Sea

$$A= \begin{pmatrix}
\lambda & 1\\
0 & \lambda
\end{pmatrix}$$

\begin{itemize}
    \item Encuentre $A^2$, $A^3$ y $A^4$\\
    
    Usando la definición: 
    
    $$A^2= A \cdot A= \begin{pmatrix}
    \lambda & 1\\
    0 & \lambda
    \end{pmatrix}
    \begin{pmatrix}
    \lambda & 1\\
    0 & \lambda
    \end{pmatrix}=
    \begin{pmatrix}
    \lambda^2 & \lambda + \lambda \\
    0 & \lambda^2
    \end{pmatrix}= 
    \begin{pmatrix}
    \lambda^2 & 2 \lambda \\
    0 & \lambda^2
    \end{pmatrix}$$
    
    $$A^3= A^2 \cdot A= \begin{pmatrix}
    \lambda^2 & 2 \lambda \\
    0 & \lambda^2
    \end{pmatrix}
    \begin{pmatrix}
    \lambda & 1\\
    0 & \lambda
    \end{pmatrix}=
    \begin{pmatrix}
    \lambda^3 & \lambda^2 + 2 \lambda^2 \\
    0 & \lambda^3
    \end{pmatrix}= 
    \begin{pmatrix}
    \lambda^3 & 3 \lambda^2 \\
    0 & \lambda^3
    \end{pmatrix}$$
    
    $$A^4= A^3 \cdot A= \begin{pmatrix}
    \lambda^3 & 3 \lambda^2 \\
    0 & \lambda^3
    \end{pmatrix}
    \begin{pmatrix}
    \lambda & 1\\
    0 & \lambda
    \end{pmatrix}=
    \begin{pmatrix}
    \lambda^4 & \lambda^3 + 3 \lambda^3 \\
    0 & \lambda^4
    \end{pmatrix}= 
    \begin{pmatrix}
    \lambda^4 & 4 \lambda^3 \\
    0 & \lambda^4
    \end{pmatrix}$$
    
    \item Demuestre que: 
    
    $$A^n\begin{pmatrix}
    \lambda^n & n \lambda^{n-1}\\
    0 & \lambda^n
    \end{pmatrix}$$
    
    Demostraremos por inducción.\\
    
    Para el caso base podemos usar lo demostrado en el inciso anterior, supongamos que se cumple para $k \geq 1$, finalmente por definición y por la hipótesis inductiva tenemos que: 
    
    $$A^{k+1}= A^{k} \cdot A= \begin{pmatrix}
    \lambda^{k} & k \lambda^{k-1}\\
    0 & \lambda^{k}
    \end{pmatrix}
    \begin{pmatrix}
    \lambda & 1\\
    0 & \lambda
    \end{pmatrix}= 
    \begin{pmatrix}
    \lambda^{k+1} & \lambda^{k}+k \lambda^{k}\\
    0 & \lambda^{k}
    \end{pmatrix}=
    \begin{pmatrix}
    \lambda^{k+1} & (k+1) \lambda^{k}\\
    0 & \lambda^{k}
    \end{pmatrix}$$
    \item Determine $e^{At}$\\
    
    Por definición y el inciso anterior: 
    
    $$e^{At}= \displaystyle \sum_{k=0}^{\infty} A^k \frac{t^k}{k!}= \frac{t^0}{0!} \begin{pmatrix}
    1 & 0\\
    0 & 1
    \end{pmatrix}+ 
    \begin{pmatrix}
    \displaystyle \sum_{k=1}^{\infty} \frac{\lambda^k t^k}{k!} & \displaystyle \sum_{k=1}^{\infty} \frac{ k \lambda^{k-1} t^k}{k!}\\
    0 & \displaystyle \sum_{k=1}^{\infty} \frac{ \lambda^{k} t^{k}}{k!}
    \end{pmatrix}$$
    
    $$= \begin{pmatrix}
    \frac{(\lambda t)^0}{0!} + \displaystyle \sum_{k=1}^{\infty} \frac{(\lambda t)^k}{k!} & t \displaystyle \sum_{k=1}^{\infty} \frac{(\lambda t)^{k-1}}{(k-1)!}\\
    0 & \frac{(\lambda t)^0}{0!}+ \displaystyle \sum_{k=1}^{\infty} \frac{(\lambda t)^k}{k!}
    \end{pmatrix}$$ 
    
    $$= \begin{pmatrix}
    \displaystyle \sum_{k=0}^{\infty} \frac{(\lambda t)^k}{k!} & t \displaystyle \sum_{k=0}^{\infty} \frac{(\lambda t)^k}{k!}\\
    0 & \displaystyle \sum_{k=0}^{\infty} \frac{(\lambda t)^k}{k!}
    \end{pmatrix}$$
    
    $$e^{At}= \begin{pmatrix}
    e^{\lambda t} & t e^{\lambda t}\\
    0 & e^{\lambda t}
    \end{pmatrix}$$
    
\end{itemize}
