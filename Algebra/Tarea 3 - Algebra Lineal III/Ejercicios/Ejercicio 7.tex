\section{Calcular el polinomio mínimo de las siguientes matrices:}
\[a)F = \begin{pmatrix}
	5 & 1 \\
	3 & 7 \end{pmatrix}
b)H = \begin{pmatrix}
	1 & 2 & 3 \\
	0 & 2 & 3 \\
	0 & 0 & 3\end{pmatrix}
c)K = \begin{pmatrix}
	4 & -2 & 2 \\
	6 & -3 & 4 \\
	3 & -2 & 3\end{pmatrix}
d)E = \begin{pmatrix}
	3 & 2 & -1 \\
	3 & 8 & -3 \\
	3 & 6 & -1\end{pmatrix}	\]
\begin{enumerate}
\item[$a)$] Primero calculemos el polinomio característico.
\[ p_F=\det( \lambda I-F)= \left| \begin{array}{cc}	
\lambda-5 & -1\\
-3 & \lambda -7 \end{array} \right| \]
\[ = (\lambda -5)(\lambda -7)- (-1)(-3)=\lambda ^2-12\lambda+35-3=\lambda ^2 -12\lambda +32=(\lambda-4)(\lambda-8)\]
\[\therefore p_F =(\lambda-4)(\lambda-8) \]
Como toda matriz cuadrada es una raíz de su polinomio característico y tenemos que evaluando el polinomio característico en la propia matriz $F$ obtenemos la matriz nula, tenemos que se trata del polinomio mínimo, pero como el m\'inimo debe dividir al caracter\'istico y adem\'as deben tener las mismas ra\'ices, si definimos a
\[m_F(\lambda)=(\lambda-4)(\lambda-8)=\lambda ^2 -12\lambda +32\]

Probemos esto último.\\Como
\[F = \begin{pmatrix}
	5 & 1 \\
	3 & 7 \end{pmatrix} \]
Entonces
\[F^2=F\cdot F=\begin{pmatrix}
	5 & 1 \\
	3 & 7 \end{pmatrix} \begin{pmatrix}
	5 & 1 \\
	3 & 7 \end{pmatrix}=\begin{pmatrix}
	5(5)+1(3) & 5(1)+1(7) \\
	3(5)+7(3) & 3(1)+7(7) \end{pmatrix}=\begin{pmatrix}
	28 & 12 \\
	36 & 52 \end{pmatrix}\]
Así
\[ m_F(\lambda)= F^2-12F+32I = \begin{pmatrix}
	28 & 12 \\
	36 & 52 \end{pmatrix}-12\begin{pmatrix}
	5 & 1 \\
	3 & 7 \end{pmatrix}+\begin{pmatrix}
	32 & 0 \\
	0 & 32 \end{pmatrix}=\begin{pmatrix}
	0 & 0 \\
	0 & 0 \end{pmatrix}   \]
\[\therefore m(\lambda) =(\lambda-4)(\lambda-8)= \lambda ^2 -12\lambda +32 \]


\item[$b)$] Primero calculemos el polinomio característico, notando que por ser una matriz triangular superior, solo debemos multiplicar los elementos de su diagonal principal para encontrar el determinante.
\[ p_H=\det( \lambda I-H)= \left| \begin{array}{ccc}	
\lambda-1 & -2 & -3\\
0 & \lambda -2 & -3\\
0 & 0 & \lambda - 3 \end{array} \right| =(\lambda -1)(\lambda -2)(\lambda -3)=(\lambda-1)(\lambda^2-5\lambda+6)\]\[=\lambda^3-5\lambda^2+6\lambda-\lambda^2+5\lambda-6=\lambda ^3-6\lambda ^2+11\lambda -6\] \[\therefore p_H = (\lambda -1)(\lambda -2)(\lambda -3)=\lambda ^3-6\lambda ^2+11\lambda -6\]
Notemos que el polinomio $(\lambda -1)(\lambda -2)(\lambda -3)$ es el único polinomio que cumple las propiedades para ser el polinomio mínimo, pues es divisible entre si mismo y tiene las mismas ra\'ices que $p_H$. De esta manera, comprobemos que al evaluar el polinomio minimal $m_H(\lambda) = \lambda ^3-6\lambda ^2+11\lambda -6$ en la matriz $H$ obtenemos la matriz nula.\\Como
\[H = \begin{pmatrix}
	1 & 2 & 3 \\
	0 & 2 & 3 \\
	0 & 0 & 3\end{pmatrix}\]
Entonces:
\[H^2=H\cdot H=\begin{pmatrix}
	1 & 2 & 3 \\
	0 & 2 & 3 \\
	0 & 0 & 3\end{pmatrix}\begin{pmatrix}
	1 & 2 & 3 \\
	0 & 2 & 3 \\
	0 & 0 & 3\end{pmatrix}=\begin{pmatrix}
	1(1)+2(0)+3(0) & 1(2)+2(2)+3(0) & 1(3)+2(3)+3(3) \\
	0(1)+2(0)+3(0) & 0(2)+2(2)+3(0) & 0(3)+2(3)+3(3) \\
	0(1)+0(0)+3(0) & 0(2)+0(2)+3(0) & 0(3)+0(3)+3(3) 
	\end{pmatrix}=\begin{pmatrix}
	1 & 6 & 18 \\
	0 & 4 & 15 \\
	0 & 0 & 9\end{pmatrix}\]
Además:
\[H^3=H\cdot H^2=\begin{pmatrix}
	1 & 2 & 3 \\
	0 & 2 & 3 \\
	0 & 0 & 3\end{pmatrix}\begin{pmatrix}
	1 & 6 & 18 \\
	0 & 4 & 15 \\
	0 & 0 & 9\end{pmatrix}=\begin{pmatrix}
	1(1)+2(0)+3(0) & 1(6)+2(4)+3(0) & 1(18)+2(15)+3(9) \\
	0(1)+2(0)+3(0) & 0(6)+2(4)+3(0) & 0(18)+2(15)+3(9) \\
	0(1)+0(0)+3(0) & 0(6)+0(4)+3(0) & 0(18)+0(15)+3(9) 
	\end{pmatrix}\]\[=\begin{pmatrix}
	1 & 14 & 75 \\
	0 & 8 & 57 \\
	0 & 0 & 27\end{pmatrix}\]
Así evaluando:
\[m_H(\lambda)=H^3-6H^2+11H-6I=\begin{pmatrix}
	1 & 14 & 75 \\
	0 & 8 & 57 \\
	0 & 0 & 27\end{pmatrix}-6\begin{pmatrix}
	1 & 6 & 18 \\
	0 & 4 & 15 \\
	0 & 0 & 9\end{pmatrix}+11\begin{pmatrix}
	1 & 2 & 3 \\
	0 & 2 & 3 \\
	0 & 0 & 3\end{pmatrix}-6\begin{pmatrix}
	1 & 0 & 0 \\
	0 & 1 & 0 \\
	0 & 0 & 1\end{pmatrix}\]
\[=\begin{pmatrix}
	1 & 14 & 75 \\
	0 & 8 & 57 \\
	0 & 0 & 27\end{pmatrix}-\begin{pmatrix}
	6 & 36 & 108 \\
	0 & 24 & 90 \\
	0 & 0 & 54\end{pmatrix}+\begin{pmatrix}
	11 & 22 & 33 \\
	0 & 22 & 33 \\
	0 & 0 & 33\end{pmatrix}-\begin{pmatrix}
	6 & 0 & 0 \\
	0 & 6 & 0 \\
	0 & 0 & 6\end{pmatrix}=\begin{pmatrix}
	0 & 0 & 0 \\
	0 & 0 & 0 \\
	0 & 0 & 0\end{pmatrix}\]\[\therefore m_H(\lambda) =(\lambda -1)(\lambda -2)(\lambda -3)=\lambda ^3-6\lambda ^2+11\lambda -6\]
	
	
	
\item[$c)$] Primero calculemos el polinomio característico por cofactores.
\[ p_K=\det( \lambda I-K)= \left| \begin{array}{ccc}	
\lambda-4 & 2 & -2\\
-6 & \lambda +3 & -4\\
-3 & 2 & \lambda - 3 \end{array} \right| \]
\[ = (\lambda -4)\left| \begin{array}{cc}	
\lambda+3 & -4\\
2 & \lambda -3
\end{array} \right| - (2)\left| \begin{array}{cc}	
-6 & -4\\
-3 & \lambda -3
\end{array} \right| + (-2)\left| \begin{array}{cc}	
-6 & \lambda +3\\
-3 & 2
\end{array} \right|\] \[=(\lambda -4)[(\lambda +3)(\lambda -3)-(-4)(2)] -2[(-6)(\lambda -3)-(-4)(-3)] -3[(-6)(2)-(\lambda +3)(-3)]\] \[=(\lambda -4)[\lambda ^2-1]-2[-6\lambda +6]-2[3\lambda -3]\] \[=\lambda ^3-4\lambda ^2-\lambda +4+12\lambda -12-6\lambda +6\] \[=\lambda ^3 -4\lambda ^2+5\lambda -2\] \[=\lambda ^3-\lambda ^2-3\lambda ^2+3\lambda +2\lambda -2\] \[=\lambda ^2(\lambda -1)-3\lambda(\lambda -1)+2(\lambda -1)\]\[=(\lambda -1)[\lambda (\lambda -1)-2(\lambda -1)]\] \[=(\lambda -1)^2(\lambda -2)\]\[\therefore p_K = (\lambda -1)^2(\lambda -2)\]
Sabemos que el polinomio mínimo $m_K(\lambda)$ debe dividir a $p_K$. También cada factor irreducible de $p_K$ debe ser un factor de $m(\lambda)$. Por ello, de nuestros posibles candidatos $(\lambda -1)^2(\lambda -2)$ y $(\lambda -1)(\lambda -2)$ proponemos a $m_K(\lambda)=(\lambda -1)(\lambda -2)=\lambda ^2-3\lambda +2$. Ahora probemos que al evaluar la matriz en el polinomio resulta la matriz nula.
Como
\[K = \begin{pmatrix}
	4 & -2 & 2 \\
	6 & -3 & 4 \\
	3 & -2 & 3\end{pmatrix}\]
Entonces
\[K^2=K\cdot K=\begin{pmatrix}
	4 & -2 & 2 \\
	6 & -3 & 4 \\
	3 & -2 & 3\end{pmatrix}\begin{pmatrix}
	4 & -2 & 2 \\
	6 & -3 & 4 \\
	3 & -2 & 3\end{pmatrix}= \begin{pmatrix}
	4(4)+(-2)(6)+2(3) & 4(-2)+(-2)(-3)+2(-2) & 4(2)+(-2)(4)+2(3) \\
	6(4)+(-3)(6)+4(3) & 6(-2)+(-3)(-3)+4(-2) & 6(2)+(-3)(4)+4(3) \\
	3(4)+(-2)(6)+3(3) & 3(-2)+(-2)(-3)+3(-2) & 3(2)+(-2)(4)+3(3) \end{pmatrix}\]\[=\begin{pmatrix}
	10 & -6 & 6 \\
	18 & -11 & 12 \\
	9 & -6 & 7\end{pmatrix}\]
Así
\[m(\lambda)=K^2-3K+2I=\begin{pmatrix}
	10 & -6 & 6 \\
	18 & -11 & 12 \\
	9 & -6 & 7\end{pmatrix}-3\begin{pmatrix}
	4 & -2 & 2 \\
	6 & -3 & 4 \\
	3 & -2 & 3\end{pmatrix}+2\begin{pmatrix}
	1 & 0& 0 \\
	0 & 1 & 0 \\
	0 & 0 & 1\end{pmatrix}\]
\[= \begin{pmatrix}
	10 & -6 & 6 \\
	18 & -11 & 12 \\
	9 & -6 & 7\end{pmatrix}+\begin{pmatrix}
	-12 & 6 & -6 \\
	-18 & 9 & -12 \\
	-9 & 6 & -9\end{pmatrix}+\begin{pmatrix}
	2 & 0& 0 \\
	0 & 2 & 0 \\
	0 & 0 & 2\end{pmatrix}=\begin{pmatrix}
	0 & 0& 0 \\
	0 & 0 & 0 \\
	0 & 0 & 0\end{pmatrix}\]\[\therefore m_K(\lambda) = (\lambda -1)(\lambda -2)=\lambda^2-3\lambda+2\]
	
	
	
\item[$d)$] Primero calculemos el polinomio característico por cofactores:
\[ p_E=\det( \lambda I-E)= \left| \begin{array}{ccc}	
\lambda-3 & -2 & 1\\
-3 & \lambda -8 & 3\\
-3 &-6 & \lambda +1 \end{array} \right| \]
\[ = (\lambda -3)\left| \begin{array}{cc}	
\lambda-8 & 3\\
-6 & \lambda +1 \end{array} \right| - (-2)\left| \begin{array}{cc}	
-3 & 3\\
-3 & \lambda +1 \end{array} \right| + (1)\left| \begin{array}{cc}	
-3 & \lambda -8\\
-3 & -6 \end{array} \right|\]
\[=(\lambda -3)[(\lambda -8)(\lambda +1)-(3)(-6)]+2[(-3)(\lambda +1)-(3)(-3)]+1[(-3)(-6)-(\lambda -8)(-3)]\] \[= (\lambda -3)[\lambda ^2-7\lambda +10]+2[-3\lambda +6]+1[3\lambda -6]\] \[=(\lambda -3)(\lambda -2)(\lambda -5)-6\lambda +12+3\lambda -6\]\[=\lambda ^3 -10\lambda ^2 +28\lambda -24\]\[=\lambda ^3-2\lambda ^2 -8\lambda ^2+16\lambda + 12\lambda -24\]\[=\lambda ^2(\lambda -2)-8\lambda (\lambda -2)+12(\lambda -2)\] \[=(\lambda -2)(\lambda ^2-2\lambda -6\lambda +12)\] \[(\lambda -2)(\lambda -2)(\lambda -6)=(\lambda -2)^2(\lambda -6)\]\[\therefore P_E=(\lambda -2)^2(\lambda -6)=\lambda ^3 -10\lambda ^2 +28\lambda -24\]
Nuevamente sabemos que el polinomio mínimo $m_E(\lambda)$ debe dividir a $p_E$. Y también cada factor irreducible de $p_E$ debe ser un factor de $m_E(\lambda)$. Por ello, de nuestros posibles candidatos $(\lambda -2)^2(\lambda -6)$ y $(\lambda -2)(\lambda -6)$ proponemos a $m_E(\lambda)=(\lambda -2)(\lambda -6)=\lambda ^2-8\lambda+12$. Ahora probemos que al evaluar la matriz en el polinomio resulta la matriz nula.
Como
\[E = \begin{pmatrix}
	3 & 2 & -1 \\
	3 & 8 & -3 \\
	3 & 6 & -1\end{pmatrix}\]
Entonces
\[E^2=E\cdot E= \begin{pmatrix}
	3 & 2 & -1 \\
	3 & 8 & -3 \\
	3 & 6 & -1\end{pmatrix}\begin{pmatrix}
	3 & 2 & -1 \\
	3 & 8 & -3 \\
	3 & 6 & -1\end{pmatrix}=\begin{pmatrix}
	3(3)+2(3)+(-1)(3) & 3(2)+2(8)+(-1)(6) & 3(-1)+2(-3)+(-1)(-1) \\
	3(3)+8(3)+(-3)(3) & 3(2)+8(8)+(-3)(6) & 3(-1)+8(-3)+(-3)(-1) \\
	3(3)+6(3)+(-1)(3) & 3(2)+6(8)+(-1)(6) & 3(-1)+6(-3)+(-1)(-1) \end{pmatrix}\]\[=\begin{pmatrix}
	12 & 16 & -8 \\
	24 & 52 & -24 \\
	24 & 48 & -20\end{pmatrix}\]
Así
\[m(\lambda )= E^2-8E+12I=\begin{pmatrix}
	12 & 16 & -8 \\
	24 & 52 & -24 \\
	24 & 48 & -20\end{pmatrix}-8\begin{pmatrix}
	3 & 2 & -1 \\
	3 & 8 & -3 \\
	3 & 6 & -1\end{pmatrix}+12\begin{pmatrix}
	1 & 0 & 0 \\
	0 & 1 & 0 \\
	0 & 0 & 1\end{pmatrix}\]
\[= \begin{pmatrix}
	12 & 16 & -8 \\
	24 & 52 & -24 \\
	24 & 48 & -20\end{pmatrix}+\begin{pmatrix}
	-24 & -16 & 8 \\
	-24 & -64 & 24 \\
	-24 & -48 & -8\end{pmatrix}+\begin{pmatrix}
	12 & 0 & 0 \\
	0 & 12 & 0 \\
	0 & 0 & 12\end{pmatrix}=\begin{pmatrix}
	0 & 0 & 0 \\
	0 & 0 & 0 \\
	0 & 0 & 0\end{pmatrix}\]\[\therefore m_E(\lambda) = (\lambda -2)(\lambda -6)=\lambda^2-8\lambda+12\]


\end{enumerate}
