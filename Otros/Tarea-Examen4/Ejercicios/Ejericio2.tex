\textcolor{blue}{Ejercicio 2.}
Para cada uno de los siguiente sistemas:
\begin{itemize}
    \item Encuentra los eigenvalores y los eigenvectores.
    \item Halle la soluci\'on al sistema  y clasifique el punto critico $(0,0)$ seg\'un el tipo y determine si es estable o instable.
    %\item Dibuje los planos fase correspondientes a la soluci\'on  de cada ecuaci\'on (puede usar un software si as\'i lo desea).

\end{itemize}

\begin{itemize}
    \item $ x'=\begin{pmatrix}3&-2\\ 2&-2\end{pmatrix}x $
            $$\det \begin{pmatrix}3-\lambda&-2\\ \:2&-2-\lambda\end{pmatrix}=\lambda^2-\lambda-2=\left(\lambda -2\right)\left(\lambda +1\right)=0$$
            as\'i tenemos los valores propios:
            $$\lambda_1=2\:,\: \lambda_2=-2$$
            y los vectores propios como:
            $$X_1=\begin{pmatrix}2\\ 1\end{pmatrix}e^{2t}\:,\: X_2=\begin{pmatrix}1\\ 2\end{pmatrix}e^{-2t}$$
            La soluci\'on queda como:
            
            
            
            $(0,0)$ es un punto critico inestable en el sistema.
    \item $x'=\begin{pmatrix}3&-5\\ \:1&-1\end{pmatrix}x$
        $$\det \begin{pmatrix}3-\lambda&-5\\ \:1&-1-\lambda\end{pmatrix}=\lambda^2-2\lambda+1=0$$
        de donde tenemos:
        $$\lambda_1=1+i\:,\: \lambda_2=1-i$$
        con los vectores propios:
        
        $$X_1=\begin{pmatrix}2+i\\ 1\end{pmatrix}\:,\:X_2=\begin{pmatrix}2-i\\ 1\end{pmatrix}$$
        
        y la soluci\'on queda como:
        
        $$X=c_1\left[ \begin{pmatrix}2-i\\ 1\end{pmatrix}\cos(t)-\begin{pmatrix}2+i\\ 1\end{pmatrix}\sin(t) \right]e^{t}+
        c_2\left[ \begin{pmatrix}2+i\\ 1\end{pmatrix}\sin(t)+\begin{pmatrix}2-i\\ 1\end{pmatrix}\cos(t) \right]e^{t}$$
        
        As\'i tenemos que como la parte real es mayor que $0$ tenemos que el punto $(0,0)$ es un punto critico inestable
        
        
    \item $x'=\begin{pmatrix}1&-4\\ \:\:4&-7\end{pmatrix}x$
     $$\det \begin{pmatrix}1-\lambda&-4\\ \:\:4&-7-\lambda\end{pmatrix}=\lambda^2+6\lambda+9=0$$
     Donde:
     $$\lambda_{1,2}=-3 \mbox{ Con multiplicidad de 2}$$
     y los vectores propios quedan como:
     $$X_{1,2}=\begin{pmatrix}1\\ 1\end{pmatrix}$$
     
     Y la soluci\'on queda como:
     $$X=c_1\begin{pmatrix}1\\ 1\end{pmatrix}e^{-3t}+c_2\begin{pmatrix}1\\ 1\end{pmatrix}e^{-3t}$$
     
     adem\'as dado que $\lambda_1,\lambda_2<0$ y la matriz no es diagonalizable tenemos que $(0,0)$ es un punto critico inestable
     
     
\end{itemize}