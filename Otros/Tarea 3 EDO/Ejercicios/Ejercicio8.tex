\textcolor{blue}{Ejercicio 8.}La función Gamma se denota por $\Gamma (p)$ y se define por la siguiente integral

\begin{equation*}
    \Gamma (p+1)= \int_{0}^{\infty} e^{-x}x^{p}dx 
\end{equation*}

La integral converge cuando $x \rightarrow \infty$ para todo $p$. Si $p < 0$, el integrando no es acotado, cuando $x \rightarrow 0^{+}$.\\

Sin embargo se puede demostrar que la integral converge en $x=0$ para todo $p > -1$\\

$\mathbf{1.}$ Pruebe que para todo $p > 0$, $\Gamma (p+1)= p \Gamma (p)$\\

$\mathbf{Dem:}$ Por definición: 

\begin{equation*}
    \Gamma (p+1)= \int_{0}^{\infty} e^{-x} x^{p} dx
\end{equation*}

Si $u= x^p \Rightarrow du= px^{p-1}$ y $dv=e^{-x} \Rightarrow v= -e^{-x}$, integrando por partes tenemos:

\begin{equation*}
    \Gamma(p+1)= -e^{-x}x^p \Big|_{0}^{\infty} +p \int_{0}^{\infty} e^{-x}x^{p-1} dx 
\end{equation*}

Ahora el término $-x^{p}e^{-x}$ tiende a cero cuando $x \rightarrow \infty$.Además, como $p > 0$, entonces $-x^{p}x^{-x}$ es cero valuado en cero, entonces: 

\begin{equation*}
    \Gamma (p+1)= p \int_{0}^{\infty} e^{-x}x^{p-1}dx= p \Gamma (p)
\end{equation*}

$\mathbf{2.}$ Muestre que $\Gamma (1)=1$\\

$\mathbf{Dem:}$ Sea $f(p)= \Gamma (p+1)$. Entonces $f(0)$ es: 

\begin{equation*}
    \Gamma(1)= \int_{0}^{\infty} e^{-x}x^0 dx= \int_{0}^{\infty} e^{-x}dx= -e^{-x} \Big |_{0}^{\infty}=  -\lim_{b\to\infty} \frac{1}{e^b}+ \frac{1}{e^0}= 0+1= 1
\end{equation*}

$\mathbf{3.}$Pruebe que si $p$ es un número positivo $n$ entonces $\Gamma(n+1)= n!$\\

$\mathbf{Dem:}$ Procedemos por inducción sobre $n$\\

\textit{Base de inducción:} Para $n=1$ tenemos:

\begin{equation*}
    \Gamma(n+1)= \Gamma (2)= 1 \Gamma (1)= 1= 1!
\end{equation*}

Para $n=2$: 

\begin{equation*}
    \Gamma (n+1)= \Gamma (3)= 2 \Gamma (2)= 2 \cdot 1= 2!
\end{equation*}

Finalmente, para $n=3$:

\begin{equation*}
    \Gamma (n+1)= \Gamma (4)= 3 \Gamma (3)= 3 \cdot 2 \cdot 1= 3!
\end{equation*}

\textit{Hipótesis de inducción:} Supongamos que se cumple para $n=k$, es decir $\Gamma (k+1)= k!$\\

\textit{Paso inductivo:} Queremos demostrarlo para $n=k+1$. Por el primer inciso tenemos:

\begin{equation*}
    \Gamma (n+1)= \Gamma (k+2)= (k+1) \Gamma (k+1)
\end{equation*}

Por la hipótesis de inducción:

\begin{equation*}
    \Gamma (k+2)= (k+1)k!= (k+1)!
\end{equation*}

Y así queda demostrado por inducción que: $\Gamma (k+1)= k!$\\

$\mathbf{4.}$ Demuestre que si $p > 0$ entonces: 

\begin{equation*}
    p(p+1)(p+2)...(p+n-1)= \frac{\Gamma (p+n)}{\Gamma (p)}
\end{equation*}

$\mathbf{Dem:}$ Lo haremos por inducción sobre $n$\\

\textit{Base de inducción:} Para $n=1$, por el inciso 1, $\Gamma(p+1)= p \Gamma (p)$, entonces: 

\begin{equation*}
    p= p+1-1= \frac{\Gamma (p+1)}{\Gamma (p)}
\end{equation*}

Para $n=2$ (por el inciso 1) se tiene que $\Gamma (p+2)= (p+1) \Gamma (p+1)= (p+1) p \Gamma (p)$ de donde: 

\begin{equation*}
    p (p+1)= p (p+2-1)=  \frac{\Gamma(p+2)}{\Gamma (p)}
\end{equation*}

\textit{Hipótesis de inducción:} Supongamos que se cumple para $n=k$, es decir: 

\begin{equation*}
    p(p+1)(p+2)...(p+k-1)= \frac{\Gamma (p+k)}{\Gamma(p)}
\end{equation*}

\textit{Paso inductivo} Buscamos demostrar para $n= k+1$. Por el inciso 1 se tiene que $\Gamma (p+k+1)= (p+k) \Gamma (p+k)$, así mismo por la hipótesis de inducción se tiene que: 

\begin{equation*}
    \Gamma(p+k+1)= (p+k)p(p+1)(p+2)...(p+k-1) \Gamma (p)
\end{equation*}

De ahí podemos reescribirlo como: 

\begin{equation*}
    \frac{\Gamma(p+k+1)}{\Gamma (p)}= p (p+1)(p+2)...(p+k-1)(p+(k+1)-1)
\end{equation*}

Y así, por inducción se ha demostrado que $p(p+1)(p+2)...(p+n-1)= \frac{\Gamma(p+n)}{\Gamma (p)}$\\

Tomando $\Gamma (\frac{1}{2})= \sqrt{\pi}$, por lo demostrado anteriormente tenemos que: 

\begin{equation*}
    \Gamma \Big( \frac{3}{2} \Big)= \Gamma \Big( \frac{1}{2}+1 \Big)= \frac{1}{2} \cdot \Gamma \Big( \frac{1}{2} \Big)= \frac{\sqrt{\pi}}{2}
\end{equation*}

De manera similar se tiene que:

\begin{equation*}
    \Gamma \Big(\frac{11}{2} \Big)= \Gamma \Big( \frac{1}{2}+5 \Big)= \frac{1}{2} \cdot \frac{3}{2} \cdot \frac{5}{2} \cdot \frac{7}{2} \cdot \frac{9}{2} \cdot \Gamma \Big( \frac{1}{2} \Big)= \frac{945 \sqrt{\pi}}{32}
\end{equation*}

