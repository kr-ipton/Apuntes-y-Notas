\textcolor{blue}{Ejercicio 3.} Encuentre la solución general de:
\begin{itemize}
    \item $\dot x = 
\begin{pmatrix}
-3 & \sqrt{2} \\
\sqrt{2} & -2
\end{pmatrix}x$
\end{itemize}

Determinamos los valores propios de A, entonces
\begin{equation*}
    |A - \lambda I| = 
|\begin{pmatrix}
-3 - \lambda & \sqrt{2} \\
\sqrt{2} & -2 - \lambda
\end{pmatrix}| 
= (-3 - \lambda)(-2 - \lambda)-(\sqrt{2} )(\sqrt{2} )= \lambda^{2}+5\lambda+4
\end{equation*}
\begin{equation*}
    = (\lambda +4)(\lambda +1 )= 0
\end{equation*}
Entonces $\lambda_{1}=-4$ y $\lambda_{2}=-1$ \\

Caso con $\lambda_{1}=-4$ 
\begin{equation*}
    A - (-4) I = A +4 I =
\begin{pmatrix}
-3 + 4 & \sqrt{2} \\
\sqrt{2} & -2+4
\end{pmatrix}
=
\begin{pmatrix}
1 & \sqrt{2} \\
\sqrt{2} & 2
\end{pmatrix}
\end{equation*}

Buscamos $ker(A +4 I)$
\begin{equation*}
    \begin{pmatrix}
1 & \sqrt{2} \\
\sqrt{2} & 2
\end{pmatrix}
\begin{pmatrix}
v_{1}   \\
v_{2}  
\end{pmatrix} =
\begin{pmatrix}
0   \\
0  
\end{pmatrix}
\end{equation*}
\begin{equation*}
    \Rightarrow v_{1}+ \sqrt{2}v_{2}=0 \atop
    \sqrt{2}v_{1}+2v_{2} =0
\end{equation*}
Por lo tanto $u_{1}=\begin{pmatrix}
-\sqrt{2} \\
1
\end{pmatrix}$ \\
Entonces $x_{1}=e^{-4t}\begin{pmatrix}
-\sqrt{2} \\
1
\end{pmatrix}
=
\begin{pmatrix}
-\sqrt{2}e^{-4t} \\
e^{-4t}
\end{pmatrix}$ 

Caso con $\lambda_{1}=-1$ 
\begin{equation*}
    A - (-1) I = A + I =
\begin{pmatrix}
-3 +1 & \sqrt{2} \\
\sqrt{2} & -2+1
\end{pmatrix}
=
\begin{pmatrix}
-2 & \sqrt{2} \\
\sqrt{2} & -1
\end{pmatrix}
\end{equation*}

Buscamos $ker(A + I)$
\begin{equation*}
    \begin{pmatrix}
-2 & \sqrt{2} \\
\sqrt{2} & -1
\end{pmatrix}
\begin{pmatrix}
v_{1}   \\
v_{2}  
\end{pmatrix} =
\begin{pmatrix}
0   \\
0  
\end{pmatrix}
\end{equation*}

\begin{equation*}
    \Rightarrow -2v_{1}+ \sqrt{2}v_{2}=0 \atop
    \sqrt{2}v_{1}-1v_{2} =0
\end{equation*}
Podemos tomar que $u_{2}=\begin{pmatrix}
\frac{\sqrt{2}}{2} \\
1
\end{pmatrix}$ \\
Entonces $x_{2}=e^{-t}\begin{pmatrix}
\frac{\sqrt{2}}{2} \\
1
\end{pmatrix}$ 
\begin{equation*}
    x_{2}=\begin{pmatrix}
\frac{\sqrt{2}}{2}e^{-t} \\
e^{-t}
\end{pmatrix}
\end{equation*}

Por lo que la matriz fundamental es 

\begin{equation*}
    \Phi(t)= \begin{pmatrix}
-\sqrt{2}e^{-4t}& \frac{\sqrt{2}}{2}e^{-t} \\
e^{-4t} & e^{-t}
\end{pmatrix}
\end{equation*}
Y la solución es 
\begin{equation*}
    X=c_{1}x_{1} + c_{2}x_{2} =c_{1}\begin{pmatrix}
-\sqrt{2}e^{-4t} \\
e^{-4t}
\end{pmatrix} + c_{2} \begin{pmatrix}
    \frac{\sqrt{2}}{2}e^{-t}  \\
    e^{-t}
    \end{pmatrix}
\end{equation*}
\newpage
\begin{itemize}
    \item $\dot x = 
\begin{pmatrix}
3 & -2 \\
4 & -1
\end{pmatrix}x$
\end{itemize}
Determinamos los valores propios de A, entonces
\begin{equation*}
    |A - \lambda I| = 
|\begin{pmatrix}
3 - \lambda & -2\\
4 & -1 - \lambda
\end{pmatrix}| 
= (3 - \lambda)(-1 - \lambda)-(-2)(4)= \lambda^{2}-2\lambda+5 = 0
\end{equation*}
Usando la fórmula general obtenemos que:

\begin{equation*}
\begin{split}
    &\lambda= \frac{2-\sqrt{-16}}{2} \\ \Rightarrow &\lambda_{1}= 1-2i \\
    &\lambda_{2}=1+2i
\end{split}
\end{equation*}
Caso con $\lambda_{1}=1-2i$ 
\begin{equation*}
    A - (1-2i) I = 
\begin{pmatrix}
3 - (1-2i) & -2 \\
4 & -1 - (1-2i)
\end{pmatrix}
=
\begin{pmatrix}
2+2i& -2\\
4 & -2 +2i
\end{pmatrix}
\end{equation*}
Buscamos $ker(A - (1-2i) I)$
\begin{equation*}
\begin{pmatrix}
2+2i & -2\\
4 & -2 +2i
\end{pmatrix}
\begin{pmatrix}
v_{1}   \\
v_{2}  
\end{pmatrix} =
\begin{pmatrix}
0   \\
0  
\end{pmatrix}
\end{equation*}
\begin{equation*}
    \Rightarrow (2+2i)v_{1} -2 v_{2}=0 \atop
   4 v_{1}+ (-2 +2i) v_{2} =0
\end{equation*}
Podemos tomar que $u_{2}=\begin{pmatrix}
\frac{1-i}{2} \\
1
\end{pmatrix}$ \\
Entonces $x_{1}=e^{1-2i}\begin{pmatrix}
\frac{1-i}{2} \\
1
\end{pmatrix}
=
e^{t}
\begin{pmatrix}
\frac{1-i}{2} \\
1
\end{pmatrix}
(\cos{(2t)}-i\sin{(2t)})
$ 
\begin{equation*}
    x_{1}= 
\begin{pmatrix}
e^{t}\frac{1-i}{2}\cos{(2t)}- e^{t}\frac{1-i}{2}i\sin{(2t)}\\
e^{t}\cos{(2t)}-e^{t}i\sin{(2t)}
\end{pmatrix}
=
\begin{pmatrix}
\frac{1}{2}e^{t}(\cos{(2t)}+ \sin{(2t)}) \\
e^{t}\cos{(2t)}
\end{pmatrix}
-i
\begin{pmatrix}
e^{t}(\frac{\sin{(2t)}}{2} + \cos{(2t)}) \\
e^{t}\sin{(2t)}
\end{pmatrix}
\end{equation*}
Por lo que encontramos que: 
\begin{equation*}
    \begin{split}
        x_{1}&=e^{t}\begin{pmatrix}
\frac{1}{2}(\cos{(2t)}+ \sin{(2t)}) \\
\cos{(2t)}
\end{pmatrix} \\
x_{2}&= e^{t} \begin{pmatrix}
-(\frac{\sin{(2t)}}{2} + \cos{(2t)}) \\
-\sin{(2t)}
\end{pmatrix}
    \end{split}
\end{equation*}

Por lo que la matriz fundamental es 

\begin{equation*}
    \Phi(t)= \begin{pmatrix}
e^{t}\frac{1}{2}(\cos{(2t)}+ \sin{(2t)}) & -e^{t}(\frac{\sin{(2t)}}{2} + \cos{(2t)}) \\
e^{t}\cos{(2t)} & -e^{t}\sin{(2t)}
\end{pmatrix}
\end{equation*}
Y la solución es 
\begin{equation*}
    X=c_{1}x_{1} + c_{2}x_{2} = c_{1}\begin{pmatrix}
    e^{t}\frac{1}{2}(\cos{(2t)}+ \sin{(2t)}) \\
    e^{t}\cos{(2t)}
    \end{pmatrix} + c_{2} \begin{pmatrix}
    -e^{t}(\frac{\sin{(2t)}}{2} + \cos{(2t)}) \\
    -e^{t}\sin{(2t)}
    \end{pmatrix}
\end{equation*}
\newpage
\begin{itemize}
    \item $\dot x=\begin{pmatrix}
    3 & -4 \\
    1 & -1
    \end{pmatrix}x$
\end{itemize}
Determinamos los valores propios de A, entonces
\begin{equation*}
    |A - \lambda I| = 
|\begin{pmatrix}
3 - \lambda & -4\\
1 & -1 - \lambda
\end{pmatrix}| 
= (3 - \lambda)(-1 - \lambda)-(1)(-4)= \lambda^{2}-2\lambda+1 = 0
\end{equation*}
\begin{equation*}
    = (\lambda - 1 )^{2}= 0
\end{equation*}
Entonces $\lambda_{1}=\lambda_{2}=1$ \\

Tomamos $\lambda=1$
\begin{equation*}
    A - (1) I = A - I =
\begin{pmatrix}
3 - 1 & -4\\
1 & -1 - 1
\end{pmatrix}
=
\begin{pmatrix}
2 & -4\\
1 & -2
\end{pmatrix}
\end{equation*}
Buscamos $ker(A -I)$
\begin{equation*}
\begin{pmatrix}
2 & -4\\
1 & -2
\end{pmatrix}
\begin{pmatrix}
v_{1}   \\
v_{2}  
\end{pmatrix} =
\begin{pmatrix}
0   \\
0  
\end{pmatrix}
\end{equation*}
\begin{equation*}
    \Rightarrow 2v_{1}-4v_{2}=0 \atop
    v_{1}-2v_{2} =0
\end{equation*}
Podemos tomar que $u_{1}=\begin{pmatrix}
2 \\
1
\end{pmatrix}$ \\

Entonces $x_{1}=e^{t}\begin{pmatrix}
2\\
1
\end{pmatrix}
=
\begin{pmatrix}
2e^{t} \\
e^{t}
\end{pmatrix}$ \\
Debido a que $\lambda_{1}=\lambda_{2}$, buscamos el vector propio generalizado.
\begin{equation*}
\begin{split}
    (A-I)u_{2}=u_{1} \\
\begin{pmatrix}
2 & -4\\
1 & -2
\end{pmatrix}
\begin{pmatrix}
v_{1}   \\
v_{2}  
\end{pmatrix} =
\begin{pmatrix}
2   \\
1  
\end{pmatrix}
\end{split}
\end{equation*}
\begin{equation*}
    \Rightarrow 2v_{1}+ \sqrt{2}v_{2}=2 \atop
    \sqrt{2}v_{1}+3v_{2} =1
\end{equation*}
Podemos tomar que $u_{2}=\begin{pmatrix}
3 \\
1
\end{pmatrix}$ \\

Entonces
\begin{equation*}
    x_{2}= e^{t}\left[t 
    \begin{pmatrix}
    2\\
    1
    \end{pmatrix} + 
    \begin{pmatrix}
    3 \\
    1
    \end{pmatrix} \right]
    =
    e^{t}\begin{pmatrix}
    2t +3 \\
    t + 1
    \end{pmatrix}
\end{equation*}
Por lo que la matriz fundamental es 

\begin{equation*}
    \Phi(t)= \begin{pmatrix}
2e^{t} &  2te^{t}+ 3e^{t}\\
e^{t} & te^{t} + e^{t}
\end{pmatrix}
\end{equation*}
Y la solución es 
\begin{equation*}
    X=c_{1}x_{1} + c_{2}x_{2} = c_{1}\begin{pmatrix}
2e^{t} \\
e^{t}
\end{pmatrix} + c_{2}\begin{pmatrix}
2te^{t}+ 3e^{t}\\
te^{t} + e^{t}
\end{pmatrix}
\end{equation*}