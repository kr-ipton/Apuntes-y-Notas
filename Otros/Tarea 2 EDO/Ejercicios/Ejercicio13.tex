\section{Determine la solución general de:}

\begin{equation*}
    y^{\prime \prime}+\lambda^{2} y=\sum_{m=1}^{N} a_{m} \sin (m \pi x)
\end{equation*}

donde $\lambda>0$ y $\lambda \neq m \pi$ para $m=1,2, \ldots . N$.\\

\textbf{Solución.} La ecuación homogénea es la misma que la del oscilador del ejercicio 11. La solución es entonces, la misma.\\
$$
y(t)=c_{1} \cos (\lambda t)+c_{2} \sin (\lambda t)
$$\\

Sea $g_{m}(x)=a_{m} \sin (m \pi x)$, para $m=1, \ldots, N$. Ahora, encontramos una solución particular para cada $m$. Como $\lambda \neq m \pi$, podemos suponer que:\\

$$
y_{m}=A_{m} \sin (m \pi x)+B_{m} \cos (m \pi x) \quad \ldots(1)
$$\\

Derivando (1), se tiene que:\\

\begin{center}
   $y_{m}^{\prime}=m \pi A_{m} \cos (m \pi x)-m \pi B_{m} \sin (m \pi x)$ \\
    
    $y_{m}^{\prime \prime}=-m^{2} \pi^{2} A_{m} \sin (m \pi x)-m^{2} \pi^{2} B_{m} \cos (m \pi x)$
\end{center}

Sustituyendo en la ecuación, tenemos que:\\

$-m^{2} \pi^{2} A_{m} \sin (m \pi x)-m^{2} \pi^{2} B_{m} \cos (m \pi x)+\lambda^{2}\left(A_{m} \sin (m \pi x)+B_{m} \cos (m \pi x)\right)=a_{m} \sin (m \pi x)$\\

Así, obtenemos el siguiente sistema de ecuaciones:\\
$$
\left\{\begin{array}{ll}
-m^{2} \pi^{2} A_{m}+\lambda^{2} A_{m} & =a_{m} \\
-m^{2} \pi^{2} B_{m}+\lambda^{2} B_{m} & =0
\end{array}\right.
$$\\

Ahora, como $\lambda \neq m \pi \Rightarrow \lambda^{2} \neq m^{2} \pi^{2}$, entonces:

\begin{center}

$A_{m}=\frac{a_{m}}{\lambda^{2}-m^{2} \pi^{2}} ; \quad B_{m}=0$

\end{center}

para todo $m=1, \ldots, N$. Por lo tanto la solución particular es:\\

\begin{center}

$y_{m}=\frac{a_{m}}{\lambda^{2}-m^{2} \pi^{2}} \sin (m \pi x)$

\end{center}

Ahora, al sumar cada una de las $N$ soluciones particuales, tenemos que la solución general de la ecuación es:\\
$$
y(x)=c_{1} \cos (\lambda t)+c_{2} \sin (\lambda t)+\sum_{m=1}^{N} \frac{a_{m}}{\lambda^{2}-m^{2} \pi^{2}} \sin (m \pi x)
$$