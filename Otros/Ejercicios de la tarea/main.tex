\documentclass{article}
\usepackage[utf8]{inputenc}

\title{Ejercicios de la tarea}
\author{kr\_ipton }
\date{October 2022}

\begin{document}

\maketitle

\section{Introduction}
EVALUE las siguientes afirmaciones (determine si es verdadera o falsa, bajo qué condiciones
es verdadera o falsa, sea explícito y claro en su punto de vista; si lo cree necesario, incluya ejemplos)

\begin{itemize}
    \item En un juego, la competencia induce a equilibrios de Nash debido a que ésta es un rasgo
de racionalidad
    
        Aunque la competencia por su naturaleza egoísta para cada jugador tenemos que existe una tendencia a ir a un equilibrio de Nash sin embargo en modelo donde los jugadores puedan llegar a un acuerdo para lograr mejores ganancias no seria estrictamente necesario, aunque aquí entramos en la discusión entre si este comportamiento seria una competencia o no.
    \item En un juego, la cooperación induce a equilibrios de Nash debido a que ésta produce los mayores pagos para los jugadores.
        
        Falso, dado que la cooperación no es condición necesaria para que cada jugador elija la mejor decisión en individual, dicho de otra forma no la mejor decisión individual no es la misma que es mejor para el conjunto de jugadores, por lo tanto se tiene un caso como el dilema del prisionero donde la cooperación induce un resultado que no es el equilibrio de Nash
    \item En un juego en forma estratégica, si todos los jugadores eligen la estrategia que es mejor respuesta dentro de su conjunto de estrategias, entonces las estrategias elegidas conforman un perfil de Equilibrio de Nash
        Por definición de equilibrio de Nash esta proposición es cierta siempre y cuando se asegure que cada jugador toma su mejor decisión en individual.
    \item Una estrategia que es mejor respuesta es también una estrategia dominante
        Cierto, una estrategia al ser una mejor respuesta, necesita dominar a las demás, por lo tanto 
    \item Todos los equilibrios obtenidos de la eliminación iterativa de estrategias dominadas (EIED) conforman perfiles de mejor respuesta para cada jugador.
        Cierto, dado que por definición de equilibrio son las mejores decisiones de cada jugador
    \item El proceso de EIED elimina al menos un EN en estrategias puras (ep)
        La condición de la eliminación iterativa de estrategias implica que llegará a un equilibrio de Nash por lo que si solo existe uno en el conjunto de decisiones si se elimina no se llegaria a ninguno por lo que se llegaria a un absurdo, entonces la eliminación de equilibrios de Nash no es una condición necesaria.
    \item Para calcular EN por medio de EIED es necesario asumir a priori que existen equilibrios en el juego
        
        Para el EIED no es necesario recurrir a asumirlo dado que solo debe asegurarse que es un problema de estrategias mixtas y es de forma rectangular para asegurar la existencia de al menos un equilibrio de Nash.
        
    \item   Si en un juego solo hubiera un equilibrio de Nash en ep, necesariamente tenemos que llegar a este por medio de EIED
        Cierto, el EIED tiene la condición de que siempre llega a un equilibrio de Nash por lo que si este fuera unico el algoritmo llegaria por conclusión a dicho equilibrio.
    
    
\end{itemize}

\end{document}
