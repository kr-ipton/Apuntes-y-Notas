\section{Enuncie el Teorema de Existencia y Unicidad de Picard-Lindelöf y conteste la pregunta}
El teorema establece cuando existe solución para una ecuación diferencial y si esta es única.\\
Parte de mostrar a la Ecuación de la forma:\\
$$\frac{dy}{dx}=f(x,y) \mbox{ en cualquier }(x_0,y_0) \mbox{ y verifica:}$$
\begin{enumerate}
    \item $ \exists f(x,y) \forall x,y$ lo que asegura la existencia
    \item $\frac{\partial f}{\partial y}$ es continua para todo $x, y$ lo cual asegura la unicidad
\end{enumerate}

\begin{itemize}
    \item ¿Tiene solución el PVI $x'= \ln(tx),\mbox{ con } x(0) = 2$?
            Se tiene que la función f(x,t) es continua si se cumple que:
            $$tx>0$$
            Además la derivada parcial tal que:
            $$\frac{\partial f}{\partial x}=\frac{t}{\ln(tx)}$$
            donde también es continua tal que: $tx>0$
            
            ahora verificando que exista el punto dado en la intersección de las condiciones de continuidad tenemos
            
            $$(x,t)=(0,2)$$
            pero al exaluar $tx$ queda $0\cdot 2=0$ y por lo tanto no existe solución
                    
    \item ¿Tiene solución el PVI $x'= \ln(tx), \mbox{ con } x(1) = 2$?
            
            Dado que la ecuación es igual al inciso anterior tenemos que evaluar solamente para verificar la condición, entonces:
            
            $tx=2\cdot 1=2>0$ lo cual cumple a nuestra condición y por lo tanto tenemos que si existe solución la cual es unica
            
\end{itemize}