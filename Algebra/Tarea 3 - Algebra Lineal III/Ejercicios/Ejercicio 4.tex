\section{Estudiar si las siguientes matrices son diagonalizables sobre $\mathbb{R}$. En caso afirmativo, determinar su diagonal, una matriz de paso y los respectivos subespacios propios}

\begin{itemize}
    \item $$A=\:\begin{pmatrix}5&-1\\ 1&3\end{pmatrix}$$
    Calculando determinante:
    $$\det(A-\lambda I)= \begin{vmatrix}5-\lambda&-1\\ 1&3-\lambda\end{vmatrix}=(5-\lambda)(3-\lambda)-(1)(-1)=15-8\lambda+\lambda^2+1=\lambda^2-8\lambda+16=(\lambda-4)^2=0$$
    $$\Longrightarrow \lambda=4~~~ \mbox{con multiplicidad de 2}(\text{ma}(4)=2)$$
    Y sus vectores propios sustituyendo primero $\lambda$:
    $$A-4\lambda=\begin{pmatrix}5&-1\\ 1&3\end{pmatrix}-4\begin{pmatrix}1&0\\ 0&1\end{pmatrix}=\begin{pmatrix}5-4&-1-0\\ 1-0&3-4\end{pmatrix}=\begin{pmatrix}1&-1\\ 1&-1\end{pmatrix}$$
    Por lo que para encontrar el vector propio, resolvemos el sistema asociado con el valor propio $\lambda=4$, tal que $(A-4I)\vec{v}=(0)$, con $\vec{v}=(x,y)$
    $$\left(A-4I\right)\begin{pmatrix}x\\ y\end{pmatrix}=\begin{pmatrix}1&-1\\1&-1\end{pmatrix}\begin{pmatrix}x\\ y\end{pmatrix}=\begin{pmatrix}1(x)+(-1)y\\1(x)+(-1)y\end{pmatrix}=\begin{pmatrix}x-y\\ x-y\end{pmatrix}=\begin{pmatrix}0\\ 0\end{pmatrix}$$
    donde tenemos la siguiente condición:
    $x-y=0 \Rightarrow x=y$.\\
    Por lo que si fijamos el valor de $y$, tenemos que el subespacio propio de $\lambda=4$ lo podemos definir como:
    \[E(4)=\{(\alpha,\alpha)=\alpha(1,1)~|~\alpha\in\mathbb{R}\}\]
    Por lo que es f\'acil ver que $\langle (1,1) \rangle=E(4)$, por lo que $\dim(E(4))=1$.\\
    Pero recordemos que $\text{ma}(4)=2$, entonces tenemos que:
    \[\dim(E(4))=1\neq2=\text{ma}(4) \]
    Por lo que se incumple la segunda condici\'on del \textbf{Teorema 23} visto en clase, entonces: \textsc{La matriz no es diagonalizable}
    
    \item $$B=\begin{pmatrix}2&1&1\\ 2&3&2\\ 1&1&2\end{pmatrix}$$
    
    Calculando determinante por cofactores:
    $$\det(B-\lambda I)= \begin{vmatrix}2-\lambda &1&1\\ 2&3-\lambda &2\\ 1&1&2-\lambda \end{vmatrix}=(2-\lambda)\begin{vmatrix}3-\lambda &2\\ 1&2-\lambda \end{vmatrix}-(1)\begin{vmatrix} 2&2\\ 1&2-\lambda \end{vmatrix}+(1)\begin{vmatrix} 2&3-\lambda \\ 1&1 \end{vmatrix}$$
    \[(2-\lambda)[(3-\lambda)(2-\lambda)-(1)(2)]-[(2)(2-\lambda)-(1)(2)]+[(2)(1)-(1)(3-\lambda)]=(2-\lambda)[6-5\lambda+\lambda^2-2]-[4-2\lambda-2]+[2+\lambda-3]\]\[=(2-\lambda)[4-5\lambda+\lambda^2]-[2-2\lambda]+[\lambda-1]=8-10\lambda+2\lambda^2-4\lambda+5\lambda^2-\lambda^3-2+2\lambda+\lambda-1\]
     $$=-\lambda ^3+7\lambda ^2-11\lambda +5=-\lambda(\lambda^2-6\lambda+5)+1(\lambda^2-6\lambda+5)=(1-\lambda)(\lambda^2-6\lambda+5)=(1-\lambda)(1-\lambda)(5-\lambda)=0$$
    
    $\Longrightarrow \lambda_1=\lambda_2=1\mathrm{\:con\:multiplicidad\:de\:}2$ (\text{ma}(1)=2) y $\lambda_3=5$ (\text{ma}(1)=2).\\
    
    Encontrando los vectores propios de cada valor propio:
    \begin{itemize}
        \item Para $\lambda=1$ tenemos que sustituyendo primero $\lambda=1$:
    $$B-1I=\begin{pmatrix}2&1&1\\ 2&3&2\\ 1&1&2\end{pmatrix}-1\cdot \begin{pmatrix}1&0&0\\ 0&1&0\\ 0&0&1\end{pmatrix}=\begin{pmatrix}1&1&1\\ 2&2&2\\ 1&1&1\end{pmatrix}$$
    Por lo que para encontrar el vector propio, resolvemos el sistema asociado con el valor propio $\lambda=1$, tal que $(B-1I)\vec{v}=(0)$, con $\vec{v}=(x,y,z)$
    Lo que haremos ser\'a simplificar la matriz primero para eliminar los renglones linealmente dependientes. Intercambiando $R_2 \leftrightarrow R_1$    $$=\begin{pmatrix}2&2&2\\ 1&1&1\\ 1&1&1\end{pmatrix}$$    Sea $R_2=R_2-\frac{R_1}{2}$   $$=\begin{pmatrix}2&2&2\\ 0&0&0\\ 1&1&1\end{pmatrix}$$    Sea $R_3=R_3-\frac{R_1}{2}$    $$\begin{pmatrix}2&2&2\\ 0&0&0\\ 0&0&0\end{pmatrix}$$    Y finalmente $R_1=\frac{R_1}{2}$    $$\begin{pmatrix}1&1&1\\ 0&0&0\\ 0&0&0\end{pmatrix}$$    Y así el sistema asociado a $\lambda=1$ queda simplificado a:    $$\begin{pmatrix}1&1&1\\ 0&0&0\\ 0&0&0\end{pmatrix}\begin{pmatrix}x\\ y\\ z\end{pmatrix}=\begin{pmatrix}1(x)+1(y)+1(z)\\ 0(x)+0(y)+0(z)\\0(x)+0(y)+0(z)\end{pmatrix}=\begin{pmatrix}x+y+z\\ 0\\0\end{pmatrix}=\begin{pmatrix}0\\ 0\\ 0\end{pmatrix}$$
    Por lo que hemos obtenido que $x+y+z=0$, despejando, tenemos $x=-y-z$, y como no hay otras restricciones, tenemos que $y$ y $z$ son variables libres, por lo que tenemos que el subespacio propio de $\lambda=1$ es:
    \[E(1)=\{(-\alpha-\beta,\alpha, \beta)~|~\alpha,\beta\in\mathbb{R}\}=\{\alpha(-1,1,0)+\beta(-1,0,1)~|~\alpha,\beta\in\mathbb{R}\}\]
    De este modo podemos ver que $(-1,1,0)$ y $(-1,0,1)$ son base del subespacio, pues
    \[(-\alpha-\beta,\alpha, \beta)=\alpha(-1,1,0)+\beta(-1,0,1)=(0,0,0)\]
    Por lo que podemos darnos cuenta que generan a $E(1)$ y adem\'as son linealmente depedientes (pues la \'unica combinaci\'on nula es la trivial), de este modo, tenemos que $\text{dim}(E(1))=2$., por lo que podemos fijar estos vectores propios como: $$\begin{pmatrix}-1\\ 1\\ 0\end{pmatrix}~~~~\text{y}~~~~\begin{pmatrix}-1\\ 0\\ 1\end{pmatrix}$$
    
    \item Para $\lambda=5$ tenemos que sustituyendo primero $\lambda=5$:
    $$B-5I=\begin{pmatrix}2&1&1\\ 2&3&2\\ 1&1&2\end{pmatrix}-5\cdot \begin{pmatrix}1&0&0\\ 0&1&0\\ 0&0&1\end{pmatrix}=\begin{pmatrix}-3&1&1\\ 2&-2&2\\ 1&1&-3\end{pmatrix}$$
    Por lo que para encontrar el vector propio, resolvemos el sistema asociado con el valor propio $\lambda=5$, tal que $(B-5I)\vec{v}=(0)$, con $\vec{v}=(x,y,z)$.\\
    Lo que haremos ser\'a simplificar la matriz primero para eliminar los renglones linealmente dependientes.\\
    
    Sea $R_2=R_2+\frac{2R_1}{3}$
    $$\begin{pmatrix}-3&1&1\\ 0&-\frac{4}{3}&\frac{8}{3}\\ 1&1&-3\end{pmatrix}$$
    Sea $R_3=R_3+\frac{R_1}{3}$
    $$\begin{pmatrix}-3&1&1\\ 0&-\frac{4}{3}&\frac{8}{3}\\ 0&\frac{4}{3}&-\frac{8}{3}\end{pmatrix}$$
    Sea $R_3=R_3+R_2$
    $$\begin{pmatrix}-3&1&1\\ 0&-\frac{4}{3}&\frac{8}{3}\\ 0&0&0\end{pmatrix}$$
    Sea $R_2=-\frac{3R_2}{4}$
    $$\begin{pmatrix}-3&1&1\\ 0&1&-2\\ 0&0&0\end{pmatrix}$$
    Sea $R_1=R_1-R_2$
    $$\begin{pmatrix}-3&0&3\\ 0&1&-2\\ 0&0&0\end{pmatrix}$$
    Sea $R_1=-\frac{R_1}{3}$
    $$\begin{pmatrix}1&0&-1\\ 0&1&-2\\ 0&0&0\end{pmatrix}$$
    Y el sistema asociado a $\lambda=5$ nos quedar\'ia:
    $$\begin{pmatrix}1&0&-1\\ 0&1&-2\\ 0&0&0\end{pmatrix}\begin{pmatrix}x\\ y\\ z\end{pmatrix}=\begin{pmatrix}1(x)+0(y)+(-1)z\\ 0(x)+1(y)+(-2)z\\0(x)+0(y)+0(z)\end{pmatrix}=\begin{pmatrix}x-z\\ y-2z\\ 0\end{pmatrix}=\begin{pmatrix}0\\ 0\\ 0\end{pmatrix}$$
    Por lo que tenemos el sistema:
    \begin{eqnarray*}
    x-z&=&0\\
    y-2z&=&0
    \end{eqnarray*}
    Despejando, tenemos que $x=z$ y que $2z=y$, y como no hay otras restricciones, tenemos que $z$ es una variable libre, por lo que tenemos que el subespacio propio de $\lambda=5$ es:
    \[E(5)=\{(\alpha,2\alpha,\alpha)~|~\alpha\in\mathbb{R}\}=\{\alpha(1,2,1)~|~\alpha,\beta\in\mathbb{R}\}\]
    De este modo podemos ver que $(1,2,1)$ es base del subespacio $E(5)$, de este modo, tenemos que $\text{dim}(E(5))=1$, por lo que podemos fijar este vector propios como:
    $$\begin{pmatrix}1\\ 2\\ 1\end{pmatrix}$$
    
    \end{itemize}
    De este modo por el \textbf{Teorema 23} visto en clase, tenemos que cumple las condiciones de que $\lambda_1,\lambda_2,\lambda_3\in K=\mathbb{R}$ y tenemos que como:
    \[\text{dim}(E(1))=2=\text{ma}(1)~~~~\text{y}~~~~\text{dim}(E(5))=1=\text{ma}(5)\]
    por lo tanto podemos obtener la matriz diagonal $\mathcal{D}$ en la cual sus elementos en la diagonal son los valores propios en el orden dado, es decir $\lambda_1=\lambda_2=1$ y $\lambda_3=5$, por tanto tenemos que la matriz diagonal $\mathcal{D}$ es:
    $$\mathcal{D}=\begin{pmatrix}1&0&0\\ 0&1&0\\ 0&0&5\end{pmatrix}$$
   Por lo que podemos definir a la matriz de paso $P$ como los vectores propios asociados a cada valor propio en el orden establecido, es decir:
    $$P=\begin{pmatrix}-1&-1&1\\ 1&0&2\\ 0&1&1\end{pmatrix}$$
    
    
    \item
    $$C=\begin{pmatrix}1&-3&3\\ 3&-5&3\\ 6&-6&4\end{pmatrix}$$
    Calculando el determinante:
    $$\det(C-\lambda I)= \begin{vmatrix}1-\lambda &-3&3\\ 3&-5-\lambda &3\\ 6&-6&4-\lambda \end{vmatrix}=(1-\lambda)\begin{vmatrix} -5-\lambda &3\\
    -6&4-\lambda \end{vmatrix}-(-3)\begin{vmatrix} 3 &3\\ 6&4-\lambda \end{vmatrix}+3\begin{vmatrix} 3&-5-\lambda \\ 6&-6 \end{vmatrix}$$
    \[=(1-\lambda)[(-5-\lambda)(4-\lambda)-(-6)(3)]+3[(3)(4-\lambda)-(6)(3)]+3[(3)(-6)-(6)(-5-\lambda)]\]\[=(1-\lambda)[-20+\lambda+\lambda^2+18]+3[12-3\lambda-18]+3[-18+30+6\lambda]=(1-\lambda)[-2+\lambda+\lambda^2]+3[-3\lambda-6]+3[12+6\lambda]\]\[=-2+\lambda+\lambda^2+2\lambda-\lambda^2-\lambda^3-9\lambda-18+36+18\lambda\]$$=-\lambda ^3+12\lambda +16=0$$Factorizando:
    \[-\lambda ^3+12\lambda +16
=-\lambda ^3+2\lambda^2+8\lambda-2\lambda^2+4\lambda +16=-\lambda(\lambda^2-2\lambda-8)-2(\lambda^2-2\lambda-8)=(-2-\lambda)(\lambda^2-2\lambda-8)\]\[=(-2-\lambda)(-2-\lambda)(4-\lambda)=(-2-\lambda)^2(4-\lambda)=0\]
    
    $$ \Longrightarrow \lambda_1=\lambda_2 =-2~~\mathrm{\:con\:multiplicidad\:de\:}2~(\text{ma}(-2)=2),~~~\:\lambda_3 =4$$
    Encontrando los vectores propios para cada valor propio
    \begin{itemize}
        \item Para $\lambda=-2$ tenemos que susituyendo primero $\lambda=-2$
    $$C-\lambda I=\begin{pmatrix}1&-3&3\\ 3&-5&3\\ 6&-6&4\end{pmatrix}-\left(-2\right)\begin{pmatrix}1&0&0\\ 0&1&0\\ 0&0&1\end{pmatrix}=\begin{pmatrix}3&-3&3\\ 3&-3&3\\ 6&-6&6\end{pmatrix}$$
    Por lo que para encontrar el vector propio, resolvemos el sistema asociado con el valor propio $\lambda=-2$, tal que $(C-(-2)I)\vec{v}=(0)$, con $\vec{v}=(x,y,z)$.\\
    Lo que haremos ser\'a simplificar la matriz primero para eliminar los renglones linealmente dependientes.
    Intercambiando $R_1 \leftrightarrow R_3$
    $$\begin{pmatrix}6&-6&6\\ 3&-3&3\\ 3&-3&3\end{pmatrix}$$
    Sea $R_{2,3}=R_{2,3}-\frac{R_1}{2}$
    $$\begin{pmatrix}6&-6&6\\ 0&0&0\\ 0&0&0\end{pmatrix}$$
    Sea $R_1=\frac{R_1}{6}$
    $$\begin{pmatrix}1&-1&1\\ 0&0&0\\ 0&0&0\end{pmatrix}$$
    Y as\'i el sistema asociado a $\lambda=-2$ queda simplificado como:
    $$\begin{pmatrix}1&-1&1\\ 0&0&0\\ 0&0&0\end{pmatrix}\begin{pmatrix}x\\ y\\ z\end{pmatrix}=\begin{pmatrix}1(x)+(-1)(y)+1(z)\\ 0(x)+0(y)+0(z)\\ 0(x)+0(y)+0(z)\end{pmatrix}=\begin{pmatrix}x-y+z\\ 0\\ 0\end{pmatrix}=\begin{pmatrix}0\\ 0\\ 0\end{pmatrix}$$
    Por lo que tenemos que $x-y+z=0$, por lo que $x=y-z$, pero como no hay restricciones para $y$, ni para $z$, entonces son variables libres, por lo que podemos definir al subespacio propio de $\lambda=-2$ como:
    \[E(-2)=\{(\alpha-\beta,\alpha,\beta)~|~\alpha,\beta\in\mathbb{R}\}=\{\alpha(1,1,0)+\beta(-1,0,1)~|~\alpha,\beta\in\mathbb{R}\}\]
    Por lo que $(1,1,0)$ y $(-1,0,1)$ son una base para $E(-2)$, pues:
    \[(\alpha-\beta,\alpha,\beta)=\alpha(1,1,0)+\beta(-1,0,1)=(0,0,0)\]
    Por lo que podemos darnos cuenta que generan a $E(-2)$ y adem\'as son linealmente independientes (pues la \'unica combinaci\'on nula es la trivial), de este modo, tenemos que $\text{dim}(E(-2))=2$, por lo que podemos fijar estos vectores propios como:
    $$\begin{pmatrix}1\\ 1\\ 0\end{pmatrix}~~~\text{y}~~~\begin{pmatrix}-1\\ 0\\ 1\end{pmatrix}$$
    
    \item Para $\lambda=4$ tenemos que sustituyendo primero en $\lambda=4$
    $$C-\lambda I=\begin{pmatrix}1&-3&3\\ 3&-5&3\\ 6&-6&4\end{pmatrix}-4\begin{pmatrix}1&0&0\\ 0&1&0\\ 0&0&1\end{pmatrix}=\begin{pmatrix}-3&-3&3\\ 3&-9&3\\ 6&-6&0\end{pmatrix}$$
    Por lo que para encontrar el vector propio, resolvemos el sistema asociado con el valor propio $\lambda=4$, tal que $(C-4I)\vec{v}=(0)$, con $\vec{v}=(x,y,z)$.\\
    Lo que haremos ser\'a simplificar la matriz primero para eliminar los renglones linealmente dependientes.
    Intercambiando $R_1 \leftrightarrow R_3$
    $$\begin{pmatrix}6&-6&0\\ 3&-9&3\\ -3&-3&3\end{pmatrix}$$
    Sea $R_2-\frac{R_1}{2}$
    $$\begin{pmatrix}6&-6&0\\ 0&-6&3\\ -3&-3&3\end{pmatrix}$$
    Sea $R_3+\frac{R_1}{2}$
    $$\begin{pmatrix}6&-6&0\\ 0&-6&3\\ 0&-6&3\end{pmatrix}$$
    Sea $R_3=R_3-R_2$
    $$\begin{pmatrix}6&-6&0\\ 0&-6&3\\ 0&0&0\end{pmatrix}$$
    Sea $R_2=-\frac{R_2}{6}$
    $$\begin{pmatrix}6&-6&0\\ 0&1&-\frac{1}{2}\\ 0&0&0\end{pmatrix}$$
    Sea $R_1=R_1+6R_2$
    $$\begin{pmatrix}6&0&-3\\ 0&1&-\frac{1}{2}\\ 0&0&0\end{pmatrix}$$
    Sea $R_1=\frac{R_1}{6}$
    $$\begin{pmatrix}1&0&-\frac{1}{2}\\ 0&1&-\frac{1}{2}\\ 0&0&0\end{pmatrix}$$
    El sistema asociado con $\lambda=4$ nos queda simplificado como:
    $$\begin{pmatrix}1&0&-\frac{1}{2}\\ 0&1&-\frac{1}{2}\\ 0&0&0\end{pmatrix}\begin{pmatrix}x\\ y\\ z\end{pmatrix}=\begin{pmatrix}1(x)+0(y)+-\frac{1}{2}z\\ 0(x)+1(y)+-\frac{1}{2}z\\0(x)+0(y)+0(z)\end{pmatrix}=\begin{pmatrix}x-\frac{1}{2}z\\ y-\frac{1}{2}z\\ 0\end{pmatrix}=\begin{pmatrix}0\\ 0\\ 0\end{pmatrix}$$
    Por lo que tenemos el sistema:
    \begin{eqnarray*}
    x-\frac{1}{2}z&=&0\\
    y-\frac{1}{2}z&=&0
    \end{eqnarray*}
    Despejando, tenemos que $x=\frac{1}{2}z$ y que $y=\frac{1}{2}z$, por lo que $x=y$ o que $z=2x=2y$, y como no hay otras restricciones, tenemos que $z$ es una variable libre, por lo que tenemos que el subespacio propio de $\lambda=4$ es:
    \[E(4)=\{(2\alpha,2\alpha,\alpha)~|~\alpha\in\mathbb{R}\}=\{\alpha(2,2,1)~|~\alpha,\beta\in\mathbb{R}\}\]
    De este modo podemos ver que $(2,2,1)$ es base del subespacio $E(4)$, de este modo, tenemos que $\text{dim}(E(4))=1$, por lo que podemos fijar este vector propios como:
    $$\begin{pmatrix}2\\ 2\\ 1\end{pmatrix}$$
    \end{itemize}
    De este modo por el \textbf{Teorema 23} visto en clase, tenemos que cumple las condiciones de que $\lambda_1,\lambda_2,\lambda_3\in K=\mathbb{R}$ y tenemos que como:
    \[\text{dim}(E(-2))=2=\text{ma}(-2)~~~~\text{y}~~~~\text{dim}(E(4))=1=\text{ma}(4)\]
    por lo tanto podemos obtener la matriz diagonal $\mathcal{D}$ en la cual sus elementos en la diagonal son los valores propios en el orden dado, es decir $\lambda_1=\lambda_2=-2$ y $\lambda_3=4$, por tanto tenemos que la matriz diagonal $\mathcal{D}$ es:
    $$\mathcal{D}==\begin{pmatrix}-2&0&0\\ 0&-2&0\\ 0&0&4\end{pmatrix}$$
   Por lo que podemos definir a la matriz de paso $P$ como los vectores propios asociados a cada valor propio en el orden establecido, es decir:
    $$P=\begin{pmatrix}1&-1&1\\ 1&0&1\\ 0&1&2\end{pmatrix}$$
    
    \item 
    $$D=\begin{pmatrix}3&1&0\\ 0&3&1\\ 0&0&3\end{pmatrix}$$
    Calculando determinante, que al ser una matriz triangular superior, sabemos que es multiplicar los elementos de la diagonal, de forma que:
    $$\det(D-\lambda I)= \begin{vmatrix}3-\lambda &1&0\\ 0&3-\lambda &1\\ 0&0&3-\lambda \end{vmatrix}=(3-\lambda)(3-\lambda)(3-\lambda)=-\lambda ^3+9\lambda ^2-27\lambda +27=0$$
    $\Longrightarrow \lambda=3~~\mathrm{\:con\:multiplicidad\:de\:}3$ ($\text{ma}(3)=3$).\\
    Por lo que para verificar que sea diagonizable, debemos encontrar el subespacio propio para $\lambda=3$, por lo que resolvemos el sistema asociado con el valor propio $\lambda=3$, tal que $(D-3I)\vec{v}=(0)$, con $\vec{v}=(x,y,z)$:
    $$\left(A-3I\right)\vec{v}=\left[\begin{pmatrix}3&1&0\\ 0&3&1\\ 0&0&3\end{pmatrix}-3\begin{pmatrix}1&0&0\\ 0&1&0\\ 0&0&1\end{pmatrix}\right]\begin{pmatrix}x\\ y\\ z\end{pmatrix}=\begin{pmatrix}0&1&0\\ 0&0&1\\ 0&0&0\end{pmatrix}\begin{pmatrix}x\\ y\\ z\end{pmatrix}=\begin{pmatrix}0(x)+1(y)+0(z)\\ 0(x)+0(y)+1(z)\\0(x)+0(y)+0(z)\end{pmatrix}=\begin{pmatrix}y\\ z\\ 0\end{pmatrix}=\begin{pmatrix}0\\ 0\\ 0\end{pmatrix}$$
    Por lo que tenemos el sistema:
    \begin{eqnarray*}
    y&=&0\\
    z&=&0
    \end{eqnarray*}
    Por lo que si fijamos el valor de $x$, ya que no se tiene ninguna restricci\'on, tenemos que el subespacio propio de $\lambda=3$ lo podemos definir como:
    \[E(3)=\{(\alpha,0,0)=\alpha(1,0,0)~|~\alpha\in\mathbb{R}\}\]
    Por lo que es f\'acil ver que $\langle (1,0,0) \rangle=E(3)$, por lo que $\dim(E(3))=1$.\\
    Pero recordemos que $\text{ma}(3)=3$, entonces tenemos que:
    \[\dim(E(3))=1\neq3=\text{ma}(3) \]
    Por lo que se incumple la segunda condici\'on del \textbf{Teorema 23} visto en clase, entonces: \textsc{La matriz no es diagonalizable}
\end{itemize}